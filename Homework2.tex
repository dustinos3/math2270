\documentclass[10pt,twoside,reqno]{article}
\usepackage[marginparsep=1em]{geometry}
\geometry{lmargin=1.0in,rmargin=1.0in, bmargin=0.75in,  tmargin=0.75in}
\usepackage[usenames,dvipsnames,svgnames,table]{xcolor}
\usepackage{graphicx}
\usepackage{amssymb}
\usepackage{epstopdf}
\usepackage{tikz}
\usepackage{enumerate}
\usepackage{amsthm}
 \usepackage{pgfplots}
 \usepackage{tikz-3dplot}
 \usetikzlibrary{shapes.geometric}
 \usepackage{float}
\usepackage{amsmath}
\usepackage{fancyhdr}
\usepackage{lmodern}
\usepackage{chngcntr}
\usepackage{multicol, comment}

\pagestyle{fancy}
\fancyhf{}
\renewcommand{\sectionmark}[1]{\markright{\thesection.\ #1}}
\lhead{\fancyplain{}{}} 
\fancyhead[RE,RO]{MATH 2270}
\fancyfoot[RE,LO]{Dr. Heavilin}
\fancyfoot[LE,RO]{\thepage}


\begin{document}
\begin{flushright}
\begin{minipage}{.25\textwidth}
Dustin Ginos: \\
A01233669\\
Chandler Kinch: \\
A01662772\\
Jeff Wasden: \\
A01657029\\

\today
\end{minipage}
\end{flushright}

\center{\textbf{\underline{Homework 2}}}\\
\vspace{5mm}
\textbf{Chapter 2.1}
\begin{enumerate}
\item[2.1.9]Compute each $Ax$ by dot products of the rows with the column vector: \\
\vspace{3mm}
$
$$
\hspace{85pt}
(a) 
\begin{bmatrix}
1&2&4\\
-2&3&1\\
-4&1&2\\
\end{bmatrix}
\begin{bmatrix}
2\\
2\\
3\\
\end{bmatrix}
\hspace{45pt}
(b) 
\begin{bmatrix}
2&1&0&0\\
1&2&1&0\\
0&1&2&1\\
0&0&1&2\\
\end{bmatrix}
\begin{bmatrix}
1\\
1\\
1\\
2\\
\end{bmatrix}
$$
$\\
\vspace{3mm}
%~~~~~~~~~~~~~~~~~~~~~ ANSWER TO 2.1.9 ~~~~~~~~~~~~~~~~~~~~~~~~
\item[2.1.10]Compute each Ax in Problem 9 as a combination of the columns: \\
\vspace{3mm}
\hspace{15pt}
9(a) becomes
$
$$
\hspace{15pt}
Ax=2
\begin{bmatrix}
1\\
-2\\
-4\\
\end{bmatrix}
+2
\begin{bmatrix}
2\\
3\\
1\\
\end{bmatrix}
+3
\begin{bmatrix}
4\\
1\\
2\\
\end{bmatrix}
=
\begin{bmatrix}
2+4+12\\
-4+6+3\\
-8+2+6\\
\end{bmatrix}
=
\begin{bmatrix}
18\\
5\\
0\\
\end{bmatrix}
.
$$
$
\\
\vspace{3mm}
How many separate multiplications for $Ax$, when the matrix is "3 by 3"?\\
%~~~~~~~~~~~~~~~~~~~~~ ANSWER TO 2.1.10 ~~~~~~~~~~~~~~~~~~~~~~~~
\item[2.1.12]Multiply $A$ times $x$ to find three components of $Ax$ : \\
\vspace{3mm}
\hspace{45pt}
$
$$
\begin{bmatrix}
2&3\\
5&1\\
\end{bmatrix}
\begin{bmatrix}
4\\
2\\
\end{bmatrix}
$$
$
\hspace{15pt}
and
\hspace{15pt}
$
$$
\begin{bmatrix}
3&6\\
6&12\\
\end{bmatrix}
\begin{bmatrix}
2\\
-1\\
\end{bmatrix}
$$
$
\hspace{15pt}
and
\hspace{15pt}
$
$$
\begin{bmatrix}
1&2&4\\
2&0&1\\
\end{bmatrix}
\begin{bmatrix}
3\\
1\\
1\\
\end{bmatrix}
$$
$
\\
\vspace{3mm}
%~~~~~~~~~~~~~~~~~~~~~ ANSWER TO 2.1.12 ~~~~~~~~~~~~~~~~~~~~~~~~
\item[2.1.17]Find the matrix $P$ that multiplies $(x, y, z)$ to give $(y, z, x)$. Find the matrix $Q$ that multiplies $(y, z, x)$ to bring back $(x, y, z)$. \\
\vspace{3mm}
%~~~~~~~~~~~~~~~~~~~~~ ANSWER TO 2.1.17 ~~~~~~~~~~~~~~~~~~~~~~~~
\item[2.1.19]What 3 by 3 matrix $E$ multiplies $(x, y, z)$ to give $(x, y, z + x)$? What matrix $E^{-1}$ multiplies $(x, y, z)$ to give $(x, y, z - x)$? If you multiply $(3,4,5)$ by $E$ and then multiply by $E^{-1}$, the two results are $(\underline{\hspace{8mm}})$ and $(\underline{\hspace{8mm}})$. \\
\vspace{3mm}
%~~~~~~~~~~~~~~~~~~~~~ ANSWER TO 2.1.19 ~~~~~~~~~~~~~~~~~~~~~~~~
\item[2.1.33]\textbf{Multiplying by $A$ is a "linear transformation"}. Those important words mean: If $w$ is a combination of $u$ and $v$, then $Aw$ is the same combination of $Au$ and $Av$. It is this "\textbf{\textit{linearity}}" $Aw = cAu + dAv$ that gives us the name \textit{linear algebra}. Problem: If 
$
$$
u=
\begin{bmatrix}
1\\
0\\
\end{bmatrix}
$$
$
 and 
$
$$
v=
\begin{bmatrix}
0\\
1\\
\end{bmatrix}
$$
$
 then $Au$ and $Av$ are the columns of $A$. Combine $w = cu + dv$. \textbf{If 
$
$$
\pmb{w=
\begin{bmatrix}
5\\
7\\
\end{bmatrix}}
$$
$
 how is $\pmb{Aw}$ connected to $\pmb{Au}$ and $\pmb{Av}$?} \\
\end{enumerate}
\vspace{3mm}
%~~~~~~~~~~~~~~~~~~~~~ ANSWER TO 2.1.19 ~~~~~~~~~~~~~~~~~~~~~~~~
\vspace{5mm}
\textbf{Chapter 2.2}
\begin{enumerate}
\item[2.2.5]Choose a right side which gives no solution and another right side which gives infinitely many solutions. What are two of those solutions? \\
\vspace{3mm}
\hspace{100pt}
\textbf{Singular system}
\hspace{45pt}
$3x + 2y = 10$\\
\hspace{232pt}
$6x + 4y =$\\
\vspace{3mm}
%~~~~~~~~~~~~~~~~~~~~~ ANSWER TO 2.2.5 ~~~~~~~~~~~~~~~~~~~~~~~~
\item[2.2.6]Choose $a$ coefficient $b$ that makes this system singular. Then choose a right side $g$ that makes it solvable. Find two solutions in that singular case. \\
$$2x + by = 16$$ $$4x + 8y = g$$.
%~~~~~~~~~~~~~~~~~~~~~ ANSWER TO 2.2.6 ~~~~~~~~~~~~~~~~~~~~~~~~
\item[2.2.11](Recommended) A system of linear equations can't have exactly two solutions. \textit{Why}? \\
\vspace{3mm}
\hspace{10pt}(a) If $(x, y, z)$ and $(X, Y, Z)$ are two solutions, what is another solution?\\
\vspace{3mm}
%~~~~~~~~~~~~~~~~~~~~~ ANSWER TO 2.2.11 (a) ~~~~~~~~~~~~~~~~~~~~~~~~
\hspace{10pt}(b) If 25 planes meet at two points, where else do they meet? 
\vspace{3mm}
%~~~~~~~~~~~~~~~~~~~~~ ANSWER TO 2.2.11 (b) ~~~~~~~~~~~~~~~~~~~~~~~~
\item[2.2.14]Which number $d$ forces a row exchange, and what is the triangular system (not singular) for that $d$? Which $d$ makes this system singular (no third pivot)?
$$2x + 5y + z = 0$$ $$4x + dy + z = 2$$ $$y -z = 3$$. 
%~~~~~~~~~~~~~~~~~~~~~ ANSWER TO 2.2.14 ~~~~~~~~~~~~~~~~~~~~~~~~
\item[2.2.18]Construct a 3 by 3 example that has 9 different coefficients on the left side, but rows 2 and 3 become zero in elimination. How many solutions to your system with $\pmb{b} = (1,10,100)$ and how many with $\pmb{b} = (0,0,0)$? 
\vspace{3mm}
%~~~~~~~~~~~~~~~~~~~~~ ANSWER TO 2.2.18 ~~~~~~~~~~~~~~~~~~~~~~~~
\item[2.2.21]Find the pivots and the solution for both systems ($Ax = b$ and $Kx = b$): \\
\hspace{122pt}$2x+ y      =0$\hspace{60pt}$2x- y      =0$\\
\hspace{105pt}$ x+2y+ z   =0$\hspace{35pt}$-x+2y- z   =0$\\
\hspace{107pt}$    y+2z+ t=0$\hspace{46pt}$    y+2z- t=0$\\
\hspace{125pt}$       z+2t=5$\hspace{55pt}$     - z+2t=5$.\\
\vspace{3mm}
%~~~~~~~~~~~~~~~~~~~~~ ANSWER TO 2.2.21 ~~~~~~~~~~~~~~~~~~~~~~~~
\item[2.2.25]For which three numbers $a$ will elimination fail to give three pivots? \\
\vspace{3mm}
\hspace{95pt}
$
$$
A=
\begin{bmatrix}
a&2&3\\
a&a&4\\
a&a&a\\
\end{bmatrix}
$$
$
is singular for three values of $a$.
\vspace{3mm}
%~~~~~~~~~~~~~~~~~~~~~ ANSWER TO 2.2.25 ~~~~~~~~~~~~~~~~~~~~~~~~
\end{enumerate}
\vspace{5mm}
\textbf{Chapter 2.3}
\begin{enumerate}
\item[2.3.3]
\item[2.3.7]
\item[2.3.10]
\item[2.3.17]
\item[2.3.18]
\item[2.3.19]
\item[2.3.21]
\item[2.3.24]
\end{enumerate}
\vspace{5mm}
\textbf{Chapter 2.4}
\begin{enumerate}
\item[2.4.5]
\item[2.4.6]
\item[2.4.13]
\item[2.4.14]
\item[2.4.22]
\item[2.4.23]
\item[2.4.29]
\end{enumerate}
\vspace{5mm}
\textbf{Chapter 2.5}
\begin{enumerate}
\item[2.5.7]
\item[2.5.13]
\item[2.5.29]
\item[2.5.30]
\item[2.5.31]
\end{enumerate}
\vspace{5mm}
\textbf{Chapter 2.6}
\begin{enumerate}
\item[2.6.7]
\item[2.6.8]
\item[2.6.13]
\item[2.6.16]
\end{enumerate}
\vspace{5mm}
\textbf{Chapter 2.7}
\begin{enumerate}
\item[2.7.3]
\item[2.7.6]
\item[2.7.16]
\item[2.7.31]
\item[2.7.40]
\end{enumerate}
\end{document}