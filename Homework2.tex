\documentclass[10pt,twoside,reqno]{article}
\usepackage[marginparsep=1em]{geometry}
\geometry{lmargin=1.0in,rmargin=1.0in, bmargin=0.75in,  tmargin=0.75in}
\usepackage[usenames,dvipsnames,svgnames,table]{xcolor}
\usepackage{graphicx}
\usepackage{amssymb}
\usepackage{epstopdf}
\usepackage{tikz}
\usepackage{enumerate}
\usepackage{amsthm}
 \usepackage{pgfplots}
 \usepackage{tikz-3dplot}
 \usetikzlibrary{shapes.geometric}
 \usepackage{float}
\usepackage{amsmath}
\usepackage{fancyhdr}
\usepackage{lmodern}
\usepackage{chngcntr}
\usepackage{multicol, comment}

\pagestyle{fancy}
\fancyhf{}
\renewcommand{\sectionmark}[1]{\markright{\thesection.\ #1}}
\lhead{\fancyplain{}{}} 
\fancyhead[RE,RO]{MATH 2270}
\fancyfoot[RE,LO]{Dr. Heavilin}
\fancyfoot[LE,RO]{\thepage}


\begin{document}
\begin{flushright}
\begin{minipage}{.25\textwidth}
Dustin Ginos: \\
A01233669\\
Chandler Kinch: \\
A01662772\\
Jeff Wasden: \\
A01657029\\

\today
\end{minipage}
\end{flushright}

\center{\textbf{\underline{Homework 2}}}\\
\vspace{5mm}
\textbf{Chapter 2.1}
\begin{enumerate}
\item[2.1.9]Compute each $Ax$ by dot products of the rows with the column vector: \\
\vspace{3mm}
$
$$
\hspace{85pt}
(a) 
\begin{bmatrix}
1&2&4\\
-2&3&1\\
-4&1&2\\
\end{bmatrix}
\begin{bmatrix}
2\\
2\\
3\\
\end{bmatrix}
\hspace{45pt}
(b) 
\begin{bmatrix}
2&1&0&0\\
1&2&1&0\\
0&1&2&1\\
0&0&1&2\\
\end{bmatrix}
\begin{bmatrix}
1\\
1\\
1\\
2\\
\end{bmatrix}
$$
$\\
\vspace{3mm}
%~~~~~~~~~~~~~~~~~~~~~ ANSWER TO 2.1.9 ~~~~~~~~~~~~~~~~~~~~~~~~
$
$$
\hspace{85pt}
(a) 
\begin{bmatrix}
1&2&4\\
-2&3&1\\
-4&1&2\\
\end{bmatrix}
\begin{bmatrix}
2\\
2\\
3\\
\end{bmatrix}
=
\begin{bmatrix}
18\\
5\\
0\\
\end{bmatrix}
\hspace{45pt}
(b) 
\begin{bmatrix}
2&1&0&0\\
1&2&1&0\\
0&1&2&1\\
0&0&1&2\\
\end{bmatrix}
\begin{bmatrix}
1\\
1\\
1\\
2\\
\end{bmatrix}
= 
\begin{bmatrix}
3\\
4\\
5\\
5\\
\end{bmatrix}
$$
$\\
a) 
\item[2.1.10]Compute each Ax in Problem 9 as a combination of the columns: \\
\vspace{3mm}
\hspace{15pt}
9(a) becomes
$
$$
\hspace{15pt}
Ax=2
\begin{bmatrix}
1\\
-2\\
-4\\
\end{bmatrix}
+2
\begin{bmatrix}
2\\
3\\
1\\
\end{bmatrix}
+3
\begin{bmatrix}
4\\
1\\
2\\
\end{bmatrix}
=
\begin{bmatrix}
2+4+12\\
-4+6+3\\
-8+2+6\\
\end{bmatrix}
=
\begin{bmatrix}
18\\
5\\
0\\
\end{bmatrix}
$$
$
\\
\vspace{3mm}
How many separate multiplications for $Ax$, when the matrix is "3 by 3"?\\
\vspace{3mm}
%~~~~~~~~~~~~~~~~~~~~~ ANSWER TO 2.1.10 ~~~~~~~~~~~~~~~~~~~~~~~~
9 seperate multiplications
\item[2.1.12]Multiply $A$ times $x$ to find three components of $Ax$ : \\
\vspace{3mm}
\hspace{45pt}
$
$$
\begin{bmatrix}
2&3\\
5&1\\
\end{bmatrix}
\begin{bmatrix}
4\\
2\\
\end{bmatrix}
$$
$
\hspace{15pt}
and
\hspace{15pt}
$
$$
\begin{bmatrix}
3&6\\
6&12\\
\end{bmatrix}
\begin{bmatrix}
2\\
-1\\
\end{bmatrix}
$$
$
\hspace{15pt}
and
\hspace{15pt}
$
$$
\begin{bmatrix}
1&2&4\\
2&0&1\\
\end{bmatrix}
\begin{bmatrix}
3\\
1\\
1\\
\end{bmatrix}
$$
$
\\
\vspace{3mm}
%~~~~~~~~~~~~~~~~~~~~~ ANSWER TO 2.1.12 ~~~~~~~~~~~~~~~~~~~~~~~~
\item[2.1.17]Find the matrix $P$ that multiplies $(x, y, z)$ to give $(y, z, x)$. Find the matrix $Q$ that multiplies $(y, z, x)$ to bring back $(x, y, z)$. \\
\vspace{3mm}
%~~~~~~~~~~~~~~~~~~~~~ ANSWER TO 2.1.17 ~~~~~~~~~~~~~~~~~~~~~~~~
\item[2.1.19]What 3 by 3 matrix $E$ multiplies $(x, y, z)$ to give $(x, y, z + x)$? What matrix $E^{-1}$ multiplies $(x, y, z)$ to give $(x, y, z - x)$? If you multiply $(3,4,5)$ by $E$ and then multiply by $E^{-1}$, the two results are $(\underline{\hspace{8mm}})$ and $(\underline{\hspace{8mm}})$. \\
\vspace{3mm}
%~~~~~~~~~~~~~~~~~~~~~ ANSWER TO 2.1.19 ~~~~~~~~~~~~~~~~~~~~~~~~
\item[2.1.33]\textbf{Multiplying by $A$ is a "linear transformation"}. Those important words mean: If $w$ is a combination of $u$ and $v$, then $Aw$ is the same combination of $Au$ and $Av$. It is this "\textbf{\textit{linearity}}" $Aw = cAu + dAv$ that gives us the name \textit{linear algebra}. Problem: If 
$
$$
u=
\begin{bmatrix}
1\\
0\\
\end{bmatrix}
$$
$
 and 
$
$$
v=
\begin{bmatrix}
0\\
1\\
\end{bmatrix}
$$
$
 then $Au$ and $Av$ are the columns of $A$. Combine $w = cu + dv$. \textbf{If 
$
$$
\pmb{w=
\begin{bmatrix}
5\\
7\\
\end{bmatrix}}
$$
$
 how is $\pmb{Aw}$ connected to $\pmb{Au}$ and $\pmb{Av}$?} \\
\end{enumerate}
\vspace{3mm}
%~~~~~~~~~~~~~~~~~~~~~ ANSWER TO 2.1.19 ~~~~~~~~~~~~~~~~~~~~~~~~
\vspace{5mm}
\textbf{Chapter 2.2}
\begin{enumerate}
\item[2.2.5]Choose a right side which gives no solution and another right side which gives infinitely many solutions. What are two of those solutions? \\
\vspace{3mm}
\hspace{100pt}
\textbf{Singular system}
\hspace{45pt}
$3x + 2y = 10$\\
\hspace{232pt}
$6x + 4y =$\\
\vspace{3mm}
%~~~~~~~~~~~~~~~~~~~~~ ANSWER TO 2.2.5 ~~~~~~~~~~~~~~~~~~~~~~~~
\item[2.2.6]Choose $a$ coefficient $b$ that makes this system singular. Then choose a right side $g$ that makes it solvable. Find two solutions in that singular case. \\
$$2x + by = 16$$ $$4x + 8y = g$$.
%~~~~~~~~~~~~~~~~~~~~~ ANSWER TO 2.2.6 ~~~~~~~~~~~~~~~~~~~~~~~~
\item[2.2.11](Recommended) A system of linear equations can't have exactly two solutions. \textit{Why}? \\
\vspace{3mm}
\hspace{10pt}(a) If $(x, y, z)$ and $(X, Y, Z)$ are two solutions, what is another solution?\\
\vspace{3mm}
%~~~~~~~~~~~~~~~~~~~~~ ANSWER TO 2.2.11 (a) ~~~~~~~~~~~~~~~~~~~~~~~~
\hspace{10pt}(b) If 25 planes meet at two points, where else do they meet? 
\vspace{3mm}
%~~~~~~~~~~~~~~~~~~~~~ ANSWER TO 2.2.11 (b) ~~~~~~~~~~~~~~~~~~~~~~~~
\item[2.2.14]Which number $d$ forces a row exchange, and what is the triangular system (not singular) for that $d$? Which $d$ makes this system singular (no third pivot)?
$$2x + 5y + z = 0$$ $$4x + dy + z = 2$$ $$y -z = 3$$. 
%~~~~~~~~~~~~~~~~~~~~~ ANSWER TO 2.2.14 ~~~~~~~~~~~~~~~~~~~~~~~~
\item[2.2.18]Construct a 3 by 3 example that has 9 different coefficients on the left side, but rows 2 and 3 become zero in elimination. How many solutions to your system with $\pmb{b} = (1,10,100)$ and how many with $\pmb{b} = (0,0,0)$? 
\vspace{3mm}
%~~~~~~~~~~~~~~~~~~~~~ ANSWER TO 2.2.18 ~~~~~~~~~~~~~~~~~~~~~~~~
\item[2.2.21]Find the pivots and the solution for both systems ($Ax = b$ and $Kx = b$): \\
\hspace{122pt}$2x+ y      =0$\hspace{60pt}$2x- y      =0$\\
\hspace{105pt}$ x+2y+ z   =0$\hspace{35pt}$-x+2y- z   =0$\\
\hspace{107pt}$    y+2z+ t=0$\hspace{46pt}$    y+2z- t=0$\\
\hspace{125pt}$       z+2t=5$\hspace{55pt}$     - z+2t=5$.\\
\vspace{3mm}
%~~~~~~~~~~~~~~~~~~~~~ ANSWER TO 2.2.21 ~~~~~~~~~~~~~~~~~~~~~~~~
\item[2.2.25]For which three numbers $a$ will elimination fail to give three pivots? \\
\vspace{3mm}
\hspace{95pt}
$
$$
A=
\begin{bmatrix}
a&2&3\\
a&a&4\\
a&a&a\\
\end{bmatrix}
$$
$
is singular for three values of $a$.
\vspace{3mm}
%~~~~~~~~~~~~~~~~~~~~~ ANSWER TO 2.2.25 ~~~~~~~~~~~~~~~~~~~~~~~~
\end{enumerate}
\vspace{5mm}
\textbf{Chapter 2.3}
\begin{enumerate}
\item[2.3.3]Which three matrices $E_{21} , E_{31} , E_{32}$ put $A$ into triangular form $U$? \\
\vspace{3mm}
\hspace{100pt}
$
$$
A=
\begin{bmatrix}
1&1&0\\
4&6&1\\
-2&2&0\\
\end{bmatrix}
$$
$
\hspace{10pt}
and
\hspace{10pt}
$E_{32}E_{31}E_{21}A=u$.\\
\vspace{3mm}
Multiply those $E$'s to get one matrix $M$ that does elimination: $MA=U$. 
\vspace{3mm}
%~~~~~~~~~~~~~~~~~~~~~ ANSWER TO 2.3.3 ~~~~~~~~~~~~~~~~~~~~~~~~
\item[2.3.7]Suppose $E$ subtracts 7 times row 1 from row 3.\\
\hspace{25pt}(a) To \textit{invert} that step you should $\underline{\hspace{8mm}}$ 7 times row $\underline{\hspace{8mm}}$ to row $\underline{\hspace{8mm}}$ \\
\hspace{25pt}(b) What "inverse matrix" $E^{-1}$ takes that reverse step (so $E^{-1}E = I$)? \\
\hspace{25pt}(c) If the reverse step is applied first (and then $E$) show that $E E^{-1} = I$. 
\vspace{3mm}
%~~~~~~~~~~~~~~~~~~~~~ ANSWER TO 2.3.7 ~~~~~~~~~~~~~~~~~~~~~~~~
\item[2.3.10](a) What 3 by 3 matrix $E_{13}$ will add row 3 to row 1? \\
(b) What matrix adds row 1 to row 3 and at the same time row 3 to row 1? \\
(c) What matrix adds row 1 to row 3 and then adds row 3 to row 1?
\vspace{3mm}
%~~~~~~~~~~~~~~~~~~~~~ ANSWER TO 2.3.10 ~~~~~~~~~~~~~~~~~~~~~~~~
\item[2.3.17]The parabola $y = a + bx + cx^2$ goes through the points $(x, y) = (1,4)$ and $(2,8)$ and $(3, 14)$. Find and solve a matrix equation for the unknowns $(a, b, c)$. 
\vspace{3mm}
%~~~~~~~~~~~~~~~~~~~~~ ANSWER TO 2.3.17 ~~~~~~~~~~~~~~~~~~~~~~~~
\item[2.3.18]Multiply these matrices in the orders $EF$ and $FE$: \\
\vspace{3mm}
\hspace{105pt}
$
$$
E=
\begin{bmatrix}
1&0&0\\
a&1&0\\
b&0&1\\
\end{bmatrix}
\hspace{35pt}
F=
\begin{bmatrix}
1&0&0\\
0&1&0\\
0&c&1\\
\end{bmatrix}
$$
$
\vspace{3mm}\\
Also compute $E^2 = EE$ and $F^3 = FFF$. You can guess $F^{100}$. 
\vspace{3mm}
%~~~~~~~~~~~~~~~~~~~~~ ANSWER TO 2.3.18 ~~~~~~~~~~~~~~~~~~~~~~~~
\item[2.3.19]Multiply these row exchange matrices in the orders $PQ$ and $QP$ and $P^2$ : \\
\vspace{3mm}
\hspace{105pt}
$
$$
P=
\begin{bmatrix}
0&1&0\\
1&0&0\\
0&0&1\\
\end{bmatrix}
\hspace{10pt}
$$
$
and
$
$$
\hspace{10pt}
Q=
\begin{bmatrix}
0&0&1\\
0&1&0\\
1&0&0\\
\end{bmatrix}
$$
$
\vspace{3mm}\\
Find another non-diagonal matrix whose square is $M^2 = I$.
\vspace{3mm}
%~~~~~~~~~~~~~~~~~~~~~ ANSWER TO 2.3.19 ~~~~~~~~~~~~~~~~~~~~~~~~
\item[2.3.21]If $E$ adds row 1 to row 2 and $F$ adds row 2 to row 1, does $EF$ equal $FE$? \\
\vspace{3mm}
%~~~~~~~~~~~~~~~~~~~~~ ANSWER TO 2.3.21 ~~~~~~~~~~~~~~~~~~~~~~~~
\item[2.3.24]Apply elimination to the 2 by 3 augmented matrix $[A\hspace{8pt}b\hspace{1pt}]$. What is the triangular system $U x = c$? What is the solution $x$? \\
\vspace{3mm}
$
$$
\hspace{145pt}
Ax=
\begin{bmatrix}
2&3\\
4&1\\
\end{bmatrix}
\begin{bmatrix}
x_{1}\\
x_{2}\\
\end{bmatrix}
=
\begin{bmatrix}
1\\
17\\
\end{bmatrix}
$$
$
\vspace{3mm}
%~~~~~~~~~~~~~~~~~~~~~ ANSWER TO 2.3.24 ~~~~~~~~~~~~~~~~~~~~~~~~
\end{enumerate}
\vspace{5mm}
\textbf{Chapter 2.4}
\begin{enumerate}
\item[2.4.5]Compute $A^2$ and $A^3$. Make a prediction for $A^5$ and $A^n$: \\
\vspace{3mm}
\hspace{130pt}
$
$$
A=
\begin{bmatrix}
1&b \\
0&1 \\
\end{bmatrix}
\hspace{10pt}
$$
$
and
$
$$
\hspace{10pt}
A=
\begin{bmatrix}
2&2 \\
0&0 \\
\end{bmatrix}
$$
$\\
\vspace{3mm}
%~~~~~~~~~~~~~~~~~~~~~ ANSWER TO 2.4.5 ~~~~~~~~~~~~~~~~~~~~~~~~
\item[2.4.6]Show that $(A + B)^2$ is different from $A^2 + 2AB + B^2$, when \\
\vspace{3mm}
\hspace{130pt}
$
$$
A=
\begin{bmatrix}
1&2 \\
0&0 \\
\end{bmatrix}
\hspace{10pt}
$$
$
and
$
$$
\hspace{10pt}
A=
\begin{bmatrix}
1&0 \\
3&0 \\
\end{bmatrix}
$$
$.\\
\vspace{3mm} 
Write down the correct rule for $(A + B)(A + B) = A^2 + \underline{\hspace{8mm}} + B^2$.
\vspace{3mm}
%~~~~~~~~~~~~~~~~~~~~~ ANSWER TO 2.4.6 ~~~~~~~~~~~~~~~~~~~~~~~~
\item[2.4.13]Which of the following matrices are guaranteed to equal $(A - B)^2$:\hspace{15pt}$A^2 - B^2, (B - A)^2, A^2 - 2AB + B^2, A(A - B) - B(A - B), A^2 - AB - BA + B^2$? 
\vspace{3mm}
%~~~~~~~~~~~~~~~~~~~~~ ANSWER TO 2.4.13 ~~~~~~~~~~~~~~~~~~~~~~~~
\item[2.4.14]True or false: \\
\hspace{15pt}(a) If $A^2$ is defined then $A$ is necessarily square. \\
\hspace{15pt}(b) If $AB$ and $BA$ are defined then $A$ and $B$ are square. \\
\hspace{15pt}(c) If $AB$ and $BA$ are defined then $AB$ and $BA$ are square. \\
\hspace{15pt}(d) If $A B = B$ then $A = I$. \\
\vspace{3mm}
%~~~~~~~~~~~~~~~~~~~~~ ANSWER TO 2.4.14 ~~~~~~~~~~~~~~~~~~~~~~~~
\item[2.4.22]By trial and error find real nonzero 2 by 2 matrices such that \\
\vspace{3mm}
\hspace{35pt}
$ A^2=-I \hspace{20pt} BC=0 \hspace{20pt} DE = -ED$ (not allowing $DE = 0$). 
\vspace{3mm}
%~~~~~~~~~~~~~~~~~~~~~ ANSWER TO 2.4.22 ~~~~~~~~~~~~~~~~~~~~~~~~
\item[2.4.23]\hspace{10pt}(a) Find a nonzero matrix $A$ for which $A^2=0$. \\
\hspace{10pt}(b) Find a matrix that has $A^2 \neq 0$ but $A^3 = 0$. \\
\vspace{3mm}
%~~~~~~~~~~~~~~~~~~~~~ ANSWER TO 2.4.23 ~~~~~~~~~~~~~~~~~~~~~~~~
\item[2.4.29]Which matrices $E_{21}$ and $E_{31}$ produce zeros in the $(2, 1)$ and $(3, 1)$ positions of $E_{21} A$ and $E_{31}A$? \\
\vspace{3mm}
\hspace{162pt}
$
$$
A=
\begin{bmatrix}
2&1&0 \\
-2&0&1 \\
8&5&2 \\
\end{bmatrix}
$$
$. \\
\vspace{3mm}
Find the single matrix $E = E_{31} E_{21}$ that produces both zeros at once. Multiply $EA$. 
\vspace{3mm}
%~~~~~~~~~~~~~~~~~~~~~ ANSWER TO 2.4.29 ~~~~~~~~~~~~~~~~~~~~~~~~
\end{enumerate}
\vspace{5mm}
\textbf{Chapter 2.5}
\begin{enumerate}
\item[2.5.7] (Important) If $A$ has row $1 +$ row $3$, show that $A$ is  not invertible: \\
(a.) Explain why $Ax = (1, 0, 0)$ cannot have a solution. \\
(b.) Which right sides $(b_1, b_2, b_3)$ might allow a solution to $Ax = b$?\\
(c.) What happens to row 3 in elimination?\\
\item[2.5.13] If the product $M = ABC$ of three wquare matries is invertible, then $B$ is invertible. \\
(So are $A$ and $C$.) Find a formula for $B^{-1}$ that involves $M^{-1}$ and $A$ and $C$.\\
\item[2.5.29] True or false (with a counterexample if false and a reason if true):\\
(a) A 4 by 4 matrix with a row of zeros is not invertible.\\
(b) Every matrix with 1's down the main diagonal is invertible.\\
(c) If $A$ is invertible then $A^{-1}$ and $A^2$ are invertible.\\
\item[2.5.30] For which three numbers $c$ is this matrix not invertible, and why not?\\
\begin{center}
$
$$
A=
\begin{bmatrix}
2&c&c\\
c&c&c\\
8&7&c\\
\end{bmatrix}
$$
$
\end{center}
\item[2.5.31] Prove that $A$ is invertible if $a\neq 0$ and $a \neq b$ (find the pivots or $A^{-1}$):\\
\begin{center}
$
$$
A=
\begin{bmatrix}
a&b&b\\
a&a&b\\
a&a&a\\
\end{bmatrix}
$$
$
\end{center}
\end{enumerate}
\vspace{5mm}
\textbf{Chapter 2.6}
\begin{enumerate}
\item[2.6.7] What three eleimination matrices $E_{21}, E_{31}, E_{32}$ put $A$ into its upper triangular form $E_{21}, E_{31}, E_{32}A = U$? Multiply by $E_{32}^{-1}, E_{31}^{-1}$ and  $E_{21}^{-1}$ to factor $A$ into $L$ times $U$:\\
\begin{center}
$
$$
A=
\begin{bmatrix}
1&0&1\\
2&2&2\\
3&4&5\\
\end{bmatrix}
$$
$
$L = E_{32}^{-1} E_{31}^{-1} E_{21}^{-1}$\\
\end{center}
\item[2.6.8] Suppose $A$ is already lower trianglular with 1's on the diagonal. Then $U = I$!\\
\begin{center}
$
$$
A= L=
\begin{bmatrix}
1&0&0\\
a&1&0\\
b&c&0\\
\end{bmatrix}
$$
$
\end{center}
The elimination matrices $E_{21}, E_{31}, E_{32}$ contain -a then -b then -c.\\
(a) Multiply $E_{32}, E_{31}, E_{21}$ to find the single matric $E$ that produces $EA = I$.\\
(b) Multiply $E_{21}^{-1} E_{31}^{-1} E_{32}^{-1}$ to bring back $L$ (nicer than $E$).\\
\item[2.6.13] Compute $L$ and $U$ for the symmetric matrix $A$:\\
\begin{center}
$
$$
A=
\begin{bmatrix}
a&r&r&r\\
a&b&s&s\\
a&b&c&t\\
a&b&c&d\\
\end{bmatrix}
$$
$
\end{center}
Find four conditions on $a, b, c, d$ to get $A = LU$ with four pivots.\\
\item[2.6.16] Solve $L\pmb{c} = \pmb{b}$ to find $\pmb{c}$. Then solve $U\pmb{x} = \pmb{c}$ to find $\pmb{x}$. \textbf{What was} $A$?\\
\begin{center}
$
$$
L=
\begin{bmatrix}
1&0&0\\
1&1&0\\
1&1&1\\
\end{bmatrix}
$$
$
and
$
$$
U =
\begin{bmatrix}
1&1&1\\
0&1&1\\
0&0&1\\
\end{bmatrix}
$$
$
and 
$
$$
\pmb{b} =
\begin{bmatrix}
4\\
5\\
6\\
\end{bmatrix}
$$
$
\end{center}
\end{enumerate}
\vspace{5mm}
\textbf{Chapter 2.7}
\begin{enumerate}
\item[2.7.3] (a) The matrix $((AB)^{-1})^T$ comes from $(A^{-1})^T$ and $(B^{-1})^T$. \textit{In what order}?\\
(b) If $U$ is upper triangular then $(U^{-1})^T$ is \underline{\hspace{6mm}} triangular.\\
\item[2.7.6] The transpose of a block matrix $M = \left[\begin{smallmatrix} \pmb{A}\ \pmb{B} \\ \pmb{C}\ \pmb{D} \end{smallmatrix} \right]$ is $M^T =$ \underline{\hspace{6mm}}. Test an example.\\
Under what conditions on $A, B, C, D$ is the block matrix symmetric?\\
\item[2.7.16] If $A = A^{T}$ and $B = B^{T}$, which of these matrices are certainly symmetric?\\
(a) $A^2 - B^2$ \hspace{10mm} (b) $(A + B)(A - B)$ \hspace{10mm} (c) $ABA$ \hspace{10mm} (d) $ABAB$.\\
\item[2.7.31] Producing $x_1$ trucks and $x_2$ planes needs $ x_1 + 50x_2$ tons of steel, $40x_1 + 1000x_2$ pounds of rubber, and $2x_1 + 50x_2$ months of labor. If the unit costs $y_1, y_2, y_3$ are \$700 per ton, \$3 per pound, and \$3000 per month, what are the values of one truck and one plane? Those are the components of $A^T\pmb{y}$.\\
\item[2.7.40] Suppose $Q^T equals \hspace{1mm} Q^{-1}$ (transpose equals inverse, so $Q^TQ = I$).\\
(a) Show that the columns $q_1, \cdots, q_n$ are unit vectors: $\lVert \pmb{q}_i \rVert^2 = 1$.\\
(b) Show that every two columns of $Q$ are perpendicular: $\pmb{q}_1^2\pmb{q}_2 = 0$.\\
(c) Find a 2 by 2 example with first entry $q_{11} = cos\theta$.\\
\end{enumerate}
\end{document}