\documentclass[10pt,twoside,reqno]{article}
\usepackage[marginparsep=1em]{geometry}
\geometry{lmargin=1.0in,rmargin=1.0in, bmargin=0.75in,  tmargin=0.75in}
\usepackage[usenames,dvipsnames,svgnames,table]{xcolor}
\usepackage{graphicx}
\usepackage{amssymb}
\usepackage{epstopdf}
\usepackage{tikz}
\usepackage{enumerate}
\usepackage{amsthm}
 \usepackage{pgfplots}
 \usepackage{tikz-3dplot}
 \usetikzlibrary{shapes.geometric}
 \usepackage{float}
\usepackage{amsmath}
\usepackage{fancyhdr}
\usepackage{lmodern}
\usepackage{chngcntr}
\usepackage{multicol, comment}

\pagestyle{fancy}
\fancyhf{}
\renewcommand{\sectionmark}[1]{\markright{\thesection.\ #1}}
\lhead{\fancyplain{}{}} 
\fancyhead[RE,RO]{MATH 2270}
\fancyfoot[RE,LO]{Dr. Heavilin}
\fancyfoot[LE,RO]{\thepage}


\begin{document}
\begin{flushright}
\begin{minipage}{.25\textwidth}
Dustin Ginos: \\
A01233669\\
Chandler Kinch: \\
A01662772\\
Jeff Wasden: \\
A01657029\\

\today
\end{minipage}
\end{flushright}

\center{\textbf{\underline{Homework 4}}}\\
\vspace{5mm}
\textbf{Chapter 4.1}
\begin{enumerate}
\item[4.1.3] Construct a matrix with the required property or say why that is impossible: \\
{\addtolength{\leftskip}{5mm}
(a)Column space contains 
$
\begin{bmatrix}
1\\
2\\
-3\\
\end{bmatrix}
$
 and 
$
\begin{bmatrix}
2\\
-3\\
5\\
\end{bmatrix}
$
, nullspace contains 
$
\begin{bmatrix}
1\\
1\\
1\\
\end{bmatrix}
$ \\
(b) Row space contains 
$
\begin{bmatrix}
1\\
2\\
-3\\
\end{bmatrix}
$
 and 
$
\begin{bmatrix}
2\\
-3\\
5\\
\end{bmatrix}
$
, nullspace contains 
$
\begin{bmatrix}
1\\
1\\
1\\
\end{bmatrix}
$ \\
(c) $Ax=
\begin{bmatrix}
1\\
1\\
1\\
\end{bmatrix}
$ has a solution and $A^T
\begin{bmatrix}
1\\
0\\
0\\
\end{bmatrix}
=
\begin{bmatrix}
0\\
0\\
0\\
\end{bmatrix}
$ \\
(d) Every row is orthogonal to every column ($A$ is not the zero matrix) \\
(e) Columns add up to a column of zeros, rows add to a row of 1's. \\
}
%~~~~~~~~~~~~~~~~~~~~~ ANSWER TO 4.1.3 ~~~~~~~~~~~~~~~~~~~~~~~~
\vspace{3mm}

\vspace{3mm}
%~~~~~~~~~~~~~~~~~~~~~~~~~~~~~~~~~~~~~~~~~~~~~~~~~~~~~~~~~~~~~~~
\item[4.1.11] (Recommended) Draw Figure 4.2 to show each subspace correctly for \\
\begin{center}
$
A=
\begin{bmatrix}
1&2\\
3&6\\
\end{bmatrix}
$
 and 
$
B=
\begin{bmatrix}
1&0\\
3&0\\
\end{bmatrix}
$.
\end{center}
%~~~~~~~~~~~~~~~~~~~~~ ANSWER TO 4.1.11 ~~~~~~~~~~~~~~~~~~~~~~~~
\vspace{3mm}

\vspace{3mm}
%~~~~~~~~~~~~~~~~~~~~~~~~~~~~~~~~~~~~~~~~~~~~~~~~~~~~~~~~~~~~~~~
\item[4.1.16] Prove that every $y$ in $N(A^T)$ is perpendicular to every $Ax$ in the column space, using the matrix shorthand of equation (2). Start from $A^Ty = 0$. 
%~~~~~~~~~~~~~~~~~~~~~ ANSWER TO 4.1.16 ~~~~~~~~~~~~~~~~~~~~~~~~
\vspace{3mm}

\vspace{3mm}
%~~~~~~~~~~~~~~~~~~~~~~~~~~~~~~~~~~~~~~~~~~~~~~~~~~~~~~~~~~~~~~~
\item[4.1.17] If $S$ is the subspace of $R^3$ containing only the zero vector, what is $S^{\perp}$? If $S$ is spanned by (1, 1, 1), what is $S^{\perp}$? If $S$ is spanned by (1, 1, 1) and (1, 1, -1), what is a basis for $S^{\perp}$? \\ 
%~~~~~~~~~~~~~~~~~~~~~ ANSWER TO 4.1.17 ~~~~~~~~~~~~~~~~~~~~~~~~
\vspace{3mm}

\vspace{3mm}
%~~~~~~~~~~~~~~~~~~~~~~~~~~~~~~~~~~~~~~~~~~~~~~~~~~~~~~~~~~~~~~~
\item[4.1.18] Suppose $S$ only contains two vectors (1,5,1) and (2,2,2) (not a subspace), Then $S^{\perp}$ is the nullspace of the matrix $A =\underline{\hspace{8mm}}$. $S^{\perp}$ is a subspace even if $S$ is not. \\
%~~~~~~~~~~~~~~~~~~~~~ ANSWER TO 4.1.18 ~~~~~~~~~~~~~~~~~~~~~~~~
\vspace{3mm}

\vspace{3mm}
%~~~~~~~~~~~~~~~~~~~~~~~~~~~~~~~~~~~~~~~~~~~~~~~~~~~~~~~~~~~~~~~
\item[4.1.21] Suppose $S$ is spanned by the vectors (1,2,2,3) and (1,3,3,2). Find two vectors that span $S^{\perp}$, This is the same as solving $Ax = 0$ for which $A$? \\
%~~~~~~~~~~~~~~~~~~~~~ ANSWER TO 4.1.21 ~~~~~~~~~~~~~~~~~~~~~~~~
\vspace{3mm}

\vspace{3mm}
%~~~~~~~~~~~~~~~~~~~~~~~~~~~~~~~~~~~~~~~~~~~~~~~~~~~~~~~~~~~~~~~
\item[4.1.22] If $P$ is the plane of vectors in $R^4$ satisfying $x_1 + x_2 + x_3 + x_4 = 0$, write a basis for $P^{\perp}$. Construct a matrix that has $P$ as its nullspace.  \\
%~~~~~~~~~~~~~~~~~~~~~ ANSWER TO 4.1.22 ~~~~~~~~~~~~~~~~~~~~~~~~
\vspace{3mm}

\vspace{3mm}
%~~~~~~~~~~~~~~~~~~~~~~~~~~~~~~~~~~~~~~~~~~~~~~~~~~~~~~~~~~~~~~~
\item[4.1.24] Suppose an $n$ by $n$ matrix is invertible: $AA^{-1} = I$. Then the first column of $A^{-1}$ is orthogonal to the space spanned by which rows of $A$? \\
%~~~~~~~~~~~~~~~~~~~~~ ANSWER TO 4.1.24 ~~~~~~~~~~~~~~~~~~~~~~~~
\vspace{3mm}

\vspace{3mm}
%~~~~~~~~~~~~~~~~~~~~~~~~~~~~~~~~~~~~~~~~~~~~~~~~~~~~~~~~~~~~~~~
\item[4.1.25] Find $A^T A$ if the columns of $A$ are unit vectors, all mutually perpendicular. \\
%~~~~~~~~~~~~~~~~~~~~~ ANSWER TO 4.1.25 ~~~~~~~~~~~~~~~~~~~~~~~~
\vspace{3mm}

\vspace{3mm}
%~~~~~~~~~~~~~~~~~~~~~~~~~~~~~~~~~~~~~~~~~~~~~~~~~~~~~~~~~~~~~~~
\item[4.1.28] Why is each of these statements false? \\
{\addtolength{\leftskip}{5mm}
(a) (1, 1, 1) is perpendicular to (1,1, \-2) so the planes $x + y + z = 0$ and $x + y - 2z = 0$ are orthogonal subspaces. \\
(b) The subspace spanned by (1,1,0,0,0) and (0,0,0,1,1) is the orthogonal complement of the subspace spanned by (1,-1,0,0,0) and (2,-2,3,4,-4). \\
(c) Two subspaces that meet only in the zero vector are orthogonal. \\
}
%~~~~~~~~~~~~~~~~~~~~~ ANSWER TO 4.1.28 ~~~~~~~~~~~~~~~~~~~~~~~~
\vspace{3mm}

\vspace{3mm}
%~~~~~~~~~~~~~~~~~~~~~~~~~~~~~~~~~~~~~~~~~~~~~~~~~~~~~~~~~~~~~~~
\item[4.1.33] Suppose I give you eight vectors $r_1, r_2, n_l, n_2, c_1, c_2, l_1,l_2$ in $R^4$. \\
{\addtolength{\leftskip}{5mm}
(a) What are the conditions for those pairs to be bases for the four fundamental subspaces of a 4 by 4 matrix? \\
(b) What is one possible matrix $A$? \\
}
%~~~~~~~~~~~~~~~~~~~~~ ANSWER TO 4.1.33 ~~~~~~~~~~~~~~~~~~~~~~~~
\vspace{3mm}

\vspace{3mm}
%~~~~~~~~~~~~~~~~~~~~~~~~~~~~~~~~~~~~~~~~~~~~~~~~~~~~~~~~~~~~~~~
\end{enumerate}
\vspace{5mm}
\textbf{Chapter 4.2}
\begin{enumerate}
\item[4.2.2] Draw the projection of $b$ onto $a$ and also compute it from $p = \hat{x}a$: \\
\begin{center}
(a) 
$
b=
\begin{bmatrix}
cos\theta\\
sin\theta\\
\end{bmatrix}
$
\hspace{3mm}and\hspace{3mm}
$
\begin{bmatrix}
1\\
0\\
\end{bmatrix}
$
\hspace{8mm}(b) 
$
b=
\begin{bmatrix}
1\\
1\\
\end{bmatrix}
$
\hspace{3mm}and\hspace{3mm}
$
a=
\begin{bmatrix}
1\\
-1\\
\end{bmatrix}
$. \\
\end{center}
%~~~~~~~~~~~~~~~~~~~~~ ANSWER TO 4.2.2 ~~~~~~~~~~~~~~~~~~~~~~~~
\vspace{3mm}

\vspace{3mm}
%~~~~~~~~~~~~~~~~~~~~~~~~~~~~~~~~~~~~~~~~~~~~~~~~~~~~~~~~~~~~~~~
\item[4.2.13] (Quick and Recommended) Suppose $A$ is the 4 by 4 identity matrix with its last column removed. $A$ is 4 by 3. Project $b = (1,2,3,4)$ onto the column space of $A$. What shape is the projection matrix $P$ and what is $P$? \\
%~~~~~~~~~~~~~~~~~~~~~ ANSWER TO 4.2.13 ~~~~~~~~~~~~~~~~~~~~~~~~
\vspace{3mm}

\vspace{3mm}
%~~~~~~~~~~~~~~~~~~~~~~~~~~~~~~~~~~~~~~~~~~~~~~~~~~~~~~~~~~~~~~~
\item[4.2.16] What linear combination of (1,2,-1) and (1,0,1) is closest to $b = (2, 1, 1)$? \\
%~~~~~~~~~~~~~~~~~~~~~ ANSWER TO 4.2.16 ~~~~~~~~~~~~~~~~~~~~~~~~
\vspace{3mm}

\vspace{3mm}
%~~~~~~~~~~~~~~~~~~~~~~~~~~~~~~~~~~~~~~~~~~~~~~~~~~~~~~~~~~~~~~~
\item[4.2.17] (Important) If $P^2 = P$ show that $(I - P)^2 = I - P$. When $P$ projects onto the column space of $A$, $I- P$ projects onto the \underline{\hspace{8mm}}. \\
%~~~~~~~~~~~~~~~~~~~~~ ANSWER TO 4.2.17 ~~~~~~~~~~~~~~~~~~~~~~~~
\vspace{3mm}

\vspace{3mm}
%~~~~~~~~~~~~~~~~~~~~~~~~~~~~~~~~~~~~~~~~~~~~~~~~~~~~~~~~~~~~~~~
\item[4.2.18] (a) If $P$ is the 2 by 2 projection matrix onto the line through (1,1), then $I - P$ is the projection matrix onto \underline{\hspace{8mm}}. \\
 (b) If $P$ is the 3 by 3 projection matrix onto the line through (1,1,1), then $I - P$ is the projection matrix onto \underline{\hspace{8mm}}. \\
%~~~~~~~~~~~~~~~~~~~~~ ANSWER TO 4.2.18 ~~~~~~~~~~~~~~~~~~~~~~~~
\vspace{3mm}

\vspace{3mm}
%~~~~~~~~~~~~~~~~~~~~~~~~~~~~~~~~~~~~~~~~~~~~~~~~~~~~~~~~~~~~~~~
\item[4.2.19] To find the projection matrix onto the plane $x - y - 2z = 0$, choose two vectors in that plane and make them the columns of $A$. The plane should be the column space. Then compute $P = A(A^T A)^{-l} A^T$. \\
%~~~~~~~~~~~~~~~~~~~~~ ANSWER TO 4.2.19 ~~~~~~~~~~~~~~~~~~~~~~~~
\vspace{3mm}

\vspace{3mm}
%~~~~~~~~~~~~~~~~~~~~~~~~~~~~~~~~~~~~~~~~~~~~~~~~~~~~~~~~~~~~~~~
\item[4.2.26] If an $m$ by $m$ matrix has $A^2 = A$ and its rank is $m$, prove that $A = I$. \\
%~~~~~~~~~~~~~~~~~~~~~ ANSWER TO 4.2.26 ~~~~~~~~~~~~~~~~~~~~~~~~
\vspace{3mm}

\vspace{3mm}
%~~~~~~~~~~~~~~~~~~~~~~~~~~~~~~~~~~~~~~~~~~~~~~~~~~~~~~~~~~~~~~~
\item[4.2.27] The important fact that ends the section is this: If $A^TAx = 0$ then $Ax = 0$. New Proof: The vector $Ax$ is in the nullspace of \underline{\hspace{8mm}}. $Ax$ is always in the column space of \underline{\hspace{8mm}}. To be in both of those perpendicular spaces, $Ax$ must be zero. \\
%~~~~~~~~~~~~~~~~~~~~~ ANSWER TO 4.2.27 ~~~~~~~~~~~~~~~~~~~~~~~~
\vspace{3mm}

\vspace{3mm}
%~~~~~~~~~~~~~~~~~~~~~~~~~~~~~~~~~~~~~~~~~~~~~~~~~~~~~~~~~~~~~~~
\item[4.2.29] If $B$ has rank $m$ (full row rank, independent rows) show that $BB^T$ is invertible. 
%~~~~~~~~~~~~~~~~~~~~~ ANSWER TO 4.2.29 ~~~~~~~~~~~~~~~~~~~~~~~~
\vspace{3mm}

\vspace{3mm}
%~~~~~~~~~~~~~~~~~~~~~~~~~~~~~~~~~~~~~~~~~~~~~~~~~~~~~~~~~~~~~~~
\item[4.2.30] (a) Find the projection matrix $Pc$ onto the column space of $A$ (after looking closely at the matrix!) \\
\begin{center}
$
A=
\begin{bmatrix}
3&6&6\\
4&8&8\\
\end{bmatrix}
$
\end{center}
 (b) Find the 3 by 3 projection matrix $P_R$ onto the row space of $A$. Multiply $B = P_CAP_R$. Your answer $B$ should be a little surprising-can you explain it?  \\
%~~~~~~~~~~~~~~~~~~~~~ ANSWER TO 4.2.30 ~~~~~~~~~~~~~~~~~~~~~~~~
\vspace{3mm}

\vspace{3mm}
%~~~~~~~~~~~~~~~~~~~~~~~~~~~~~~~~~~~~~~~~~~~~~~~~~~~~~~~~~~~~~~~

\end{enumerate}
\vspace{5mm}
\textbf{Chapter 4.3}
\begin{enumerate}
\item[4.3.6] Project $\pmb{b} = (0, 8, 8, 20)$ onto the line through $\pmb{a} = (1, 1, 1, 1)$. Find $\hat{x} = \pmb{a}^T\pmb{b}/\pmb{a}^T\pmb{a}$ and the projection $\pmb{p} = \hat{x}\pmb{a}$. Check that $\pmb{e} = \pmb{b} - \pmb{p}$ is perpendicular to $\pmb{a}$, and find the shortest distance $\lVert\pmb{e}\rVert$ from $\pmb{b}$ to the line through $\pmb{a}$\\
%~~~~~~~~~~~~~~~~~~~~~ ANSWER TO 4.3.6 ~~~~~~~~~~~~~~~~~~~~~~~~
\vspace{3mm}
$
$$
a =
\begin{bmatrix}
1\\
1\\
1\\
1\\
\end{bmatrix}
\hspace{3mm}
\begin{bmatrix}
0\\
8\\
8\\
20\\
\end{bmatrix}
$$
$\\
$\hat{x} = \frac{a^Tb}{a^Ta} =$
$
$$
\frac{
\begin{bmatrix}
1&1&1&1
\end{bmatrix}
\begin{bmatrix}
0\\
8\\
8\\
20\\
\end{bmatrix}
}{
\begin{bmatrix}
1&1&1&1
\end{bmatrix}
\begin{bmatrix}
1\\
1\\
1\\
1\\
\end{bmatrix}
}
= \frac{36}{4} = 9 \hspace{3mm} p = \hat{x}a =
\begin{bmatrix}
9\\
9\\
9\\
9\\
\end{bmatrix}
$$
$\\
$
$$
b - p = e
\begin{bmatrix}
0\\
8\\
8\\
20\\
\end{bmatrix}
-
\begin{bmatrix}
-9\\
-1\\
-1\\
11\\
\end{bmatrix}
\pmb{\cdot}
\begin{bmatrix}
1\\
1\\
1\\
1\\
\end{bmatrix}
= 0 \hspace{1mm}\checkmark
$$
$\\
$\lVert e \rVert = \sqrt{204}$\\

\vspace{3mm}
%~~~~~~~~~~~~~~~~~~~~~~~~~~~~~~~~~~~~~~~~~~~~~~~~~~~~~~~~~~~~~~~
\item[4.3.9] For the closest parabola $b = C + Dt + Et^2$ to the same four points, write down the unsolvable equations $Ax = b$ in three unknowns $x = (C, D, E)$. Set up the three normal equations $A^TA\hat{x} = A^Tb$ (solution not required). In Figure 4.9a you are now fitting a parabola to 4 points - what is happening in Figure 4.9b?\\
%~~~~~~~~~~~~~~~~~~~~~ ANSWER TO 4.3.9 ~~~~~~~~~~~~~~~~~~~~~~~~
\vspace{3mm}
$
$$
\begin{bmatrix}
1&0&0\\
1&1&1\\
1&3&9\\
1&4&16\\
\end{bmatrix}
\begin{bmatrix}
C\\
D\\
E\\
\end{bmatrix}
=
\begin{bmatrix}
0\\
8\\
8\\
20\\
\end{bmatrix}
\rightarrow
\begin{bmatrix}
1&1&1&1\\
0&1&3&4\\
0&1&9&16\\
\end{bmatrix}
\begin{bmatrix}
1&0&0\\
1&1&1\\
1&3&9\\
1&4&16\\
\end{bmatrix}
\begin{bmatrix}
C\\
D\\
E\\
\end{bmatrix}
=
\begin{bmatrix}
1&1&1&1\\
0&1&3&4\\
0&1&9&16\\
\end{bmatrix}
\begin{bmatrix}
0\\
8\\
8\\
20\\
\end{bmatrix}
\rightarrow
$$
$\\
\vspace{3mm}
$
$$
\begin{bmatrix}
4&8&26\\
8&26&92\\
26&92&338\\
\end{bmatrix}
\begin{bmatrix}
C&D&E
\end{bmatrix}
=
\begin{bmatrix}
36\\
112\\
400\\
\end{bmatrix}
$$
$\\
\vspace{6mm}
%~~~~~~~~~~~~~~~~~~~~~~~~~~~~~~~~~~~~~~~~~~~~~~~~~~~~~~~~~~~~~~~
\item[4.3.12] (Recommended) This problem projects $\pmb{b} = (b_1, ...,b_m)$ onto the line through $a = (1,...,1)$. We solve $m$ equations $ax = b$ in 1 unknown (by least squares).\\
(a) Solve $a^Ta\hat{x} = a^Tb$ to show that $\hat{x}$ is the \textit{mean} (the average) of the $b$'s.\\
%~~~~~~~~~~~~~~~~~~~~~ ANSWER TO 4.3.12 ~~~~~~~~~~~~~~~~~~~~~~~~
\vspace{3mm}
$
$$
\begin{bmatrix}
11\cdots11\\
\end{bmatrix}
\begin{bmatrix}
1\\
1\\
\cdot\\
\cdot\\
\cdot\\
1\\
1\\
\end{bmatrix}
\hat{x}
=
\begin{bmatrix}
11\cdots11\\
\end{bmatrix}
\begin{bmatrix}
b_1\\
\cdot\\
\cdot\\
\cdot\\
b_m
\end{bmatrix}
$$
$\\
$=$ (number of components in a) $\hat{x} = \sum_{i = 1}^{m}b_i \rightarrow\frac{ \hat{x} = \sum_{i = 1}^{m}b_i}{number \hspace{1mm}of\hspace{1mm} components} =$ average\\
\vspace{3mm}
%~~~~~~~~~~~~~~~~~~~~~~~~~~~~~~~~~~~~~~~~~~~~~~~~~~~~~~~~~~~~~~~
(b) Find $e = b - a\hat{x}$ and the \textit{variance} $\lVert e \rVert^2$ and the \textit{standard deviation} $\lVert e \rVert$.\\
%~~~~~~~~~~~~~~~~~~~~~ ANSWER TO 4.3.12 ~~~~~~~~~~~~~~~~~~~~~~~~
\vspace{3mm}
$
$$
e =
\begin{bmatrix}
b_1\\
\cdot\\
\cdot\\
\cdot\\
b_m
\end{bmatrix}
-
\begin{bmatrix}
\hat{x}\\
\cdot\\
\cdot\\
\cdot\\
\hat{x}\\
\end{bmatrix}
=
\begin{bmatrix}
b_1 - \hat{x}\\
\cdot\\
\cdot\\
\cdot\\
b_m - \hat{x}\\
\end{bmatrix}
$$
$\\
\vspace{3mm}
$
$$
\lVert e \rVert^2 = \sum_{i = 1}^m(b_i - \hat{x})^2 =
$$
$
variance\\
\vspace{3mm}
%~~~~~~~~~~~~~~~~~~~~~~~~~~~~~~~~~~~~~~~~~~~~~~~~~~~~~~~~~~~~~~~
(c) The horizontal line $\hat{b} = 3$ is closest to $b = (1, 2, 6)$. Check that $p = (3, 3, 3)$ is perpendicular to $e$ \\
\hspace{6mm}and find the 3 by 3 projection matrix $P$.\\
%~~~~~~~~~~~~~~~~~~~~~ ANSWER TO 4.3.12 ~~~~~~~~~~~~~~~~~~~~~~~~
\vspace{3mm}
$
$$
e = b - p =
\begin{bmatrix}
1\\
2\\
6\\
\end{bmatrix}
-
\begin{bmatrix}
3\\
3\\
3\\
\end{bmatrix}
=
\begin{bmatrix}
-2\\
-1\\
3\\
\end{bmatrix}
\pmb{\cdot}
\begin{bmatrix}
3\\
3\\
3\\
\end{bmatrix}
= -6-3+9 = 0 \hspace{2mm}\checkmark
$$
$\\
\vspace{3mm}
$
$$
P = 
\frac{1}{3}
\begin{bmatrix}
1&1&1\\
1&1&1\\
1&1&1\\
\end{bmatrix}
$$
$
\vspace{3mm}
%~~~~~~~~~~~~~~~~~~~~~~~~~~~~~~~~~~~~~~~~~~~~~~~~~~~~~~~~~~~~~~~
\item[4.3.22] Find the best line $C + Dt$ to fit $b = 4, 2, -1, 0, 0$ at times $t = -2, -1, 0, 1, 2$.\\
%~~~~~~~~~~~~~~~~~~~~~ ANSWER TO 4.3.22 ~~~~~~~~~~~~~~~~~~~~~~~~
\vspace{3mm}
$
$$
\begin{bmatrix}
1&-2\\
1&-1\\
1&0\\
1&1\\
1&2\\
\end{bmatrix}
\begin{bmatrix}
C\\
D\\
\end{bmatrix}
=
\begin{bmatrix}
4\\
2\\
-1\\
0\\
0\\
\end{bmatrix}
\rightarrow
\begin{bmatrix}
1&1&1&1&1\\
-2&-1&0&1&2\\
\end{bmatrix}
\begin{bmatrix}
1&-2\\
1&-1\\
1&0\\
1&1\\
1&2\\
\end{bmatrix}
\begin{bmatrix}
C\\
D\\
\end{bmatrix}
=
\begin{bmatrix}
1&1&1&1&1\\
-2&-1&0&1&2\\
\end{bmatrix}
\begin{bmatrix}
4\\
2\\
-1\\
0\\
0\\
\end{bmatrix}
\rightarrow
$$
$\\
\vspace{3mm}
$
$$
\begin{bmatrix}
5&0\\
0&10\\
\end{bmatrix}
\begin{bmatrix}
C\\
D\\
\end{bmatrix}
=
\begin{bmatrix}
5\\
-10\\
\end{bmatrix}
\hspace{3mm} C = 1 \hspace{3mm} D = -1
$$
$\\
\vspace{3mm}
%~~~~~~~~~~~~~~~~~~~~~~~~~~~~~~~~~~~~~~~~~~~~~~~~~~~~~~~~~~~~~~~
\item[4.3.26] Find the \textit{plane} that gives the best fit to the 4 values $b = (0, 1, 3, 4)$ at the corners $(1, 0)$ and $(0, 1)$ and $(-1, 0) and (0, -1)$ of a square. The equations $C + Dx + Ey = b$ at those 4 points are $Ax = b$ with 3 unknowns $x = (C, D, E)$. What is $A$? At the center $(0, 0)$ of the square, show that $C + Dx + Ey =$ average of the $b$'s.\\
%~~~~~~~~~~~~~~~~~~~~~ ANSWER TO 4.3.26 ~~~~~~~~~~~~~~~~~~~~~~~~
\vspace{3mm}
$
$$
\begin{bmatrix}
1&1&0\\
1&0&1\\
1&-1&0\\
1&0&-1\\
\end{bmatrix}
\begin{bmatrix}
C\\
D\\
E\\
\end{bmatrix}
=
\begin{bmatrix}
0\\
1\\
3\\
4\\
\end{bmatrix}
\rightarrow
\begin{bmatrix}
1&1&1&1\\
1&0&-1&0\\
0&1&0&-1\\
\end{bmatrix}
\begin{bmatrix}
1&1&0\\
1&0&1\\
1&-1&0\\
1&0&-1\\
\end{bmatrix}
\begin{bmatrix}
C\\
D\\
E\\
\end{bmatrix}
=
\begin{bmatrix}
1&1&1&1\\
1&0&-1&-\\
0&1&0&-1\\
\end{bmatrix}
\begin{bmatrix}
0\\
1\\
3\\
4\\
\end{bmatrix}
$$
$\\
$
$$
\begin{bmatrix}
4&0&0\\
0&2&0\\
0&0&2\\
\end{bmatrix}
\begin{bmatrix}
C\\
D\\
E\\
\end{bmatrix}
=
\begin{bmatrix}
8\\
-3\\
-3\\
\end{bmatrix}
\hspace{3mm} C = 2 \hspace{3mm} D = \frac{-3}{2} = E
$$
$\\
$p = 2 - \frac{3}{2}x - \frac{3}{2}y$\\
at $x = y = 0 \hspace{3mm} p = 2$, which is the average of the square\\
\vspace{3mm}
%~~~~~~~~~~~~~~~~~~~~~~~~~~~~~~~~~~~~~~~~~~~~~~~~~~~~~~~~~~~~~~~

\end{enumerate}
\vspace{5mm}
\textbf{Chapter 4.4}
\begin{enumerate}
\item[4.4.1] Are these pairs of vectors orthonormal or only orthogonal or only independent?\\
\vspace{2mm}
(a) $\left[\begin{smallmatrix} 1\\ 0 \end{smallmatrix} \right]$ and $\left[\begin{smallmatrix} -1\\ 1 \end{smallmatrix} \right]$ \hspace{3mm} (b) $\left[\begin{smallmatrix} .6\\ .8 \end{smallmatrix} \right]$ and $\left[\begin{smallmatrix} .4\\ -.3 \end{smallmatrix} \right]$ \hspace{3mm} (c) $\left[\begin{smallmatrix} cos\theta\\ sin\theta \end{smallmatrix} \right]$ and $\left[\begin{smallmatrix} -sin\theta\\ cos\theta \end{smallmatrix} \right]$.\\
\vspace{2mm}
Change the second vector when necessary to produce orthonormal vectors.\\
%~~~~~~~~~~~~~~~~~~~~~ ANSWER TO 4.4.1 ~~~~~~~~~~~~~~~~~~~~~~~~
\vspace{3mm}
(a) independent, the second vector would be $\left[\begin{smallmatrix} 0\\ 1 \end{smallmatrix} \right]$ to be orthonormal\\
\vspace{3mm}
(b) orthogonal, the second vector would be $\left[\begin{smallmatrix} .8\\ -.6\end{smallmatrix} \right]$ to be orthonoraml and independent\\
\vspace{3mm}
(c) orthonormal and independent\\
\vspace{3mm}

%~~~~~~~~~~~~~~~~~~~~~~~~~~~~~~~~~~~~~~~~~~~~~~~~~~~~~~~~~~~~~~~
\item[4.4.4] Give an example of each of the following:\\
(a) A matrix $Q$ that has orthonormal columns but $QQ^T \neq I$.\\
\vspace{3mm}
$
$$
Q =
\begin{bmatrix}
2&1\\
2&-1\\
\end{bmatrix}
QQT=
\begin{bmatrix}
2&1\\
2&-1\\
\end{bmatrix}
\begin{bmatrix}
2&2\\
1&-1\\
\end{bmatrix}
=
\begin{bmatrix}
5&3\\
3&5\\
\end{bmatrix}
\neq I
$$
$\\
\vspace{3mm}
(b) Two orthogonal vectors that are not linearly independent.\\
\vspace{3mm}
$
$$
\begin{bmatrix}
0\\
1\\
\end{bmatrix}
\begin{bmatrix}
0\\
0\\
\end{bmatrix}
$$
$\\
\vspace{3mm}
(c) An orthonormal basis for $\pmb{R}^3$, including the vector $q_1 = (1, 1, 1)/\sqrt{3}$.\\
%~~~~~~~~~~~~~~~~~~~~~ ANSWER TO 4.4.4 ~~~~~~~~~~~~~~~~~~~~~~~~
\vspace{3mm}
$
$$
\begin{bmatrix}
\frac{1}{\sqrt{3}}\\
\frac{1}{\sqrt{3}}\\
\frac{1}{\sqrt{3}}\\
\end{bmatrix}
,
\begin{bmatrix}
1\\
-1\\
0\\
\end{bmatrix}
,
\begin{bmatrix}
1\\
0\\
-1\\
\end{bmatrix}
$$
$
\vspace{3mm}
%~~~~~~~~~~~~~~~~~~~~~~~~~~~~~~~~~~~~~~~~~~~~~~~~~~~~~~~~~~~~~~~
\item[4.4.10] Orthonormal vectors are automatically linearly independent.\\
(a) Vector proof: When $c_1q_1 + c_2+q_2 + c_3q_3 = 0$, what dot product leads to $c_1 = 0$? Similarly $c_2 = 0$ and $c_3 = 0$. Thus the $q$'s are independent.\\
\vspace{3mm}
If all the q's are orthonormal then the dot product of $q_1$ with $c_1q_1 + c_2q_2 + c_3q_3 = 0$ gives $c_1 = 0$. Similarly $c_2 = c_3 = 0$.\\ 
\vspace{3mm}
(b) Matrix proof: Show that $Qx = 0$ leads to $x = 0$. Since $Q$ may be rectangular, you can use $Q^T$ but not $Q^{-1}$.\\
%~~~~~~~~~~~~~~~~~~~~~ ANSWER TO 4.4.10 ~~~~~~~~~~~~~~~~~~~~~~~~
\vspace{3mm}
$Qx = 0 \rightarrow Q^TQx = 0 \rightarrow x = 0$\\
\vspace{3mm}
%~~~~~~~~~~~~~~~~~~~~~~~~~~~~~~~~~~~~~~~~~~~~~~~~~~~~~~~~~~~~~~~
\item[4.4.11] (a) Gram-Shmidt: Find orthonormal vectors $q_1$ and $q_2$ in the plane spanned by $a = (1, 3, 4, 5, 7)$ and\\
\hspace{6mm}$b = (-6, 6, 8, 0, 8)$.\\
\vspace{3mm}
$\frac{1}{10}(1, 3, 4, 5, 7) \hspace{3mm} \frac{1}{10}(-7, 3, 4, -5, 1)$\\
\vspace{3mm}
(b) Which vector in this plane is closest to $(1, 0, 0, 0, 0)$?\\
%~~~~~~~~~~~~~~~~~~~~~ ANSWER TO 4.4.11 ~~~~~~~~~~~~~~~~~~~~~~~~
\vspace{3mm}
$
$$
\frac{1}{10}
\begin{bmatrix}
1&-7\\
3&3\\
4&4\\
5&-5\\
7&1\\
\end{bmatrix}
\frac{1}{10}
\begin{bmatrix}
1&3&4&5&7\\
-7&3&4&-5&1\\
\end{bmatrix}
= 
\frac{1}{100}
\begin{bmatrix}
50&-18&-24&40&0\\
-18&18&24&0&24\\
-24&24&32&0&32\\
40&0&0&50&30\\
0&24&32&30&50\\
\end{bmatrix}
\begin{bmatrix}
1\\
0\\
0\\
0\\
0\\
\end{bmatrix}
=
\frac{1}{100}
\begin{bmatrix}
50\\
-18\\
-24\\
40\\
0\\
\end{bmatrix}
=
\begin{bmatrix}
.5\\
-.18\\
-.24\\
.4\\
0\\
\end{bmatrix}
$$
$\\
\vspace{40mm}
%~~~~~~~~~~~~~~~~~~~~~~~~~~~~~~~~~~~~~~~~~~~~~~~~~~~~~~~~~~~~~~~
\item[4.4.15] (a) Find orthonormal vectors $q_1, q_2, q_3$ such that $q_1$, $q_2$ span the column space of\\
\begin{center}
$
$$
A =
\begin{bmatrix}
1&1\\
2&-1\\
-2&4\\
\end{bmatrix}
$$
$\\
\vspace{3mm}
$q_1 = \frac{1}{3} (1, 2, -2) \hspace{3mm} q_2 = \frac{1}{3}(2, 1, 2) \hspace{3mm} q_3 = \frac{1}{3}(2, 2, -1)$\\
\vspace{3mm}
\end{center}
(b) Which of the four fundamental subspaces contains $q_3$?\\
\vspace{3mm}
the left nullspace\\
\vspace{3mm}
(c) Solve $Ax = (1, 2, 7)$ by least squares.\\
%~~~~~~~~~~~~~~~~~~~~~ ANSWER TO 4.4.15 ~~~~~~~~~~~~~~~~~~~~~~~~
\vspace{3mm}
$\hat{x} = (A^TA)^{-1}A^T$
$
$$
\begin{bmatrix}
1\\
2\\
7\\
\end{bmatrix}
=
\left(
\begin{bmatrix}
1&2&-2\\
1&-1&4\\
\end{bmatrix}
\begin{bmatrix}
1&1\\
2&-1\\
-2&4\\
\end{bmatrix}
\right)
\begin{bmatrix}
1&2&-2\\
1&-1&4\\
\end{bmatrix}
\begin{bmatrix}
1\\
2\\
7\\
\end{bmatrix}
\rightarrow
\begin{bmatrix}
9&-9\\
-9&18\\
\end{bmatrix}
^{-1}
\begin{bmatrix}
-9\\
27\\
\end{bmatrix}
=
\frac{1}{9}
\begin{bmatrix}
2&1\\
1&1\\
\end{bmatrix}
\begin{bmatrix}
-9\\
27\\
\end{bmatrix}
=
\frac{1}{9}
\begin{bmatrix}
9\\
18\\
\end{bmatrix}
=
\begin{bmatrix}
1\\
2\\
\end{bmatrix}
$$
$
\vspace{3mm}
%~~~~~~~~~~~~~~~~~~~~~~~~~~~~~~~~~~~~~~~~~~~~~~~~~~~~~~~~~~~~~~~
\item[4.4.23] Find $q_1, q_2, q_3$ (orthonormal) as combinations of $a, b, c$ (independent columns). Then write $A$ as $QR$:\\
\begin{center}
$
$$
c =
\begin{bmatrix}
1&2&4\\
0&0&5\\
0&3&6\\
\end{bmatrix}
$$
$.\\
\end{center}
%~~~~~~~~~~~~~~~~~~~~~ ANSWER TO 4.4.23 ~~~~~~~~~~~~~~~~~~~~~~~~
\vspace{3mm}
A is an invertible matrix so the vectors $q_1 = \left[\begin{smallmatrix} 1\\ 0\\ 0 \end{smallmatrix}\right]$, $q_2 = \left[\begin{smallmatrix} 0\\ 1\\ 0 \end{smallmatrix}\right]$, $q_3 = \left[\begin{smallmatrix} 0\\ 0\\ 1 \end{smallmatrix}\right]$ in the column space and are orthonormal.\\
\vspace{3mm}
$
$$
A =
\begin{bmatrix}
1&0&0\\
0&1&0\\
0&0&1\\
\end{bmatrix}
\begin{bmatrix}
1&2&4\\
0&0&5\\
0&3&6\\
\end{bmatrix}
$$
$
\vspace{3mm}
%~~~~~~~~~~~~~~~~~~~~~~~~~~~~~~~~~~~~~~~~~~~~~~~~~~~~~~~~~~~~~~~
\item[4.4.24] (a) Find a basis for the subspace $S$ in $R^4$ spanned by all solutions of\\
\begin{center}
$x_1 + x_2 + x_3 - x_4 = 0$\\
\vspace{3mm}
$
$$
S = 
\left\{
\begin{bmatrix}
1\\
0\\
0\\
1\\
\end{bmatrix}
,
\begin{bmatrix}
1\\
-1\\
0\\
0\\
\end{bmatrix}
,
\begin{bmatrix}
1\\
0\\
-1\\
0\\
\end{bmatrix}
\right\}
$$
$
\end{center}
(b) Find a basis for the orthogonal comblement $S^{\perp}$.\\
\vspace{3mm}
Since all vectors and all their linear combinations contained in S are orthogonal to the orginal matrix $\left[\begin{smallmatrix} 1\\ 1\\ 1\\ -1 \end{smallmatrix}\right], S\perp$ is the original matrix $\left[\begin{smallmatrix}1\\ 1\\ 1\\ -1\end{smallmatrix}\right]$\\
\vspace{3mm}
(c) Find $b_1$ in $S$ and $b_2$ in $S^{\perp}$ so that $b_1 + b_2 = b = (1, 1, 1, 1)$.\\
%~~~~~~~~~~~~~~~~~~~~~ ANSWER TO 4.4.24 ~~~~~~~~~~~~~~~~~~~~~~~~
\vspace{3mm}

%~~~~~~~~~~~~~~~~~~~~~~~~~~~~~~~~~~~~~~~~~~~~~~~~~~~~~~~~~~~~~~~
\item[4.4.34] $Q = I - 2uu^T$ is a reflection matrix when $u^Tu = 1$. Two reflections give $Q^2 = I$.\\
(a) Show that $Qu = -u$. The mirror is perpendicular to $u$.\\
\vspace{3mm}
$Q = I - 2uu^T \rightarrow Qu = Iu - 2uu^Tu$ Since $u^Tu = 1 \rightarrow Qu = -u$\\
\vspace{3mm}
(b) Find $Qv$ when $u^Tv = 0$. The mirror contains $v$. It reflects to itself.\\
%~~~~~~~~~~~~~~~~~~~~~ ANSWER TO 4.4.34 ~~~~~~~~~~~~~~~~~~~~~~~~
\vspace{3mm}
$Q = I - 2uu^T \rightarrow Qv = Iv - 2uu^Tv$ since $u^Tv = 0 \rightarrow Qv = v$
\vspace{3mm}
%~~~~~~~~~~~~~~~~~~~~~~~~~~~~~~~~~~~~~~~~~~~~~~~~~~~~~~~~~~~~~~~
\end{enumerate}
\end{document}  