\documentclass[10pt,twoside,reqno]{article}
\usepackage[marginparsep=1em]{geometry}
\geometry{lmargin=1.0in,rmargin=1.0in, bmargin=0.75in,  tmargin=0.75in}
\usepackage[usenames,dvipsnames,svgnames,table]{xcolor}
\usepackage{graphicx}
\usepackage{amssymb}
\usepackage{epstopdf}
\usepackage{tikz}
\usepackage{enumerate}
\usepackage{amsthm}
\usepackage{pgfplots}
\usepackage{tikz-3dplot}
\usetikzlibrary{shapes.geometric}
\usepackage{float}
\usepackage{amsmath}
\usepackage{fancyhdr}
\usepackage{lmodern}
\usepackage{chngcntr}
\usepackage{multicol, comment}

\pagestyle{fancy}
\fancyhf{}
\renewcommand{\sectionmark}[1]{\markright{\thesection.\ #1}}
\lhead{\fancyplain{}{}} 
\fancyhead[RE,RO]{MATH 2270}
\fancyfoot[RE,LO]{Dr. Heavilin}
\fancyfoot[LE,RO]{\thepage}


\begin{document}
\begin{flushright}
\begin{minipage}{.25\textwidth}
Dustin Ginos: \\
A01233669\\
Chandler Kinch: \\
A01662772\\
Jeff Wasden: \\
A01657029\\

\today
\end{minipage}
\end{flushright}

\center{\textbf{\underline{Homework 6}}}\\
\vspace{10mm}
\textbf{Chapter 6.1}
\begin{enumerate}
\item[6.1.2]  Find the eigenvalues and the eigenvectors of these two matrices:\\ 
\begin{center}
$
A=
\begin{bmatrix}
1&4\\
2&3\\
\end{bmatrix}
$
\hspace{5mm} and \hspace{5mm}
$
A+I=
\begin{bmatrix}
2&4\\
2&4\\
\end{bmatrix}
$. \\
\end{center}
$A+I$ has the \underline{\hspace{10mm}} eigenvectors as $A$. Its eigenvalues are \underline{\hspace{10mm}} by 1.
\vspace{2mm}
%~~~~~~~~~~~~~~~~~~~~~ ANSWER TO 6.1.2 ~~~~~~~~~~~~~~~~~~~~~~~~
\begin{center}
$A) \hspace{6mm}
\begin{vmatrix}
1- \lambda &4 \\
2&3- \lambda \\
\end{vmatrix}
=(1 - \lambda)(3- \lambda )-8= \lambda^2-4 \lambda -5=0
$ \\
$\lambda= \pm5,-1
\hspace{8mm}$ eigenvectors$=
\begin{bmatrix}
2\\
-1\\
\end{bmatrix},
\begin{bmatrix}
1\\
1\\
\end{bmatrix}
$
\\ \vspace{2mm}
$A+I) \hspace{6mm}
\begin{vmatrix}
2- \lambda &4 \\
2&4- \lambda \\
\end{vmatrix}
=(2 - \lambda)(4- \lambda )-8= \lambda^2-6 \lambda =0
$ \\
$\lambda= 0,6
\hspace{8mm}$ eigenvectors$=
\begin{bmatrix}
2\\
-1\\
\end{bmatrix},
\begin{bmatrix}
1\\
1\\
\end{bmatrix}
$ \\
\end{center}
$A+I$ has the \underline{\hspace{3mm}same\hspace{3mm}} eigenvectors as $A$. Its eigenvalues are \underline{\hspace{3mm}increased\hspace{3mm}} by 1. \\
\vspace{3mm}
%~~~~~~~~~~~~~~~~~~~~~~~~~~~~~~~~~~~~~~~~~~~~~~~~~~~~~~~~~~~~~~
\item[6.1.4]  Compute the eigenvalues and eigenvectors of $A$ and $A^2$:\\ 
\begin{center}
$
A=
\begin{bmatrix}
-1&3\\
2&0\\
\end{bmatrix}
$
\hspace{5mm} and \hspace{5mm}
$
A^2=
\begin{bmatrix}
7&-3\\
-2&6\\
\end{bmatrix}
$. \\
\end{center}
$A^2$ has the same \underline{\hspace{10mm}} as $A$. When $A$ has eigenvalues $\lambda_1$ and $\lambda_2$, $A^2$ has eigenvalues \underline{\hspace{10mm}}. In this example, why is $\lambda_1^2+\lambda_2^2=13$? 
\vspace{2mm}
%~~~~~~~~~~~~~~~~~~~~~ ANSWER TO 6.1.4 ~~~~~~~~~~~~~~~~~~~~~~~~
\begin{center}
$A) \hspace{6mm}
\begin{vmatrix}
-1- \lambda &3 \\
2&- \lambda \\
\end{vmatrix}
=- \lambda(-1- \lambda )-6= \lambda^2 + \lambda -6=0
$ \\
$\lambda= 2,-3
\hspace{8mm}
\begin{bmatrix}
-3&3\\
2&-2\\
\end{bmatrix}
x_1=0
\hspace{8mm}
\begin{bmatrix}
2&3\\
2&3\\
\end{bmatrix}
x_2=0
\hspace{8mm}$ eigenvectors$=
\begin{bmatrix}
1\\
1\\
\end{bmatrix},
\begin{bmatrix}
1\\
- \frac{2}{3}\\
\end{bmatrix}
$
\\ \vspace{2mm}
$A^2) \hspace{6mm}
\begin{vmatrix}
7- \lambda &3 \\
-2&6- \lambda \\
\end{vmatrix}
=(1- \lambda)(6- \lambda )-6= \lambda^2 -13 \lambda +36=0
$ \\
$\lambda= 4,9
\hspace{8mm}
\begin{bmatrix}
3&-3\\
-2&2\\
\end{bmatrix}
x_1=0
\hspace{8mm}
\begin{bmatrix}
-2&-3\\
-2&-3\\
\end{bmatrix}
x_2=0
\hspace{8mm}$ eigenvectors$=
\begin{bmatrix}
1\\
1\\
\end{bmatrix},
\begin{bmatrix}
1\\
- \frac{2}{3}\\
\end{bmatrix}
$
\\ \vspace{2mm}
\end{center}
$A^2$ has the same \underline{\hspace{3mm}eigenvectors\hspace{3mm}} as $A$. When $A$ has eigenvalues $\lambda_1$ and $\lambda_2$, $A^2$ has eigenvalues \underline{\hspace{3mm}$\lambda_1^2$ and $\lambda_2^2$\hspace{3mm}}. $\lambda_1^2+\lambda_2^2=13$ because that is the trace of $A^2$. 
\vspace{3mm}
%~~~~~~~~~~~~~~~~~~~~~~~~~~~~~~~~~~~~~~~~~~~~~~~~~~~~~~~~~~~~~~
\item[6.1.9]  What do you do to the equation $Ax = \lambda x$, in order to prove (a), (b), and (c)? \\ \vspace{2mm}
{\addtolength{\leftskip}{10mm}
(a) $\lambda^2$ is an eigenvalue of $A^2$, as in Problem 4.\\ \vspace{2mm}
%~~~~~~~~~~~~~~~~~~~~~ ANSWER TO 6.1.9a ~~~~~~~~~~~~~~~~~~~~~~~~
{\addtolength{\leftskip}{5mm}
Multiply both sides by $A$. \\
$AAx=A\lambda x \hspace{4mm}\rightarrow\hspace{4mm} A^2x=\lambda Ax \hspace{4mm}\rightarrow\hspace{4mm} A^2x=\lambda \lambda x \hspace{4mm}\rightarrow\hspace{4mm} A^2x=\lambda^2x$. \\
}
\vspace{3mm}
%~~~~~~~~~~~~~~~~~~~~~~~~~~~~~~~~~~~~~~~~~~~~~~~~~~~~~~~~~~~~~~
(b) $\lambda^{-1}$ is an eigenvalue of $A^{-1}$, as in Problem 3.\\ \vspace{2mm}
%~~~~~~~~~~~~~~~~~~~~~ ANSWER TO 6.1.9b ~~~~~~~~~~~~~~~~~~~~~~~~
{\addtolength{\leftskip}{5mm}
Multiply both sides by $A^{-1}$. \\
$A^{-1}Ax=A^{-1}\lambda x \hspace{4mm}\rightarrow\hspace{4mm} x=\lambda A^{-1}x \hspace{4mm}\rightarrow\hspace{4mm} \frac{1}{\lambda}x=A^{-1}x$. \\
}
\vspace{3mm}
%~~~~~~~~~~~~~~~~~~~~~~~~~~~~~~~~~~~~~~~~~~~~~~~~~~~~~~~~~~~~~~
(c) $\lambda+1$ is an eigenvalue of $A+I$, as in Problem 2.\\ \vspace{2mm}
%~~~~~~~~~~~~~~~~~~~~~ ANSWER TO 6.1.9c ~~~~~~~~~~~~~~~~~~~~~~~~
{\addtolength{\leftskip}{5mm}
Add $Ix=x$ to both sides. \\
$Ix+Ax=x+\lambda x \hspace{4mm}\rightarrow\hspace{4mm} (A+I)x=(\lambda +1)x$. \\
}
\vspace{3mm}
%~~~~~~~~~~~~~~~~~~~~~~~~~~~~~~~~~~~~~~~~~~~~~~~~~~~~~~~~~~~~~~
}
\item[6.1.12] Find three eigenvectors for this matrix $P$ (projection matrices have $\lambda = 1$ and 0): \\ 
\begin{center}
\textbf{Projection matrix}
\hspace{20mm}
$
P=
\begin{bmatrix}
.2&.4&0\\
.4&.8&0\\
0&0&1\\
\end{bmatrix}
$. \\
\end{center}
If two eigenvectors share the same $\lambda$, so do all their linear combinations. Find an eigenvector of $P$ with no zero components. \\
\vspace{2mm}
%~~~~~~~~~~~~~~~~~~~~~ ANSWER TO 6.1.12 ~~~~~~~~~~~~~~~~~~~~~~~~
\begin{center}
$
\lambda=1 \hspace{4mm}
\begin{bmatrix}
-.8&.4&0\\
.4&-.2&0\\
0&0&0\\
\end{bmatrix}
\begin{bmatrix}
x_1\\
x_2\\
x_3\\
\end{bmatrix}
=0 \hspace{4mm}
$
eigenvectors$=
\begin{bmatrix}
0\\
0\\
1\\
\end{bmatrix},
\begin{bmatrix}
1\\
2\\
0\\
\end{bmatrix}
$ \\ \vspace{2mm}
$
\lambda =0 \hspace{4mm}
\begin{bmatrix}
.2&.4&0\\
.4&.8&0\\
0&0&1\\
\end{bmatrix}
\begin{bmatrix}
x_1\\
x_2\\
x_3\\
\end{bmatrix}
=0 \hspace{4mm}
$
eigenvector$=
\begin{bmatrix}
-2\\
1\\
0\\
\end{bmatrix}
$ \\
\end{center}
Combine the eigenvectors when $\lambda =1$ to get an eigenvector of $P$ with no zero components: 
$
\begin{bmatrix}
1\\
2\\
1\\
\end{bmatrix}
$ \\
\vspace{3mm}
%~~~~~~~~~~~~~~~~~~~~~~~~~~~~~~~~~~~~~~~~~~~~~~~~~~~~~~~~~~~~~~
\item[6.1.13] From the unit vector \textbf{$u$} = ($\frac{1}{6}, \frac{1}{6}, \frac{3}{6}, \frac{5}{6}$) construct the rank one projection matrix $P = $\textbf{$uu^T$}. This matrix has $P^2 = P$ because \textbf{$u^Tu$} = 1. \\ \vspace{2mm}
{\addtolength{\leftskip}{10mm}
(a) $Pu=u$ comes from $(uu^T)u=u(\underline{\hspace{10mm}})$. Then $u$ is an eigenvector with $\lambda = 1$.\\ \vspace{2mm}
%~~~~~~~~~~~~~~~~~~~~~ ANSWER TO 6.1.13a ~~~~~~~~~~~~~~~~~~~~~~~~
{\addtolength{\leftskip}{5mm}
$(uu^T)u=u(\underline{\hspace{3mm}u^Tu\hspace{3mm}})$ \\
}
\vspace{3mm}
%~~~~~~~~~~~~~~~~~~~~~~~~~~~~~~~~~~~~~~~~~~~~~~~~~~~~~~~~~~~~~~
(b) If $v$ is perpendicular to $u$ show that $Pv=0$. Then $\lambda = 0$.\\ \vspace{2mm}
%~~~~~~~~~~~~~~~~~~~~~ ANSWER TO 6.1.13b ~~~~~~~~~~~~~~~~~~~~~~~~
{\addtolength{\leftskip}{5mm}
$Pv=(uu^T)v=u(u^Tv)=u*0=0$ \\
}
\vspace{3mm}
%~~~~~~~~~~~~~~~~~~~~~~~~~~~~~~~~~~~~~~~~~~~~~~~~~~~~~~~~~~~~~~
(c) Find three independent eigenvectors of $P$ all with eigenvalue $\lambda=0$.\\ \vspace{2mm}
%~~~~~~~~~~~~~~~~~~~~~ ANSWER TO 6.1.13c ~~~~~~~~~~~~~~~~~~~~~~~~
{\addtolength{\leftskip}{5mm}
$
\begin{bmatrix}
1\\
1\\
1\\
-1\\
\end{bmatrix},
\begin{bmatrix}
-1\\
1\\
0\\
0\\
\end{bmatrix},
\begin{bmatrix}
-5\\
0\\
0\\
1\\
\end{bmatrix}
$ all have $\lambda=0$ and are independent. \\
}
\vspace{3mm}
%~~~~~~~~~~~~~~~~~~~~~~~~~~~~~~~~~~~~~~~~~~~~~~~~~~~~~~~~~~~~~~
}
\item[6.1.15] Every permutation matrix leaves $x = (1,1, ... ,1)$ unchanged. Then $\lambda = 1$. Find two more $\lambda$'s (possibly complex) for these permutations, from $det(P - \lambda I) = 0$: \\ 
\begin{center}
$
P=
\begin{bmatrix}
0&1&0\\
0&0&1\\
1&0&0\\
\end{bmatrix}
$
\hspace{5mm} and \hspace{5mm}
$
\begin{bmatrix}
0&0&1\\
0&1&0\\
1&0&0\\
\end{bmatrix}
$. \\
\end{center}
\vspace{2mm}
%~~~~~~~~~~~~~~~~~~~~~ ANSWER TO 6.1.15 ~~~~~~~~~~~~~~~~~~~~~~~~
\begin{center}
$
P=
\begin{bmatrix}
0&1&0\\
0&0&1\\
1&0&0\\
\end{bmatrix}
\hspace{3mm}
\rightarrow
\hspace{3mm}
\begin{vmatrix}
-\lambda&1&0\\
0&-\lambda&1\\
1&0&-\lambda\\
\end{vmatrix}
=-\lambda^3+1=0
\hspace{5mm}
\lambda=1
$ \\
$
P=
\begin{bmatrix}
0&0&1\\
0&1&0\\
1&0&0\\
\end{bmatrix}
\hspace{3mm}
\rightarrow
\hspace{3mm}
\begin{vmatrix}
-\lambda&0&1\\
0&1-\lambda&0\\
1&0&-\lambda\\
\end{vmatrix}
=\lambda^3-\lambda^2-\lambda+1=0
\hspace{5mm}
\lambda=1,1,-1
$ \\
\end{center}

\vspace{3mm}
%~~~~~~~~~~~~~~~~~~~~~~~~~~~~~~~~~~~~~~~~~~~~~~~~~~~~~~~~~~~~~~
\item[6.1.19] A 3 by 3 matrix $B$ is known to have eigenvalues 0,1,2. This information is enough to find three of these (give the answers where possible): \\ \vspace{2mm}
{\addtolength{\leftskip}{10mm}
(a) the rank of $B$ \\ \vspace{2mm}
%~~~~~~~~~~~~~~~~~~~~~ ANSWER TO 6.1.19a ~~~~~~~~~~~~~~~~~~~~~~~~
{\addtolength{\leftskip}{5mm}
$B$ is a rank two because it has a $\lambda=0$. \\
}
\vspace{3mm}
%~~~~~~~~~~~~~~~~~~~~~~~~~~~~~~~~~~~~~~~~~~~~~~~~~~~~~~~~~~~~~~
(b) the determinate of $B^TB$ \\ \vspace{2mm}
%~~~~~~~~~~~~~~~~~~~~~ ANSWER TO 6.1.19b ~~~~~~~~~~~~~~~~~~~~~~~~
{\addtolength{\leftskip}{5mm}
$|B^TB|$ because $B^TB$ is singular. \\
}
\vspace{3mm}
%~~~~~~~~~~~~~~~~~~~~~~~~~~~~~~~~~~~~~~~~~~~~~~~~~~~~~~~~~~~~~~
(c) the eigenvalues of $B^TB$ \\ \vspace{2mm}
%~~~~~~~~~~~~~~~~~~~~~ ANSWER TO 6.1.19c ~~~~~~~~~~~~~~~~~~~~~~~~
{\addtolength{\leftskip}{5mm}
Can't determine. \\
}
\vspace{3mm}
%~~~~~~~~~~~~~~~~~~~~~~~~~~~~~~~~~~~~~~~~~~~~~~~~~~~~~~~~~~~~~~
(d) the eigenvalues of $(B^2+I)^-1$. \\ \vspace{2mm}
%~~~~~~~~~~~~~~~~~~~~~ ANSWER TO 6.1.19d ~~~~~~~~~~~~~~~~~~~~~~~~
{\addtolength{\leftskip}{5mm}
$\lambda$'s of $(B^2+I)^{-1}$ are $\lambda=1,\frac{1}{2},\frac{1}{5}$. \\
}
\vspace{3mm}
%~~~~~~~~~~~~~~~~~~~~~~~~~~~~~~~~~~~~~~~~~~~~~~~~~~~~~~~~~~~~~~
}
\item[6.1.21] \textbf{The eigenvalues of $A$ equal the eigenvalues of $A^T$.} This is because $det(A - \lambda I)$ equals $det(A^T - \lambda I)$. That is true because \underline{\hspace{10mm}}. Show by an example that the eigenvectors of $A$ and $A^T$ are \textit{not} the same.\\ \vspace{2mm}
%~~~~~~~~~~~~~~~~~~~~~ ANSWER TO 6.1.21 ~~~~~~~~~~~~~~~~~~~~~~~~
\begin{center}
It is true because every square matrix has the property $|A|=|A^T|$. \\
\vspace{2mm}
$A=
\begin{bmatrix}
1&0\\
1&2\\
\end{bmatrix}
$
\hspace{4mm} and \hspace{4mm}
$A^T=
\begin{bmatrix}
1&1\\
0&2\\
\end{bmatrix}
$ do not have the same eigen vectors. \\
Eigenvectors of $A=
\begin{bmatrix}
0\\
1\\
\end{bmatrix},
\begin{bmatrix}
-1\\
1\\
\end{bmatrix}
$ while $A^T$ has eigenvectors $
\begin{bmatrix}
1\\
1\\
\end{bmatrix},
\begin{bmatrix}
1\\
0\\
\end{bmatrix}
$. \\
\end{center}

\vspace{3mm}
%~~~~~~~~~~~~~~~~~~~~~~~~~~~~~~~~~~~~~~~~~~~~~~~~~~~~~~~~~~~~~~
\item[6.1.29] (Review) Find the eigenvalues of $A$, $B$, and $C$: \\
\begin{center}
$
A=
\begin{bmatrix}
1&2&3\\
0&4&5\\
0&0&6\\
\end{bmatrix}
$
\hspace{5mm} and \hspace{5mm}
$
B=
\begin{bmatrix}
0&0&1\\
0&2&0\\
3&0&0\\
\end{bmatrix}
$
\hspace{5mm} and \hspace{5mm}
$
C=
\begin{bmatrix}
2&2&2\\
2&2&2\\
2&2&2\\
\end{bmatrix}
$. \\
\end{center}
\vspace{2mm}
%~~~~~~~~~~~~~~~~~~~~~ ANSWER TO 6.1.29 ~~~~~~~~~~~~~~~~~~~~~~~~
\begin{center}
$
A)\hspace{5mm}
|A-\lambda I|=
\begin{vmatrix}
1-\lambda&2&3\\
0&4-\lambda&5\\
0&0&6-\lambda\\
\end{vmatrix}
=(1-\lambda)(4-\lambda)(6-\lambda)=0
\hspace{4mm} \lambda=1,4,6
$ \\ \vspace{2mm}
$
B)\hspace{5mm}
|B-\lambda I|=
\begin{vmatrix}
-\lambda&0&1\\
0&2-\lambda&0\\
3&0&-\lambda\\
\end{vmatrix}
=(\lambda^2-3)(\lambda+2)=0
\hspace{4mm} \lambda=2,\pm\sqrt{3}
$ \\ \vspace{2mm}
C is a rank one matrix, meaning that two of its $\lambda$'s are zero. The last $\lambda$ is the sum of the diagonals.
\hspace{6mm} $\lambda=0,0,6$ \\
\end{center}

\vspace{3mm}
%~~~~~~~~~~~~~~~~~~~~~~~~~~~~~~~~~~~~~~~~~~~~~~~~~~~~~~~~~~~~~~ 
\end{enumerate}
\vspace{5mm}
\textbf{Chapter 6.2}
\begin{enumerate}
\item[6.2.2] If $A$ has $\lambda_1 = 2$ with eigenvector $x_1=
\begin{bmatrix}
1\\
0\\
\end{bmatrix}$ and $\lambda_2=5$ with $x_2=
\begin{bmatrix}
1\\
1\\
\end{bmatrix}$, use $S \Lambda S^{-1}$ to find $A$. No other matrix has the same $\lambda$'s and $x$'s. \\ \vspace{2mm}
%~~~~~~~~~~~~~~~~~~~~~ ANSWER TO 6.2.2 ~~~~~~~~~~~~~~~~~~~~~~~~


\vspace{3mm}
%~~~~~~~~~~~~~~~~~~~~~~~~~~~~~~~~~~~~~~~~~~~~~~~~~~~~~~~~~~~~~~ 
\item[6.2.8] Diagonalize the Fibonacci matrix by completing $S^{-1}$: \\
\begin{center}
$
\begin{bmatrix}
1&1\\
1&0\\
\end{bmatrix}
=
\begin{bmatrix}
\lambda_1&\lambda_2\\
1&1\\
\end{bmatrix}
\begin{bmatrix}
\lambda_1&0\\
0&\lambda_2\\
\end{bmatrix}
\begin{bmatrix}
&\\
&\\
\end{bmatrix}
$. \\
\end{center}
Do the multiplication $S \Lambda S^{-1}
\begin{bmatrix}
1\\
0\\
\end{bmatrix}$ to find its second component. This is the \textit{k}th Fibonacci number $F_k=(\lambda_1^k-\lambda_2^k)/(\lambda_1-\lambda_2)$.
\vspace{2mm}
%~~~~~~~~~~~~~~~~~~~~~ ANSWER TO 6.2.8 ~~~~~~~~~~~~~~~~~~~~~~~~


\vspace{3mm}
%~~~~~~~~~~~~~~~~~~~~~~~~~~~~~~~~~~~~~~~~~~~~~~~~~~~~~~~~~~~~~~ 
\item[6.2.9] Suppose $G_{k+2}$ is the \textit{average} of the two previous numbers $G_{k+1}$ and $G_k$: \\ 
\begin{center}
$G_{k+2}=\frac{1}{2}G_{k+1}+\frac{1}{2}G_k \hspace{8mm} G_{k+1}=G_{k+1}$
\hspace{10mm} and \hspace{10mm}
$
\begin{bmatrix}
G_{k+2}\\
G_{k+1}\\
\end{bmatrix}
=
\begin{bmatrix}
A\\
\end{bmatrix}
\begin{bmatrix}
G_{k+1}\\
G_k\\
\end{bmatrix}
$. \\
\end{center}
\vspace{2mm}
{\addtolength{\leftskip}{10mm}
(a) Find the eigenvalues and eigenvectors of $A$. \\ \vspace{2mm}
%~~~~~~~~~~~~~~~~~~~~~ ANSWER TO 6.2.9a ~~~~~~~~~~~~~~~~~~~~~~~~
{\addtolength{\leftskip}{5mm}
. \\
}
\vspace{3mm}
%~~~~~~~~~~~~~~~~~~~~~~~~~~~~~~~~~~~~~~~~~~~~~~~~~~~~~~~~~~~~~~
(b) Find the limist as $n \rightarrow \infty$ of the matrices $A^n=S \Lambda S^{-1}$. \\ \vspace{2mm}
%~~~~~~~~~~~~~~~~~~~~~ ANSWER TO 6.2.9b ~~~~~~~~~~~~~~~~~~~~~~~~
{\addtolength{\leftskip}{5mm}
. \\
}
\vspace{3mm}
%~~~~~~~~~~~~~~~~~~~~~~~~~~~~~~~~~~~~~~~~~~~~~~~~~~~~~~~~~~~~~~
(c) If $G_0=0$ and $G_1=1$ show that the Gibonacci numbers approach $\frac{2}{3}$. \\ \vspace{2mm}
%~~~~~~~~~~~~~~~~~~~~~ ANSWER TO 6.2.9c ~~~~~~~~~~~~~~~~~~~~~~~~
{\addtolength{\leftskip}{5mm}
. \\
}
\vspace{3mm}
%~~~~~~~~~~~~~~~~~~~~~~~~~~~~~~~~~~~~~~~~~~~~~~~~~~~~~~~~~~~~~~
}
\item[6.2.10] Prove that every third Fibonacci number in 0,1,1,2,3,... is even.\\ \vspace{2mm}
%~~~~~~~~~~~~~~~~~~~~~ ANSWER TO 6.2.10 ~~~~~~~~~~~~~~~~~~~~~~~~


\vspace{3mm}
%~~~~~~~~~~~~~~~~~~~~~~~~~~~~~~~~~~~~~~~~~~~~~~~~~~~~~~~~~~~~~~
\item[6.2.11] True or false: If the eigenvalues of $A$ are 2,2,5 then the matrix is certainly \\ 
\begin{center}
(a) invertible \hspace{10mm} (b) diagonalizable \hspace{10mm} (c) not diagonalizable. \\
\end{center}
\vspace{2mm}
%~~~~~~~~~~~~~~~~~~~~~ ANSWER TO 6.2.11 ~~~~~~~~~~~~~~~~~~~~~~~~


\vspace{3mm}
%~~~~~~~~~~~~~~~~~~~~~~~~~~~~~~~~~~~~~~~~~~~~~~~~~~~~~~~~~~~~~~
\item[6.2.15] $A^k=S \Lambda S^{-1}$ approaches the zero matrix as $k \rightarrow \infty$ if and only if every $\lambda$ has absolute value less than \underline{\hspace{10mm}}. Which of these matrices has $A^k \rightarrow 0$? \\ 
\begin{center}
$
A_1=
\begin{bmatrix}
.6&.9\\
.4&.1\\
\end{bmatrix}
$
\hspace{8mm} and \hspace{8mm}
$
A_2=
\begin{bmatrix}
.6&.9\\
.1&.6\\
\end{bmatrix}
$. \\
\end{center}
\vspace{2mm}
%~~~~~~~~~~~~~~~~~~~~~ ANSWER TO 6.2.15 ~~~~~~~~~~~~~~~~~~~~~~~~


\vspace{3mm}
%~~~~~~~~~~~~~~~~~~~~~~~~~~~~~~~~~~~~~~~~~~~~~~~~~~~~~~~~~~~~~~
\item[6.2.16] (Recommended) Find $\Lambda$ and $S$ to diagonalize $A_1$ in Problem 15. What is the limit of $\Lambda^k$ as $k \rightarrow \infty$? What is the limit of $S \Lambda^k S^{-1}$? In the columns of this limiting matrix you see the \underline{\hspace{10mm}}. \\ \vspace{2mm}
%~~~~~~~~~~~~~~~~~~~~~ ANSWER TO 6.2.16 ~~~~~~~~~~~~~~~~~~~~~~~~


\vspace{3mm}
%~~~~~~~~~~~~~~~~~~~~~~~~~~~~~~~~~~~~~~~~~~~~~~~~~~~~~~~~~~~~~~
\item[6.2.19] Diagonalize $B$ and compute $S \Lambda^k S^{-1}$ to prove this formula for $B^k$: \\
\begin{center}
$
B=
\begin{bmatrix}
5&1\\
0&4\\
\end{bmatrix}
$
\hspace{8mm} has \hspace{8mm}
$
B^k=
\begin{bmatrix}
5^k&5^k-4^k\\
0&4^k\\
\end{bmatrix}
$. \\
\end{center}
\vspace{2mm}
%~~~~~~~~~~~~~~~~~~~~~ ANSWER TO 6.2.19 ~~~~~~~~~~~~~~~~~~~~~~~~


\vspace{3mm}
%~~~~~~~~~~~~~~~~~~~~~~~~~~~~~~~~~~~~~~~~~~~~~~~~~~~~~~~~~~~~~~
\item[6.2.36] The \textit{n}th power of rotation through $\theta$ is rotation through $n \theta$: \\ 
\begin{center}
$
A^n=
\begin{bmatrix}
\cos\theta&-\sin\theta\\
\sin\theta&\cos\theta\\
\end{bmatrix}^n
=
\begin{bmatrix}
\cos n\theta&-\sin n\theta\\
\sin n\theta&\cos n\theta\\
\end{bmatrix}
$. \\
\end{center}
Prove that neat formula by diagonalizing $A=S \Lambda S^{-1}$. The eigenvectors (columns of $S$) are (1,$i$) and ($i$,1). You need to know Euler's formula $e^{i\theta}=\cos\theta+i\sin\theta$.
\vspace{2mm}
%~~~~~~~~~~~~~~~~~~~~~ ANSWER TO 6.2.36 ~~~~~~~~~~~~~~~~~~~~~~~~


\vspace{3mm}
%~~~~~~~~~~~~~~~~~~~~~~~~~~~~~~~~~~~~~~~~~~~~~~~~~~~~~~~~~~~~~~
\end{enumerate}
\vspace{5mm}
\textbf{Chapter 6.3}
\begin{enumerate}
\item[6.3.1]  .\\ \vspace{2mm}
%~~~~~~~~~~~~~~~~~~~~~ ANSWER TO 6.3.1 ~~~~~~~~~~~~~~~~~~~~~~~~


\vspace{3mm}
%~~~~~~~~~~~~~~~~~~~~~~~~~~~~~~~~~~~~~~~~~~~~~~~~~~~~~~~~~~~~~~ 
\end{enumerate}
\vspace{5mm}
\textbf{Chapter 6.4}
\begin{enumerate}
\item[6.4.4] Find an orthogonal matrix $Q$ that diagonalizes $A=
\begin{bmatrix}
-2&6\\
6&7\\
\end{bmatrix}$. What is $\Lambda$?
\\ \vspace{2mm}
%~~~~~~~~~~~~~~~~~~~~~ ANSWER TO 6.4.4 ~~~~~~~~~~~~~~~~~~~~~~~~


\vspace{3mm}
%~~~~~~~~~~~~~~~~~~~~~~~~~~~~~~~~~~~~~~~~~~~~~~~~~~~~~~~~~~~~~~ 
\item[6.4.6] Find \textit{all} orthogonal matrices that diagonalize $A = 
\begin{bmatrix}
9&12\\
12&16\\
\end{bmatrix}$. \\ \vspace{2mm}
%~~~~~~~~~~~~~~~~~~~~~ ANSWER TO 6.4.6 ~~~~~~~~~~~~~~~~~~~~~~~~


\vspace{3mm}
%~~~~~~~~~~~~~~~~~~~~~~~~~~~~~~~~~~~~~~~~~~~~~~~~~~~~~~~~~~~~~~ 
\item[6.4.11] Write $A$ and $B$ in the form $\lambda_1x_1x_1^T+\lambda_2x_2x_2^T$ of the spectral theorem $Q \Lambda Q^T$: \\
\begin{center}
$
A=
\begin{bmatrix}
3&1\\
1&3\\
\end{bmatrix}
\hspace{8mm}
B=
\begin{bmatrix}
9&12\\
12&16\\
\end{bmatrix}
$
\hspace{6mm} (keep $||x_1||=||x_2||=1$). \\
\end{center}
\vspace{2mm}
%~~~~~~~~~~~~~~~~~~~~~ ANSWER TO 6.4.11 ~~~~~~~~~~~~~~~~~~~~~~~~


\vspace{3mm}
%~~~~~~~~~~~~~~~~~~~~~~~~~~~~~~~~~~~~~~~~~~~~~~~~~~~~~~~~~~~~~~ 
\item[6.4.21] \textbf{True} (with reason) or \textbf{false} (with example). "Orthonormal" is not assumed. \\ \vspace{2mm}
{\addtolength{\leftskip}{10mm}
(a) A matrix with real eigenvalues and eigenvectors is symmetric.\\ \vspace{2mm}
%~~~~~~~~~~~~~~~~~~~~~ ANSWER TO 6.4.21a ~~~~~~~~~~~~~~~~~~~~~~~~
{\addtolength{\leftskip}{5mm}
. \\
}
\vspace{3mm}
%~~~~~~~~~~~~~~~~~~~~~~~~~~~~~~~~~~~~~~~~~~~~~~~~~~~~~~~~~~~~~~
(b) A matrix with real eigenvalues and orthogonal eigenvectors is symmetric.\\ \vspace{2mm}
%~~~~~~~~~~~~~~~~~~~~~ ANSWER TO 6.4.21b ~~~~~~~~~~~~~~~~~~~~~~~~
{\addtolength{\leftskip}{5mm}
. \\
}
\vspace{3mm}
%~~~~~~~~~~~~~~~~~~~~~~~~~~~~~~~~~~~~~~~~~~~~~~~~~~~~~~~~~~~~~~
(c) The inverse of a symmetric matrix is symmetric.\\ \vspace{2mm}
%~~~~~~~~~~~~~~~~~~~~~ ANSWER TO 6.4.21c ~~~~~~~~~~~~~~~~~~~~~~~~
{\addtolength{\leftskip}{5mm}
. \\
}
\vspace{3mm}
%~~~~~~~~~~~~~~~~~~~~~~~~~~~~~~~~~~~~~~~~~~~~~~~~~~~~~~~~~~~~~~
(d) The eigenvector matrix $S$ of a symmetric matrix is symmetric.\\ \vspace{2mm}
%~~~~~~~~~~~~~~~~~~~~~ ANSWER TO 6.4.21d ~~~~~~~~~~~~~~~~~~~~~~~~
{\addtolength{\leftskip}{5mm}
. \\
}
\vspace{3mm}
%~~~~~~~~~~~~~~~~~~~~~~~~~~~~~~~~~~~~~~~~~~~~~~~~~~~~~~~~~~~~~~
}
%~~~~~~~~~~~~~~~~~~~~~ ANSWER TO 6.4.21 ~~~~~~~~~~~~~~~~~~~~~~~~


\vspace{3mm}
%~~~~~~~~~~~~~~~~~~~~~~~~~~~~~~~~~~~~~~~~~~~~~~~~~~~~~~~~~~~~~~ 
\end{enumerate}
\vspace{5mm}
\textbf{Chapter 6.5}
\begin{enumerate}
\item[6.5.7]  .\\ \vspace{2mm}
%~~~~~~~~~~~~~~~~~~~~~ ANSWER TO 6.5.7 ~~~~~~~~~~~~~~~~~~~~~~~~


\vspace{3mm}
%~~~~~~~~~~~~~~~~~~~~~~~~~~~~~~~~~~~~~~~~~~~~~~~~~~~~~~~~~~~~~~ 
\end{enumerate}
\vspace{5mm}
\textbf{Chapter 6.6}
\begin{enumerate}
\item[6.6.17]  .\\ \vspace{2mm}
%~~~~~~~~~~~~~~~~~~~~~ ANSWER TO 6.6.17 ~~~~~~~~~~~~~~~~~~~~~~~~


\vspace{3mm}
%~~~~~~~~~~~~~~~~~~~~~~~~~~~~~~~~~~~~~~~~~~~~~~~~~~~~~~~~~~~~~~ 
\end{enumerate}
\vspace{5mm}
\textbf{Chapter 6.7}
\begin{enumerate}
\item[6.7.4]  .\\ \vspace{2mm}
%~~~~~~~~~~~~~~~~~~~~~ ANSWER TO 6.7.4 ~~~~~~~~~~~~~~~~~~~~~~~~


\vspace{3mm}
%~~~~~~~~~~~~~~~~~~~~~~~~~~~~~~~~~~~~~~~~~~~~~~~~~~~~~~~~~~~~~~ 
\end{enumerate}
\end{document}  