\documentclass[10pt,twoside,reqno]{article}
\usepackage[marginparsep=1em]{geometry}
\geometry{lmargin=1.0in,rmargin=1.0in, bmargin=0.75in,  tmargin=0.75in}
\usepackage[usenames,dvipsnames,svgnames,table]{xcolor}
\usepackage{graphicx}
\usepackage{amssymb}
\usepackage{epstopdf}
\usepackage{tikz}
\usepackage{enumerate}
\usepackage{amsthm}
\usepackage{pgfplots}
\usepackage{tikz-3dplot}
\usetikzlibrary{shapes.geometric}
\usepackage{float}
\usepackage{amsmath}
\usepackage{fancyhdr}
\usepackage{lmodern}
\usepackage{chngcntr}
\usepackage{multicol, comment}

\pagestyle{fancy}
\fancyhf{}
\renewcommand{\sectionmark}[1]{\markright{\thesection.\ #1}}
\lhead{\fancyplain{}{}} 
\fancyhead[RE,RO]{MATH 2270}
\fancyfoot[RE,LO]{Dr. Heavilin}
\fancyfoot[LE,RO]{\thepage}


\begin{document}
\begin{flushright}
\begin{minipage}{.25\textwidth}
Dustin Ginos: \\
A01233669\\
Chandler Kinch: \\
A01662772\\
Jeff Wasden: \\
A01657029\\

\today
\end{minipage}
\end{flushright}

\center{\textbf{\underline{Homework 6}}}\\
\vspace{10mm}
\textbf{Chapter 6.1}
\begin{enumerate}
\item[6.1.2]  Find the eigenvalues and the eigenvectors of these two matrices:\\ 
\begin{center}
$
A=
\begin{bmatrix}
1&4\\
2&3\\
\end{bmatrix}
$
\hspace{5mm} and \hspace{5mm}
$
A+I=
\begin{bmatrix}
2&4\\
2&4\\
\end{bmatrix}
$. \\
\end{center}
$A+I$ has the \underline{\hspace{10mm}} eigenvectors as $A$. Its eigenvalues are \underline{\hspace{10mm}} by 1.
\vspace{2mm}
%~~~~~~~~~~~~~~~~~~~~~ ANSWER TO 6.1.2 ~~~~~~~~~~~~~~~~~~~~~~~~
\begin{center}
$A) \hspace{6mm}
\begin{vmatrix}
1- \lambda &4 \\
2&3- \lambda \\
\end{vmatrix}
=(1 - \lambda)(3- \lambda )-8= \lambda^2-4 \lambda -5=0
$ \\
$\lambda= 5,-1
\hspace{8mm}$ eigenvectors$=
\begin{bmatrix}
2\\
-1\\
\end{bmatrix},
\begin{bmatrix}
1\\
1\\
\end{bmatrix}
$
\\ \vspace{2mm}
$A+I) \hspace{6mm}
\begin{vmatrix}
2- \lambda &4 \\
2&4- \lambda \\
\end{vmatrix}
=(2 - \lambda)(4- \lambda )-8= \lambda^2-6 \lambda =0
$ \\
$\lambda= 0,6
\hspace{8mm}$ eigenvectors$=
\begin{bmatrix}
2\\
-1\\
\end{bmatrix},
\begin{bmatrix}
1\\
1\\
\end{bmatrix}
$ \\
\end{center}
$A+I$ has the \underline{\hspace{3mm}same\hspace{3mm}} eigenvectors as $A$. Its eigenvalues are \underline{\hspace{3mm}increased\hspace{3mm}} by 1. \\
\vspace{3mm}
%~~~~~~~~~~~~~~~~~~~~~~~~~~~~~~~~~~~~~~~~~~~~~~~~~~~~~~~~~~~~~~
\item[6.1.4]  Compute the eigenvalues and eigenvectors of $A$ and $A^2$:\\ 
\begin{center}
$
A=
\begin{bmatrix}
-1&3\\
2&0\\
\end{bmatrix}
$
\hspace{5mm} and \hspace{5mm}
$
A^2=
\begin{bmatrix}
7&-3\\
-2&6\\
\end{bmatrix}
$. \\
\end{center}
$A^2$ has the same \underline{\hspace{10mm}} as $A$. When $A$ has eigenvalues $\lambda_1$ and $\lambda_2$, $A^2$ has eigenvalues \underline{\hspace{10mm}}. In this example, why is $\lambda_1^2+\lambda_2^2=13$? 
\vspace{2mm}
%~~~~~~~~~~~~~~~~~~~~~ ANSWER TO 6.1.4 ~~~~~~~~~~~~~~~~~~~~~~~~
\begin{center}
$A) \hspace{6mm}
\begin{vmatrix}
-1- \lambda &3 \\
2&- \lambda \\
\end{vmatrix}
=- \lambda(-1- \lambda )-6= \lambda^2 + \lambda -6=0
$ \\
$\lambda= 2,-3
\hspace{8mm}
\begin{bmatrix}
-3&3\\
2&-2\\
\end{bmatrix}
x_1=0
\hspace{8mm}
\begin{bmatrix}
2&3\\
2&3\\
\end{bmatrix}
x_2=0
\hspace{8mm}$ eigenvectors$=
\begin{bmatrix}
1\\
1\\
\end{bmatrix},
\begin{bmatrix}
1\\
- \frac{2}{3}\\
\end{bmatrix}
$
\\ \vspace{2mm}
$A^2) \hspace{6mm}
\begin{vmatrix}
7- \lambda &3 \\
-2&6- \lambda \\
\end{vmatrix}
=(1- \lambda)(6- \lambda )-6= \lambda^2 -13 \lambda +36=0
$ \\
$\lambda= 4,9
\hspace{8mm}
\begin{bmatrix}
3&-3\\
-2&2\\
\end{bmatrix}
x_1=0
\hspace{8mm}
\begin{bmatrix}
-2&-3\\
-2&-3\\
\end{bmatrix}
x_2=0
\hspace{8mm}$ eigenvectors$=
\begin{bmatrix}
1\\
1\\
\end{bmatrix},
\begin{bmatrix}
1\\
- \frac{2}{3}\\
\end{bmatrix}
$
\\ \vspace{2mm}
\end{center}
$A^2$ has the same \underline{\hspace{3mm}eigenvectors\hspace{3mm}} as $A$. When $A$ has eigenvalues $\lambda_1$ and $\lambda_2$, $A^2$ has eigenvalues \underline{\hspace{3mm}$\lambda_1^2$ and $\lambda_2^2$\hspace{3mm}}. $\lambda_1^2+\lambda_2^2=13$ because that is the trace of $A^2$. 
\vspace{3mm}
%~~~~~~~~~~~~~~~~~~~~~~~~~~~~~~~~~~~~~~~~~~~~~~~~~~~~~~~~~~~~~~
\item[6.1.9]  What do you do to the equation $Ax = \lambda x$, in order to prove (a), (b), and (c)? \\ \vspace{2mm}
{\addtolength{\leftskip}{10mm}
(a) $\lambda^2$ is an eigenvalue of $A^2$, as in Problem 4.\\ \vspace{2mm}
%~~~~~~~~~~~~~~~~~~~~~ ANSWER TO 6.1.9a ~~~~~~~~~~~~~~~~~~~~~~~~
{\addtolength{\leftskip}{5mm}
Multiply both sides by $A$. \\
$AAx=A\lambda x \hspace{4mm}\rightarrow\hspace{4mm} A^2x=\lambda Ax \hspace{4mm}\rightarrow\hspace{4mm} A^2x=\lambda \lambda x \hspace{4mm}\rightarrow\hspace{4mm} A^2x=\lambda^2x$. \\
}
\vspace{3mm}
%~~~~~~~~~~~~~~~~~~~~~~~~~~~~~~~~~~~~~~~~~~~~~~~~~~~~~~~~~~~~~~
(b) $\lambda^{-1}$ is an eigenvalue of $A^{-1}$, as in Problem 3.\\ \vspace{2mm}
%~~~~~~~~~~~~~~~~~~~~~ ANSWER TO 6.1.9b ~~~~~~~~~~~~~~~~~~~~~~~~
{\addtolength{\leftskip}{5mm}
Multiply both sides by $A^{-1}$. \\
$A^{-1}Ax=A^{-1}\lambda x \hspace{4mm}\rightarrow\hspace{4mm} x=\lambda A^{-1}x \hspace{4mm}\rightarrow\hspace{4mm} \frac{1}{\lambda}x=A^{-1}x$. \\
}
\vspace{3mm}
%~~~~~~~~~~~~~~~~~~~~~~~~~~~~~~~~~~~~~~~~~~~~~~~~~~~~~~~~~~~~~~
(c) $\lambda+1$ is an eigenvalue of $A+I$, as in Problem 2.\\ \vspace{2mm}
%~~~~~~~~~~~~~~~~~~~~~ ANSWER TO 6.1.9c ~~~~~~~~~~~~~~~~~~~~~~~~
{\addtolength{\leftskip}{5mm}
Add $Ix=x$ to both sides. \\
$Ix+Ax=x+\lambda x \hspace{4mm}\rightarrow\hspace{4mm} (A+I)x=(\lambda +1)x$. \\
}
\vspace{3mm}
%~~~~~~~~~~~~~~~~~~~~~~~~~~~~~~~~~~~~~~~~~~~~~~~~~~~~~~~~~~~~~~
}
\item[6.1.12] Find three eigenvectors for this matrix $P$ (projection matrices have $\lambda = 1$ and 0): \\ 
\begin{center}
\textbf{Projection matrix}
\hspace{20mm}
$
P=
\begin{bmatrix}
.2&.4&0\\
.4&.8&0\\
0&0&1\\
\end{bmatrix}
$. \\
\end{center}
If two eigenvectors share the same $\lambda$, so do all their linear combinations. Find an eigenvector of $P$ with no zero components. \\
\vspace{2mm}
%~~~~~~~~~~~~~~~~~~~~~ ANSWER TO 6.1.12 ~~~~~~~~~~~~~~~~~~~~~~~~
\begin{center}
$
\lambda=1 \hspace{4mm}
\begin{bmatrix}
-.8&.4&0\\
.4&-.2&0\\
0&0&0\\
\end{bmatrix}
\begin{bmatrix}
x_1\\
x_2\\
x_3\\
\end{bmatrix}
=0 \hspace{4mm}
$
eigenvectors$=
\begin{bmatrix}
0\\
0\\
1\\
\end{bmatrix},
\begin{bmatrix}
1\\
2\\
0\\
\end{bmatrix}
$ \\ \vspace{2mm}
$
\lambda =0 \hspace{4mm}
\begin{bmatrix}
.2&.4&0\\
.4&.8&0\\
0&0&1\\
\end{bmatrix}
\begin{bmatrix}
x_1\\
x_2\\
x_3\\
\end{bmatrix}
=0 \hspace{4mm}
$
eigenvector$=
\begin{bmatrix}
-2\\
1\\
0\\
\end{bmatrix}
$ \\
\end{center}
Combine the eigenvectors when $\lambda =1$ to get an eigenvector of $P$ with no zero components: 
$
\begin{bmatrix}
1\\
2\\
1\\
\end{bmatrix}
$ \\
\vspace{3mm}
%~~~~~~~~~~~~~~~~~~~~~~~~~~~~~~~~~~~~~~~~~~~~~~~~~~~~~~~~~~~~~~
\item[6.1.13] From the unit vector \textbf{$u$} = ($\frac{1}{6}, \frac{1}{6}, \frac{3}{6}, \frac{5}{6}$) construct the rank one projection matrix $P = $\textbf{$uu^T$}. This matrix has $P^2 = P$ because \textbf{$u^Tu$} = 1. \\ \vspace{3mm}

\begin{center}
$
P=\frac{1}{6}
\begin{bmatrix}
1\\
1\\
3\\
5\\
\end{bmatrix}
\frac{1}{6}
\begin{bmatrix}
1&1&3&5\\
\end{bmatrix}
=
\frac{1}{36}
\begin{bmatrix}
1&1&3&5\\
1&1&3&5\\
3&3&9&15\\
5&5&15&25\\
\end{bmatrix}
$ \\
\end{center}
\vspace{3mm}

{\addtolength{\leftskip}{10mm}
(a) $Pu=u$ comes from $(uu^T)u=u(\underline{\hspace{10mm}})$. Then $u$ is an eigenvector with $\lambda = 1$.\\ \vspace{2mm}
%~~~~~~~~~~~~~~~~~~~~~ ANSWER TO 6.1.13a ~~~~~~~~~~~~~~~~~~~~~~~~
{\addtolength{\leftskip}{5mm}
$(uu^T)u=u(\underline{\hspace{3mm}u^Tu\hspace{3mm}})$ \\
}
\vspace{3mm}
%~~~~~~~~~~~~~~~~~~~~~~~~~~~~~~~~~~~~~~~~~~~~~~~~~~~~~~~~~~~~~~
(b) If $v$ is perpendicular to $u$ show that $Pv=0$. Then $\lambda = 0$.\\ \vspace{2mm}
%~~~~~~~~~~~~~~~~~~~~~ ANSWER TO 6.1.13b ~~~~~~~~~~~~~~~~~~~~~~~~
{\addtolength{\leftskip}{5mm}
$Pv=(uu^T)v=u(u^Tv)=u*0=0$ \\
}
\vspace{3mm}
%~~~~~~~~~~~~~~~~~~~~~~~~~~~~~~~~~~~~~~~~~~~~~~~~~~~~~~~~~~~~~~
(c) Find three independent eigenvectors of $P$ all with eigenvalue $\lambda=0$.\\ \vspace{2mm}
%~~~~~~~~~~~~~~~~~~~~~ ANSWER TO 6.1.13c ~~~~~~~~~~~~~~~~~~~~~~~~
{\addtolength{\leftskip}{5mm}
$
\begin{bmatrix}
1\\
1\\
1\\
-1\\
\end{bmatrix},
\begin{bmatrix}
-1\\
1\\
0\\
0\\
\end{bmatrix},
\begin{bmatrix}
-5\\
0\\
0\\
1\\
\end{bmatrix}
$ all have $\lambda=0$ and are independent. \\
}
\vspace{3mm}
%~~~~~~~~~~~~~~~~~~~~~~~~~~~~~~~~~~~~~~~~~~~~~~~~~~~~~~~~~~~~~~
}
\item[6.1.15] Every permutation matrix leaves $x = (1,1, ... ,1)$ unchanged. Then $\lambda = 1$. Find two more $\lambda$'s (possibly complex) for these permutations, from $det(P - \lambda I) = 0$: \\ 
\begin{center}
$
P=
\begin{bmatrix}
0&1&0\\
0&0&1\\
1&0&0\\
\end{bmatrix}
$
\hspace{5mm} and \hspace{5mm}
$
\begin{bmatrix}
0&0&1\\
0&1&0\\
1&0&0\\
\end{bmatrix}
$. \\
\end{center}
\vspace{2mm}
%~~~~~~~~~~~~~~~~~~~~~ ANSWER TO 6.1.15 ~~~~~~~~~~~~~~~~~~~~~~~~
\begin{center}
$
P=
\begin{bmatrix}
0&1&0\\
0&0&1\\
1&0&0\\
\end{bmatrix}
\hspace{3mm}
\rightarrow
\hspace{3mm}
\begin{vmatrix}
-\lambda&1&0\\
0&-\lambda&1\\
1&0&-\lambda\\
\end{vmatrix}
=-\lambda^3+1=0
\hspace{5mm}
\lambda=\frac{1 \pm i\sqrt{3}}{2}
$ \\
$
P=
\begin{bmatrix}
0&0&1\\
0&1&0\\
1&0&0\\
\end{bmatrix}
\hspace{3mm}
\rightarrow
\hspace{3mm}
\begin{vmatrix}
-\lambda&0&1\\
0&1-\lambda&0\\
1&0&-\lambda\\
\end{vmatrix}
=\lambda^3-\lambda^2-\lambda+1=0
\hspace{5mm}
\lambda=1,1,-1
$ \\
\end{center}

\vspace{3mm}
%~~~~~~~~~~~~~~~~~~~~~~~~~~~~~~~~~~~~~~~~~~~~~~~~~~~~~~~~~~~~~~
\item[6.1.19] A 3 by 3 matrix $B$ is known to have eigenvalues 0,1,2. This information is enough to find three of these (give the answers where possible): \\ \vspace{2mm}
{\addtolength{\leftskip}{10mm}
(a) the rank of $B$ \\ \vspace{2mm}
%~~~~~~~~~~~~~~~~~~~~~ ANSWER TO 6.1.19a ~~~~~~~~~~~~~~~~~~~~~~~~
{\addtolength{\leftskip}{5mm}
$B$ is a rank two because it has a $\lambda=0$. \\
}
\vspace{3mm}
%~~~~~~~~~~~~~~~~~~~~~~~~~~~~~~~~~~~~~~~~~~~~~~~~~~~~~~~~~~~~~~
(b) the determinate of $B^TB$ \\ \vspace{2mm}
%~~~~~~~~~~~~~~~~~~~~~ ANSWER TO 6.1.19b ~~~~~~~~~~~~~~~~~~~~~~~~
{\addtolength{\leftskip}{5mm}
$|B^TB|=0$ because $B^TB$ is singular. \\
}
\vspace{3mm}
%~~~~~~~~~~~~~~~~~~~~~~~~~~~~~~~~~~~~~~~~~~~~~~~~~~~~~~~~~~~~~~
(c) the eigenvalues of $B^TB$ \\ \vspace{2mm}
%~~~~~~~~~~~~~~~~~~~~~ ANSWER TO 6.1.19c ~~~~~~~~~~~~~~~~~~~~~~~~
{\addtolength{\leftskip}{5mm}
Can't determine. \\
}
\vspace{3mm}
%~~~~~~~~~~~~~~~~~~~~~~~~~~~~~~~~~~~~~~~~~~~~~~~~~~~~~~~~~~~~~~
(d) the eigenvalues of $(B^2+I)^-1$. \\ \vspace{2mm}
%~~~~~~~~~~~~~~~~~~~~~ ANSWER TO 6.1.19d ~~~~~~~~~~~~~~~~~~~~~~~~
{\addtolength{\leftskip}{5mm}
$\lambda$'s of $(B^2+I)^{-1}$ are $\lambda=1,\frac{1}{2},\frac{1}{5}$. \\
}
\vspace{3mm}
%~~~~~~~~~~~~~~~~~~~~~~~~~~~~~~~~~~~~~~~~~~~~~~~~~~~~~~~~~~~~~~
}
\item[6.1.21] \textbf{The eigenvalues of $A$ equal the eigenvalues of $A^T$.} This is because $det(A - \lambda I)$ equals $det(A^T - \lambda I)$. That is true because \underline{\hspace{10mm}}. Show by an example that the eigenvectors of $A$ and $A^T$ are \textit{not} the same.\\ \vspace{2mm}
%~~~~~~~~~~~~~~~~~~~~~ ANSWER TO 6.1.21 ~~~~~~~~~~~~~~~~~~~~~~~~
\begin{center}
It is true because every square matrix has the property $|A|=|A^T|$. \\
\vspace{2mm}
$A=
\begin{bmatrix}
1&0\\
1&2\\
\end{bmatrix}
$
\hspace{4mm} and \hspace{4mm}
$A^T=
\begin{bmatrix}
1&1\\
0&2\\
\end{bmatrix}
$ do not have the same eigen vectors. \\
Eigenvectors of $A=
\begin{bmatrix}
0\\
1\\
\end{bmatrix},
\begin{bmatrix}
-1\\
1\\
\end{bmatrix}
$ while $A^T$ has eigenvectors $
\begin{bmatrix}
1\\
1\\
\end{bmatrix},
\begin{bmatrix}
1\\
0\\
\end{bmatrix}
$. \\
\end{center}

\vspace{3mm}
%~~~~~~~~~~~~~~~~~~~~~~~~~~~~~~~~~~~~~~~~~~~~~~~~~~~~~~~~~~~~~~
\item[6.1.29] (Review) Find the eigenvalues of $A$, $B$, and $C$: \\
\begin{center}
$
A=
\begin{bmatrix}
1&2&3\\
0&4&5\\
0&0&6\\
\end{bmatrix}
$
\hspace{5mm} and \hspace{5mm}
$
B=
\begin{bmatrix}
0&0&1\\
0&2&0\\
3&0&0\\
\end{bmatrix}
$
\hspace{5mm} and \hspace{5mm}
$
C=
\begin{bmatrix}
2&2&2\\
2&2&2\\
2&2&2\\
\end{bmatrix}
$. \\
\end{center}
\vspace{2mm}
%~~~~~~~~~~~~~~~~~~~~~ ANSWER TO 6.1.29 ~~~~~~~~~~~~~~~~~~~~~~~~
\begin{center}
$
A)\hspace{5mm}
|A-\lambda I|=
\begin{vmatrix}
1-\lambda&2&3\\
0&4-\lambda&5\\
0&0&6-\lambda\\
\end{vmatrix}
=(1-\lambda)(4-\lambda)(6-\lambda)=0
\hspace{4mm} \lambda=1,4,6
$ \\ \vspace{2mm}
$
B)\hspace{5mm}
|B-\lambda I|=
\begin{vmatrix}
-\lambda&0&1\\
0&2-\lambda&0\\
3&0&-\lambda\\
\end{vmatrix}
=(\lambda^2-3)(\lambda+2)=0
\hspace{4mm} \lambda=2,\pm\sqrt{3}
$ \\ \vspace{2mm}
C is a rank one matrix, meaning that two of its $\lambda$'s are zero. The last $\lambda$ is the sum of the diagonals.
\hspace{6mm} $\lambda=0,0,6$ \\
\end{center}

\vspace{3mm}
%~~~~~~~~~~~~~~~~~~~~~~~~~~~~~~~~~~~~~~~~~~~~~~~~~~~~~~~~~~~~~~ 
\end{enumerate}
\vspace{5mm}
\textbf{Chapter 6.2}
\begin{enumerate}
\item[6.2.2] If $A$ has $\lambda_1 = 2$ with eigenvector $x_1=
\begin{bmatrix}
1\\
0\\
\end{bmatrix}$ and $\lambda_2=5$ with $x_2=
\begin{bmatrix}
1\\
1\\
\end{bmatrix}$, use $S \Lambda S^{-1}$ to find $A$. No other matrix has the same $\lambda$'s and $x$'s. \\ \vspace{2mm}
%~~~~~~~~~~~~~~~~~~~~~ ANSWER TO 6.2.2 ~~~~~~~~~~~~~~~~~~~~~~~~
\begin{center}
$S \Lambda S^{-1}=
\begin{bmatrix}
1&1\\
0&1\\
\end{bmatrix}
\begin{bmatrix}
2&0\\
0&5\\
\end{bmatrix}
\begin{bmatrix}
1&-1\\
0&1\\
\end{bmatrix}
=
\begin{bmatrix}
2&5\\
0&5\\
\end{bmatrix}
\begin{bmatrix}
1&-1\\
0&1\\
\end{bmatrix}
=
\begin{bmatrix}
2&3\\
0&5\\
\end{bmatrix}
=A$
\end{center}
\vspace{3mm}
%~~~~~~~~~~~~~~~~~~~~~~~~~~~~~~~~~~~~~~~~~~~~~~~~~~~~~~~~~~~~~~ 
\item[6.2.8] Diagonalize the Fibonacci matrix by completing $S^{-1}$: \\
\begin{center}
$
\begin{bmatrix}
1&1\\
1&0\\
\end{bmatrix}
=
\begin{bmatrix}
\lambda_1&\lambda_2\\
1&1\\
\end{bmatrix}
\begin{bmatrix}
\lambda_1&0\\
0&\lambda_2\\
\end{bmatrix}
\begin{bmatrix}
&\\
&\\
\end{bmatrix}
$. \\
\end{center}
Do the multiplication $S \Lambda S^{-1}
\begin{bmatrix}
1\\
0\\
\end{bmatrix}$ to find its second component. This is the \textit{k}th Fibonacci number $F_k=(\lambda_1^k-\lambda_2^k)/(\lambda_1-\lambda_2)$.
\vspace{2mm}
%~~~~~~~~~~~~~~~~~~~~~ ANSWER TO 6.2.8 ~~~~~~~~~~~~~~~~~~~~~~~~
\begin{center}
$
S^{-1}=
\frac{1}{\lambda_1-\lambda_2}
\begin{bmatrix}
1&-\lambda_2\\
-1&\lambda_1\\
\end{bmatrix}
$ \\
$
S \Lambda^kS^{-1}=
\frac{1}{\lambda_1-\lambda_2}
\begin{bmatrix}
\lambda_1&\lambda_2\\
1&1\\
\end{bmatrix}
\begin{bmatrix}
\lambda_1^k&0\\
0&\lambda_2^k\\
\end{bmatrix}
\begin{bmatrix}
1&-\lambda_2\\
-1&\lambda_1\\
\end{bmatrix}
\begin{bmatrix}
1\\
0\\
\end{bmatrix}
\hspace{3mm}
\rightarrow
\hspace{3mm}
\frac{1}{\lambda_1-\lambda_2}
\begin{bmatrix}
\lambda_1^{k+1}&\lambda_2^{k+1}\\
\lambda_1^k&\lambda_2^k\\
\end{bmatrix}
\begin{bmatrix}
1&-\lambda_2\\
-1&\lambda_1\\
\end{bmatrix}
\begin{bmatrix}
1\\
0\\
\end{bmatrix}
$ \\
$
\rightarrow
\hspace{4mm}
\frac{1}{\lambda_1-\lambda_2}
\begin{bmatrix}
\lambda_1^{k+1}-\lambda_2^{k+1}\\
\lambda_1^k-\lambda_2^k\\
\end{bmatrix}
$
\end{center}

\vspace{3mm}
%~~~~~~~~~~~~~~~~~~~~~~~~~~~~~~~~~~~~~~~~~~~~~~~~~~~~~~~~~~~~~~ 
\item[6.2.9] Suppose $G_{k+2}$ is the \textit{average} of the two previous numbers $G_{k+1}$ and $G_k$: \\ 
\begin{center}
$G_{k+2}=\frac{1}{2}G_{k+1}+\frac{1}{2}G_k \hspace{8mm} G_{k+1}=G_{k+1}$
\hspace{10mm} and \hspace{10mm}
$
\begin{bmatrix}
G_{k+2}\\
G_{k+1}\\
\end{bmatrix}
=
\begin{bmatrix}
A\\
\end{bmatrix}
\begin{bmatrix}
G_{k+1}\\
G_k\\
\end{bmatrix}
$. \\
\end{center}
\vspace{2mm}
{\addtolength{\leftskip}{10mm}
(a) Find the eigenvalues and eigenvectors of $A$. \\ \vspace{2mm}
%~~~~~~~~~~~~~~~~~~~~~ ANSWER TO 6.2.9a ~~~~~~~~~~~~~~~~~~~~~~~~
\begin{center}
$
A=
\begin{bmatrix}
\frac{1}{2}&\frac{1}{2}\\
1&0\\
\end{bmatrix}
\hspace{6mm}
|A-\lambda |=
\begin{vmatrix}
\frac{1}{2}-\lambda&\frac{1}{2}\\
1&-\lambda\\
\end{vmatrix}
$ \\
$
\rightarrow
\hspace{3mm}
(\lambda-1)(\lambda+\frac{1}{2})=0
\hspace{3mm}
\lambda = 1, -\frac{1}{2}
$ \\
$
\lambda = 1 \hspace{8mm}
\begin{bmatrix}
-\frac{1}{2}&\frac{1}{2}\\
1&-1\\
\end{bmatrix}
x_1=0
\hspace{5mm}
x_1=
\begin{bmatrix}
1\\
1\\
\end{bmatrix}
$ \\
$
\lambda = -\frac{1}{2} \hspace{8mm}
\begin{bmatrix}
1&\frac{1}{2}\\
1&\frac{1}{2}\\
\end{bmatrix}
x_2=0
\hspace{5mm}
x_2=
\begin{bmatrix}
1\\
-1\\
\end{bmatrix}
$ \\
\end{center}
\vspace{3mm}
%~~~~~~~~~~~~~~~~~~~~~~~~~~~~~~~~~~~~~~~~~~~~~~~~~~~~~~~~~~~~~~
(b) Find the limit as $n \rightarrow \infty$ of the matrices $A^n=S \Lambda S^{-1}$. \\ \vspace{2mm}
%~~~~~~~~~~~~~~~~~~~~~ ANSWER TO 6.2.9b ~~~~~~~~~~~~~~~~~~~~~~~~
\begin{center}
$
A^{\infty}=
\begin{bmatrix}
1&1\\
1&-2\\
\end{bmatrix}
\begin{bmatrix}
1&0\\
0&0\\
\end{bmatrix}
\frac{1}{3}
\begin{bmatrix}
2&1\\
1&-1\\
\end{bmatrix}
=
\frac{1}{3}
\begin{bmatrix}
2&1\\
2&1\\
\end{bmatrix}
$ \\
\end{center}
\vspace{3mm}
%~~~~~~~~~~~~~~~~~~~~~~~~~~~~~~~~~~~~~~~~~~~~~~~~~~~~~~~~~~~~~~
(c) If $G_0=0$ and $G_1=1$ show that the Gibonacci numbers approach $\frac{2}{3}$. \\ \vspace{2mm}
%~~~~~~~~~~~~~~~~~~~~~ ANSWER TO 6.2.9c ~~~~~~~~~~~~~~~~~~~~~~~~
{\addtolength{\leftskip}{5mm}
\begin{center}
$
G^{k+1}=A^k
\begin{bmatrix}
1\\
0\\
\end{bmatrix}
=
\frac{1}{3}
\begin{bmatrix}
2&1\\
2&1\\
\end{bmatrix}
\begin{bmatrix}
1\\
0\\
\end{bmatrix}
=
\begin{bmatrix}
\frac{2}{3}\\
\frac{2}{3}\\
\end{bmatrix}
$ \\
\end{center}
}
\vspace{3mm}
%~~~~~~~~~~~~~~~~~~~~~~~~~~~~~~~~~~~~~~~~~~~~~~~~~~~~~~~~~~~~~~
}
\item[6.2.10] Prove that every third Fibonacci number in 0,1,1,2,3,... is even.\\ \vspace{2mm}
%~~~~~~~~~~~~~~~~~~~~~ ANSWER TO 6.2.10 ~~~~~~~~~~~~~~~~~~~~~~~~
{\addtolength{\leftskip}{5mm}
The fibonacci pattern is odd number + even number then odd number plus odd number, which produces an even number every third term. \\
}
\vspace{3mm}
%~~~~~~~~~~~~~~~~~~~~~~~~~~~~~~~~~~~~~~~~~~~~~~~~~~~~~~~~~~~~~~
\item[6.2.11] True or false: If the eigenvalues of $A$ are 2,2,5 then the matrix is certainly \\ 
\begin{center}
(a) invertible \hspace{10mm} (b) diagonalizable \hspace{10mm} (c) not diagonalizable. \\
\end{center}
\vspace{2mm}
%~~~~~~~~~~~~~~~~~~~~~ ANSWER TO 6.2.11 ~~~~~~~~~~~~~~~~~~~~~~~~
{\addtolength{\leftskip}{5mm}
(a) true, no zero eigenvalue \hspace{4mm} (b) false, eigenvalues are repeated \\ (c) false, repeated eigenvalues may have different eigenvectors \\
}
\vspace{3mm}
%~~~~~~~~~~~~~~~~~~~~~~~~~~~~~~~~~~~~~~~~~~~~~~~~~~~~~~~~~~~~~~
\item[6.2.15] $A^k=S \Lambda S^{-1}$ approaches the zero matrix as $k \rightarrow \infty$ if and only if every $\lambda$ has absolute value less than \underline{\hspace{10mm}}. Which of these matrices has $A^k \rightarrow 0$? \\ 
\begin{center}
$
A_1=
\begin{bmatrix}
.6&.9\\
.4&.1\\
\end{bmatrix}
$
\hspace{8mm} and \hspace{8mm}
$
A_2=
\begin{bmatrix}
.6&.9\\
.1&.6\\
\end{bmatrix}
$. \\
\end{center}
\vspace{2mm}
%~~~~~~~~~~~~~~~~~~~~~ ANSWER TO 6.2.15 ~~~~~~~~~~~~~~~~~~~~~~~~
{\addtolength{\leftskip}{5mm}
$A^k=S \Lambda S^{-1}$ approaches the zero matrix as $k \rightarrow \infty$ if and only if every $\lambda$ has absolute value less than \underline{\hspace{3mm}1\hspace{3mm}}. $A_2$ has $A_2^k \rightarrow 0$ with $\lambda =.3,.9$. \\
}
\vspace{3mm}
%~~~~~~~~~~~~~~~~~~~~~~~~~~~~~~~~~~~~~~~~~~~~~~~~~~~~~~~~~~~~~~
\item[6.2.16] (Recommended) Find $\Lambda$ and $S$ to diagonalize $A_1$ in Problem 15. What is the limit of $\Lambda^k$ as $k \rightarrow \infty$? What is the limit of $S \Lambda^k S^{-1}$? In the columns of this limiting matrix you see the \underline{\hspace{10mm}}. \\ \vspace{2mm}
%~~~~~~~~~~~~~~~~~~~~~ ANSWER TO 6.2.16 ~~~~~~~~~~~~~~~~~~~~~~~~
\begin{center}
$
|A-\lambda I|=\lambda^2-.7\lambda-.3=0 \hspace{5mm} \lambda=1,-.3
$ \\
$
\lambda = 1 \hspace{8mm}
\begin{bmatrix}
-.4&.9\\
.4&-.9\\
\end{bmatrix}
x_1=0
\hspace{5mm}
x_1=
\begin{bmatrix}
1\\
1\\
\end{bmatrix}
$ \\
$
\lambda = -.3 \hspace{8mm}
\begin{bmatrix}
.9&.9\\
.4&.4\\
\end{bmatrix}
x_2=0
\hspace{5mm}
x_2=
\begin{bmatrix}
1\\
-1\\
\end{bmatrix}
$ \\
$
\Lambda =
\begin{bmatrix}
1&0\\
0&-.3\\
\end{bmatrix}
\hspace{5mm}
S=
\begin{bmatrix}
1&1\\
1&-1\\
\end{bmatrix}
$ \\
$
\Lambda^k \hspace{3mm} \rightarrow \hspace{3mm}
\begin{bmatrix}
1&0\\
0&0\\
\end{bmatrix}
$ \\
$
S \Lambda^k S^{-1} \hspace{3mm} \rightarrow \hspace{3mm}
\frac{1}{2}
\begin{bmatrix}
1&1\\
1&1\\
\end{bmatrix}
$ In the columns of this limiting matrix you see the \underline{\hspace{3mm}steady state\hspace{3mm}}. \\
\end{center}

\vspace{3mm}
%~~~~~~~~~~~~~~~~~~~~~~~~~~~~~~~~~~~~~~~~~~~~~~~~~~~~~~~~~~~~~~
\item[6.2.19] Diagonalize $B$ and compute $S \Lambda^k S^{-1}$ to prove this formula for $B^k$: \\
\begin{center}
$
B=
\begin{bmatrix}
5&1\\
0&4\\
\end{bmatrix}
$
\hspace{8mm} has \hspace{8mm}
$
B^k=
\begin{bmatrix}
5^k&5^k-4^k\\
0&4^k\\
\end{bmatrix}
$. \\
\end{center}
\vspace{2mm}
%~~~~~~~~~~~~~~~~~~~~~ ANSWER TO 6.2.19 ~~~~~~~~~~~~~~~~~~~~~~~~
\begin{center}
$
|B-\lambda I|=(5-\lambda)(4-\lambda)=0 \hspace{8mm} \lambda=4,5
$ \\
$
\lambda=4 \hspace{8mm} 
\begin{bmatrix}
1&1\\
0&0\\
\end{bmatrix}
x_1=0
\hspace{5mm}
x_1=
\begin{bmatrix}
1\\
-1\\
\end{bmatrix}
\hspace{15mm}
\lambda=5 \hspace{8mm} 
\begin{bmatrix}
0&1\\
0&-1\\
\end{bmatrix}
x_2=0
\hspace{5mm}
x_2=
\begin{bmatrix}
1\\
0\\
\end{bmatrix}
$ \\
$
B=
\begin{bmatrix}
1&1\\
0&-1\\
\end{bmatrix}
\begin{bmatrix}
5&0\\
0&4\\
\end{bmatrix}
\begin{bmatrix}
1&1\\
0&-1\\
\end{bmatrix}
\hspace{15mm}
S \Lambda^k S^{-1}=
\begin{bmatrix}
1&1\\
0&-1\\
\end{bmatrix}
\begin{bmatrix}
5^k&0\\
0&4^k\\
\end{bmatrix}
\begin{bmatrix}
1&1\\
0&-1\\
\end{bmatrix}
=
\begin{bmatrix}
5^k&5^k-4^k\\
0&4^k\\
\end{bmatrix}
$
\end{center}
\vspace{3mm}
%~~~~~~~~~~~~~~~~~~~~~~~~~~~~~~~~~~~~~~~~~~~~~~~~~~~~~~~~~~~~~~
\item[6.2.36] The \textit{n}th power of rotation through $\theta$ is rotation through $n \theta$: \\ 
\begin{center}
$
A^n=
\begin{bmatrix}
\cos\theta&-\sin\theta\\
\sin\theta&\cos\theta\\
\end{bmatrix}^n
=
\begin{bmatrix}
\cos n\theta&-\sin n\theta\\
\sin n\theta&\cos n\theta\\
\end{bmatrix}
$. \\
\end{center}
Prove that neat formula by diagonalizing $A=S \Lambda S^{-1}$. The eigenvectors (columns of $S$) are (1,$i$) and ($i$,1). You need to know Euler's formula $e^{i\theta}=\cos\theta+i\sin\theta$.
\vspace{2mm}
%~~~~~~~~~~~~~~~~~~~~~ ANSWER TO 6.2.36 ~~~~~~~~~~~~~~~~~~~~~~~~
\begin{center}
$
|A-\lambda I| = (\cos\theta-\lambda)^2+\sin^2\theta=0
\hspace{4mm} \rightarrow \hspace{4mm}
\lambda^2-2\cos\theta\lambda+1=0 \hspace{8mm} \lambda=e^{-i\theta},e^{i\theta}
$ \\
$
A^n=
\begin{bmatrix}
1&1\\
-i&i\\
\end{bmatrix}
\begin{bmatrix}
e^{in\theta}&0\\
0&e^{-in\theta}\\
\end{bmatrix}
\frac{1}{2i}
\begin{bmatrix}
i&-1\\
i&1\\
\end{bmatrix}
$ \\
$
\rightarrow
\hspace{5mm}
\frac{1}{2i}
\begin{bmatrix}
ie^{in\theta}+ie^{-in\theta}&ie^{-in\theta}-ie^{in\theta}\\
e^{in\theta}-e^{-in\theta}&2e^{in\theta}\\
\end{bmatrix}
=
\begin{bmatrix}
\cos n\theta&-\sin n\theta\\
\sin n\theta&\cos n\theta\\
\end{bmatrix}
$ \\
\end{center}

\vspace{3mm}
%~~~~~~~~~~~~~~~~~~~~~~~~~~~~~~~~~~~~~~~~~~~~~~~~~~~~~~~~~~~~~~
\end{enumerate}
\vspace{5mm}
\textbf{Chapter 6.3}
\begin{enumerate}
\item[6.3.1] Find two $\lambda$'s and $x$'s so that $u = e^{\lambda t}x$ solves\\
\begin{center}
$
$$
\frac{du}{dt} =
\begin{bmatrix}
4 && 3\\
0 && 1\\
\end{bmatrix}
u
$$
$\\
\end{center}
What combination $u = c_1e^{\lambda_1t}x_1 + c_2e^{\lambda_2t}x_2$ starts from $u(0) = (5, -2)$?\\
%~~~~~~~~~~~~~~~~~~~~~ ANSWER TO 6.3.1 ~~~~~~~~~~~~~~~~~~~~~~~~
$
$$
A = 
\begin{bmatrix}
4&&3\\
0&&1\\
\end{bmatrix}
\hspace{5mm}
\lambda = 1, 4 
\hspace{5mm} 
A - 4I = 
\begin{bmatrix}
0&&3\\
0&&-3\\
\end{bmatrix}
\hspace{3mm}
N(A - 4I) = 
\begin{bmatrix}
1\\
0\\
\end{bmatrix}
$$
$\\
$
$$
u_1 =
e^{4t}
\begin{bmatrix}
1\\
0\\
\end{bmatrix}
\hspace{25mm}
A - I =
\begin{bmatrix}
3 && 3\\
0 && 0\\
\end{bmatrix}
\hspace{3mm}
N(A - I) =
\begin{bmatrix}
1\\
-1\\
\end{bmatrix}
$$
$\\
$
$$
u_2 =
e^t
\begin{bmatrix}
1\\
-1\\
\end{bmatrix}
$$
$\\
If 
$
$$
u(0) = 
\begin{bmatrix}
5\\
-2\\
\end{bmatrix}
$$
$
the 
$
$$
u(t) = 3e^{4t}
\begin{bmatrix}
1\\
0\\
\end{bmatrix}
+
2e^t
\begin{bmatrix}
1\\
-1\\
\end{bmatrix}
$$
$
\vspace{3mm}
%~~~~~~~~~~~~~~~~~~~~~~~~~~~~~~~~~~~~~~~~~~~~~~~~~~~~~~~~~~~~~~ 
\item[6.3.4] A door is opened between rooms that hold $v(0) = 30$ people and $w(0) = 10$ people. The movement between rooms is proportional to the difference $v - w$:\\
\begin{center}
$\frac{dv}{dt} = w - v$ \hspace{3mm} and \hspace{3mm} $\frac{dw}{dt} = v - w$.\\
\end{center}
Show that the total $v + w$ is constant (40 people). Find the matrix in $\frac{du}{dt} = Au$ and its eigenvalues and eigenvectors. What are $v$ and $w$ at $t = 1$ and $t = \infty$?\\
%~~~~~~~~~~~~~~~~~~~~~ ANSWER TO 6.3.4 ~~~~~~~~~~~~~~~~~~~~~~~~
$\frac{dv}{dt} + \frac{dw}{dt} = w-w+v-v=0$, so $v + w$ is constant at 40.\\
$
$$
A =
\begin{bmatrix}
-1 && 1\\
1 && -1
\end{bmatrix}
\hspace{5mm}
\left| A - \lambda I \right| =
\begin{bmatrix}
-1- \lambda && 1\\
1 && -1 - \lambda\\
\end{bmatrix}
=
\lambda^2 + 2\lambda
\hspace{2mm}
\lambda =  -2, 0 
= 0
$$
$\\
$
$$
A + 2I = 
\begin{bmatrix}
1 && 1\\
1 && 1\\
\end{bmatrix}
\hspace{5mm}
N(A + 2I) =
\begin{bmatrix}
1\\
-1\\
\end{bmatrix}
$$
$\\
$
$$
N(A) = 
\begin{bmatrix}
1\\
1\\
\end{bmatrix}
$$
$\\
$
$$
\lambda_1 = -2
\hspace{3mm}
x_1 =
\begin{bmatrix}
1\\
-1\\
\end{bmatrix}
$$
$\\
$
$$
\lambda_2 = 0
\hspace{2mm}
x_2 = 
\begin{bmatrix}
1\\
1\\
\end{bmatrix}
$$
$\\
$V(1) = 20 + 10e^{-2}$ \hspace{3mm} $V(\infty) = 20$\\
$W(1) = 20 - 10e^{-2}$ \hspace{3mm} $W(\infty) = 20$\\
\vspace{45mm}
%~~~~~~~~~~~~~~~~~~~~~~~~~~~~~~~~~~~~~~~~~~~~~~~~~~~~~~~~~~~~~~ 
\item[6.3.5] Reverse the diffusion of people in Problem 4 to $\frac{du}{dt} = -Au$:\\
\begin{center}
$\frac{dv}{dt} = w - v$ \hspace{3mm} and \hspace{3mm} $\frac{dw}{dt} = v - w$.\\
\end{center}
The total $v + w$ still remains constant. How are the $\lambda$'s changed now that $A$ is changed to $-A$? But show that $v(t)$ grows to infinity from $v(0) = 30$.\\
%~~~~~~~~~~~~~~~~~~~~~ ANSWER TO 6.3.5 ~~~~~~~~~~~~~~~~~~~~~~~~
$
$$
-A = \begin{bmatrix}
1 && -1\\
-1 && 1\\
\end{bmatrix}
\left|A - \lambda I \right| = \lambda^2 -2\lambda \rightarrow \lambda = 0, 2
$$
$\\
$
$$
\lambda_1 = 0
\hspace{3mm}
x_1 =
\begin{bmatrix}
1\\
1\\
\end{bmatrix}
$$
$\\
\begin{center}
$v(t) = 20 + 10e^{2t} = \infty \omega t \rightarrow \infty$\\
\end{center}
$
$$
\lambda_2 = 2
\hspace{3mm} 
x_2 = 
\begin{bmatrix}
1\\
-1\\
\end{bmatrix}
$$
$
\vspace{3mm}
%~~~~~~~~~~~~~~~~~~~~~~~~~~~~~~~~~~~~~~~~~~~~~~~~~~~~~~~~~~~~~~ 
\item[6.3.8] The rabbit population shows fast growth (from 6r) but loss to wolves (from -2w). The wolf population always grows in this model ($-w^2$ would control wolves):\\
\begin{center}
$\frac{dr}{dt} = 6r - 2w$ \hspace{3mm} and \hspace{3mm} $\frac{dw}{dt} = 2r + w$.\\
\end{center}
Find the eigenvalues and eigenvectos. If $r(0) = w(0) = 30$ what are the populations at time $t$? After a long time, what is the ratio of rabbits to wolves?\\
%~~~~~~~~~~~~~~~~~~~~~ ANSWER TO 6.3.8 ~~~~~~~~~~~~~~~~~~~~~~~~
$
$$
A =
\begin{bmatrix}
6 && -2\\
2 && 1\\
\end{bmatrix}
\hspace{5mm}
\left| A - \lambda I \right| = (6 - \lambda)(1 - \lambda) + 4 = 0 \rightarrow \lambda^2 - 7\lambda + 10 \rightarrow (\lambda - 5)(\lambda -2) \hspace{3mm} \lambda = 2, 5
$$
$\\
$
$$
A - 2I = 
\begin{bmatrix}
4 && -2\\
2 && -1\\
\end{bmatrix}
\hspace{3mm}
N(A - 2I) =
\begin{bmatrix}
1\\
2\\
\end{bmatrix}
$$
$\\
$
$$
A - 5I =
\begin{bmatrix}
1 && -2\\
2 && -4\\
\end{bmatrix}
\hspace{3mm}
N(A - 5I) = 
\begin{bmatrix}
2\\
1\\
\end{bmatrix}
$$
$\\
$
$$
\lambda_1 = 2
\hspace{3mm}
x_1 = 
\begin{bmatrix}
1\\
2\\
\end{bmatrix}
\hspace{5mm}
u(t) = 10e^{2t}
\begin{bmatrix}
1\\
2\\
\end{bmatrix}
+ 10e^{5t}
\begin{bmatrix}
2\\
1\\
\end{bmatrix}
$$
$\\
$
$$
\lambda_2 = 5
\hspace{3mm}
x_2 =
\begin{bmatrix}
2\\
1\\
\end{bmatrix}
\hspace{5mm}
r(t) = 10e^{2t} + 20e^{5t}
$$
$\\
\hspace{31mm}
$w(t) = 20e^{2t} + 10e^{5t}$\\
Ratio of wolves to rabbits will be $\frac{1}{2}$ as $t \rightarrow \infty, e^{5t}$ dominates.\\
\vspace{3mm}
%~~~~~~~~~~~~~~~~~~~~~~~~~~~~~~~~~~~~~~~~~~~~~~~~~~~~~~~~~~~~~~ 
\item[6.3.10] Find A to change the scalar equation $y'' = 5y' + 4y$ into a vector equation for $u = (y, y')$:\\
\begin{center}
$
$$
\frac{du}{dt} =
\begin{bmatrix}
y'\\
y''\\
\end{bmatrix}
=
\begin{bmatrix}
&&\\
&&\\
\end{bmatrix}
\begin{bmatrix}
y\\
y'\\
\end{bmatrix}
= Au
$$
$\\
\end{center}
What are the eigenvalues of $A$? Find them also by substituting $y = e^{\lambda t}$ into  $y'' = 5y' + 4y$.\\
%~~~~~~~~~~~~~~~~~~~~~ ANSWER TO 6.3.10 ~~~~~~~~~~~~~~~~~~~~~~~~
$
$$
\begin{bmatrix}
y'\\
y''\\
\end{bmatrix}
=
\begin{bmatrix}
0 && 1\\
4 && 5\\
\end{bmatrix}
\begin{bmatrix}
y\\
y'\\
\end{bmatrix}
= Au
$$
$\\
$
$$
\left| A - \lambda I \right|
=
\begin{vmatrix}
-\lambda && 1\\
5 && 4-\lambda\\
\end{vmatrix}
=
\lambda^2 - 5\lambda - 4 = 0
$$
$\\
$y = e^{\lambda t}$
\hspace{3mm}
$y'' = 5y' + 4y' \rightarrow \frac{1}{25} +- \sqrt{25 + 16} = \lambda$\\
\hspace{45mm}
$\lambda^2e^{\lambda t} = 5 \lambda e^{\lambda t} + 4e^{\lambda t}$\\
\hspace{45mm}
$\lambda^2 - 5 \lambda - 4 = 0$\\
\hspace{45mm}
$\lambda = \frac{1}{2} 5 +- \sqrt{41}$\\
\vspace{13mm}
%~~~~~~~~~~~~~~~~~~~~~~~~~~~~~~~~~~~~~~~~~~~~~~~~~~~~~~~~~~~~~~ 
\item[6.3.21] Write $A = \left[\begin{smallmatrix} 1 && 4\\ 0 && 0 \end{smallmatrix} \right]$ in the form $S\Lambda S^{-1}$. Find $e^{At}$ from $Se^{\Lambda t} S^{-1}$.\\
%~~~~~~~~~~~~~~~~~~~~~ ANSWER TO 6.3.21 ~~~~~~~~~~~~~~~~~~~~~~~~
$
$$
A =
\begin{bmatrix}
1 && 4\\
0 && 0\\
\end{bmatrix}
\hspace{3mm}
A - \lambda I = 
\begin{vmatrix}
1 - \lambda && 4\\
0 && - \lambda\\
\end{vmatrix}
= \lambda^2 - \lambda
\hspace{3mm} \lambda = 0, 1
$$
$\\
$
$$
N(A) =
\begin{bmatrix}
4\\
-1\\
\end{bmatrix}
$$
$\\
$
$$
\lambda_1 = 1
\hspace{3mm}
x_1 =
\begin{bmatrix}
1\\
0\\
\end{bmatrix}
$$
$\\
$
$$
\lambda_2 = 0
\hspace{3mm}
x_2 =
\begin{bmatrix}
4\\
-1\\
\end{bmatrix}
$$
$\\
$
$$
A = 
\begin{bmatrix}
1 && 4\\
0 && -1\\
\end{bmatrix}
\begin{bmatrix}
1 && 0\\
0 && 0\\
\end{bmatrix}
\begin{bmatrix}
1 && 4\\
0 && -1\\
\end{bmatrix}
$$
$\\
$
$$
e^{At} =
\begin{bmatrix}
1 && 4\\
0 && -1\\
\end{bmatrix}
\begin{bmatrix}
e^t && 0\\
0 && 1\\
\end{bmatrix}
\begin{bmatrix}
1 && 4\\
0 && -1\\
\end{bmatrix}
= 
\begin{bmatrix}
e^t && 4\\
0 && -1\\
\end{bmatrix}
\begin{bmatrix}
1 && 4\\
0 && -1\\
\end{bmatrix}
=
\begin{bmatrix}
e^t && 4e^t - 4\\
0 && 1\\
\end{bmatrix}
$$
$

\vspace{3mm}
%~~~~~~~~~~~~~~~~~~~~~~~~~~~~~~~~~~~~~~~~~~~~~~~~~~~~~~~~~~~~~~ 
\end{enumerate}
\vspace{5mm}
\textbf{Chapter 6.4}
\begin{enumerate}
\item[6.4.4] Find an orthogonal matrix $Q$ that diagonalizes $A=
\begin{bmatrix}
-2&6\\
6&7\\
\end{bmatrix}$. What is $\lambda$?
\\ \vspace{2mm}
%~~~~~~~~~~~~~~~~~~~~~ ANSWER TO 6.4.4 ~~~~~~~~~~~~~~~~~~~~~~~~
\begin{center}
$
|A- \lambda I|=
(\lambda-10)(\lambda+5) = 0
\hspace{8mm} \lambda=10,-5
$\\
$
\lambda = 10 \hspace{8mm}
\begin{bmatrix}
-12&6\\
6&-3\\
\end{bmatrix}
x_1=0
\hspace{5mm}
x_1 = 
\begin{bmatrix}
1\\
2\\
\end{bmatrix}
$ \\
$
\lambda = -5 \hspace{8mm}
\begin{bmatrix}
3&6\\
6&12\\
\end{bmatrix}
x_2=0
\hspace{5mm}
x_2 = 
\begin{bmatrix}
2\\
-1\\
\end{bmatrix}
$ \\
$
\Lambda = 
\begin{bmatrix}
10&0\\
0&-5\\
\end{bmatrix}
\hspace{10mm}
Q=
\frac{1}{\sqrt{5}}
\begin{bmatrix}
1&2\\
2&-1\\
\end{bmatrix}
$ \\
\end{center}

\vspace{3mm}
%~~~~~~~~~~~~~~~~~~~~~~~~~~~~~~~~~~~~~~~~~~~~~~~~~~~~~~~~~~~~~~ 
\item[6.4.6] Find \textit{all} orthogonal matrices that diagonalize $A = 
\begin{bmatrix}
9&12\\
12&16\\
\end{bmatrix}$. \\ \vspace{2mm}
%~~~~~~~~~~~~~~~~~~~~~ ANSWER TO 6.4.6 ~~~~~~~~~~~~~~~~~~~~~~~~
\begin{center}
$
|A-\lambda I| = \lambda^2-25\lambda=0 \hspace{8mm} \lambda=0,25
$ \\
$
\lambda = 0 \hspace{8mm}
\begin{bmatrix}
9&12\\
12&16\\
\end{bmatrix}
x_1=0
\hspace{5mm}
x_1 = 
\begin{bmatrix}
\frac{4}{5}\\
-\frac{3}{5}\\
\end{bmatrix}
$ \\
$
\lambda = 25 \hspace{8mm}
\begin{bmatrix}
-16&12\\
12&-9\\
\end{bmatrix}
x_2=0
\hspace{5mm}
x_2 = 
\begin{bmatrix}
\frac{3}{5}\\
\frac{4}{5}\\
\end{bmatrix}
$ \\
$
Q=
\frac{1}{5}
\begin{bmatrix}
4&3\\
-3&4\\
\end{bmatrix}
$ and all other combination of those columns with and without their signs reversed. \\
\end{center}

\vspace{3mm}
%~~~~~~~~~~~~~~~~~~~~~~~~~~~~~~~~~~~~~~~~~~~~~~~~~~~~~~~~~~~~~~ 
\item[6.4.11] Write $A$ and $B$ in the form $\lambda_1x_1x_1^T+\lambda_2x_2x_2^T$ of the spectral theorem $Q \Lambda Q^T$: \\
\begin{center}
$
A=
\begin{bmatrix}
3&1\\
1&3\\
\end{bmatrix}
\hspace{8mm}
B=
\begin{bmatrix}
9&12\\
12&16\\
\end{bmatrix}
$
\hspace{6mm} (keep $||x_1||=||x_2||=1$). \\
\end{center}
\vspace{2mm}
%~~~~~~~~~~~~~~~~~~~~~ ANSWER TO 6.4.11 ~~~~~~~~~~~~~~~~~~~~~~~~
\begin{center}
$
|A-\lambda I|=(\lambda-4)(\lambda-2)=0
\hspace{8mm}
\lambda=4,2
$ \\
$
\lambda = 4 \hspace{8mm}
\begin{bmatrix}
-1&1\\
1&-1\\
\end{bmatrix}
x_1=0
\hspace{5mm}
x_1 = 
\begin{bmatrix}
1\\
1\\
\end{bmatrix}
\hspace{5mm}
\hat{x}_1=
\frac{1}{\sqrt{2}}
\begin{bmatrix}
1\\
1\\
\end{bmatrix}
$ \\
$
\lambda = 2 \hspace{8mm}
\begin{bmatrix}
1&1\\
1&1\\
\end{bmatrix}
x_2=0
\hspace{5mm}
x_2 = 
\begin{bmatrix}
1\\
-1\\
\end{bmatrix}
\hspace{5mm}
\hat{x}_2=
\frac{1}{\sqrt{2}}
\begin{bmatrix}
1\\
-1\\
\end{bmatrix}
$ \\
$
A=4\frac{1}{\sqrt{2}}
\begin{bmatrix}
1\\
1\\
\end{bmatrix}
2\frac{1}{\sqrt{2}}
\begin{bmatrix}
1&1\\
\end{bmatrix}
+2\frac{1}{\sqrt{2}}
\begin{bmatrix}
1\\
-1\\
\end{bmatrix}
2\frac{1}{\sqrt{2}}
\begin{bmatrix}
1&-1\\
\end{bmatrix}
=
2
\begin{bmatrix}
1&1\\
1&1\\
\end{bmatrix}
+
\begin{bmatrix}
1&-1\\
-1&1\\
\end{bmatrix}
$ \\ \vspace{2mm}
$B$'s eigenvalues and vectors were found in problem 6.4.6 \\
$
B=0*\frac{1}{5}
\begin{bmatrix}
4\\
-3\\
\end{bmatrix}
\frac{1}{5}
\begin{bmatrix}
4&-3\\
\end{bmatrix}
+
25*\frac{1}{5}
\begin{bmatrix}
3\\
4\\
\end{bmatrix}
\frac{1}{5}
\begin{bmatrix}
3&4\\
\end{bmatrix}
=
0*1/25
\begin{bmatrix}
16&-12\\
-12&9\\
\end{bmatrix}
+
\begin{bmatrix}
9&12\\
12&16\\
\end{bmatrix}
$ \\
\end{center}

\vspace{3mm}
%~~~~~~~~~~~~~~~~~~~~~~~~~~~~~~~~~~~~~~~~~~~~~~~~~~~~~~~~~~~~~~ 
\item[6.4.21] \textbf{True} (with reason) or \textbf{false} (with example). "Orthonormal" is not assumed. \\ \vspace{2mm}
{\addtolength{\leftskip}{10mm}
(a) A matrix with real eigenvalues and eigenvectors is symmetric.\\ \vspace{2mm}
%~~~~~~~~~~~~~~~~~~~~~ ANSWER TO 6.4.21a ~~~~~~~~~~~~~~~~~~~~~~~~
{\addtolength{\leftskip}{5mm}
False,
$
\begin{bmatrix}
4&3\\
0&1\\
\end{bmatrix}
$
has $\lambda$=1,4 with eigenvectors 
$
\begin{bmatrix}
1\\
0\\
\end{bmatrix},
\begin{bmatrix}
1\\
-1\\
\end{bmatrix}
$. \\
}
\vspace{3mm}
%~~~~~~~~~~~~~~~~~~~~~~~~~~~~~~~~~~~~~~~~~~~~~~~~~~~~~~~~~~~~~~
(b) A matrix with real eigenvalues and orthogonal eigenvectors is symmetric.\\ \vspace{2mm}
%~~~~~~~~~~~~~~~~~~~~~ ANSWER TO 6.4.21b ~~~~~~~~~~~~~~~~~~~~~~~~
{\addtolength{\leftskip}{5mm}
True, $A=Q \Lambda Q^T \rightarrow A^T=(Q \Lambda Q^T)^T={Q^T}^T \Lambda^T Q^T=Q \Lambda Q^T$. \\
}
\vspace{3mm}
%~~~~~~~~~~~~~~~~~~~~~~~~~~~~~~~~~~~~~~~~~~~~~~~~~~~~~~~~~~~~~~
(c) The inverse of a symmetric matrix is symmetric.\\ \vspace{2mm}
%~~~~~~~~~~~~~~~~~~~~~ ANSWER TO 6.4.21c ~~~~~~~~~~~~~~~~~~~~~~~~
{\addtolength{\leftskip}{5mm}
True, $A=Q \Lambda Q^T \rightarrow A^{-1}=(Q \Lambda Q^T)^{-1}={Q^T}^{-1} \Lambda^{-1} Q^{-1}=Q \Lambda^{-1} Q^T$. \\
}
\vspace{3mm}
%~~~~~~~~~~~~~~~~~~~~~~~~~~~~~~~~~~~~~~~~~~~~~~~~~~~~~~~~~~~~~~
(d) The eigenvector matrix $S$ of a symmetric matrix is symmetric.\\ \vspace{2mm}
%~~~~~~~~~~~~~~~~~~~~~ ANSWER TO 6.4.21d ~~~~~~~~~~~~~~~~~~~~~~~~
{\addtolength{\leftskip}{5mm}
False, 
$
A=
\begin{bmatrix}
-1&1\\
1&-1\\
\end{bmatrix}
=
\begin{bmatrix}
1&1\\
-1&1\\
\end{bmatrix}
\begin{bmatrix}
-2&0\\
0&0\\
\end{bmatrix}
\frac{1}{2}
\begin{bmatrix}
1&-1\\
1&1\\
\end{bmatrix}
\hspace{4mm}
S=
\begin{bmatrix}
1&1\\
-1&1\\
\end{bmatrix}
\hspace{2mm}
\neq
\hspace{2mm}
S^{-1}=
\begin{bmatrix}
1&-1\\
1&1\\
\end{bmatrix}
$. \\
}
\vspace{3mm}
%~~~~~~~~~~~~~~~~~~~~~~~~~~~~~~~~~~~~~~~~~~~~~~~~~~~~~~~~~~~~~~
} 
\end{enumerate}
\vspace{5mm}
\textbf{Chapter 6.5}
\begin{enumerate}
\item[6.5.7]  Test to see if $R^R$ is positive definite in each case:\\ 
\begin{center}
$
$$
R =
\begin{bmatrix}
1 && 2\\
0 && 3\\
\end{bmatrix}
$$
$
and 
$
$$
R = 
\begin{bmatrix}
1 && 1\\
1 && 2\\
2 && 1\\
\end{bmatrix}
$$
$
and
$
$$
R = 
\begin{bmatrix}
1 && 1 && 2\\
1 && 2 && 1\\
\end{bmatrix}
$$
$
\end{center}
%~~~~~~~~~~~~~~~~~~~~~ ANSWER TO 6.5.7 ~~~~~~~~~~~~~~~~~~~~~~~~
$
$$
R = 
\begin{bmatrix}
1 && 2\\
0 && 3\\
\end{bmatrix}
\hspace{5mm}
R^TR =
\begin{bmatrix}
1 && 0\\
2 && 3\\
\end{bmatrix}
\begin{bmatrix}
1 && 2\\
0 && 3\\
\end{bmatrix}
=
\begin{bmatrix}
1 && 2\\
2 && 13\\
\end{bmatrix}
$$
$\\
$
$$
x^T
\begin{bmatrix}
1 && 2\\
2 && 13\\
\end{bmatrix}
x = x^T
\begin{bmatrix}
x_1 + 2x_2\\
2x_1 + 13x_2
\end{bmatrix}
=
x_1^2 + 4x_2x_1 + 13x_2^2 > 0
$$
$
positive definite\\
$
$$
R =
\begin{bmatrix}
1 && 1\\
1 && 2\\
2 && 1\\
\end{bmatrix}
\hspace{5mm}
R^TR =
\begin{bmatrix}
1 && 1 && 2\\
1 && 2 && 1\\
\end{bmatrix}
\begin{bmatrix}
1 && 1\\
1 && 2\\
2 && 1\\
\end{bmatrix}
=
\begin{bmatrix}
6 && 5\\
5 && 6\\
\end{bmatrix}
$$
$\\
$
$$
\begin{vmatrix}
6 - \lambda && 5\\
5 && 6- \lambda\\
\end{vmatrix}
= \lambda^2 - 12\lambda + 11 = 0 \rightarrow (\lambda - 1)(\lambda - 11) \rightarrow \lambda = 1, 11
$$
$
All eigenvalues are positive so $R^TR$ is positive definite\\
$
$$
R =
\begin{bmatrix}
1 && 1 && 2\\
1 && 2 && 1\\
\end{bmatrix}
\hspace{3mm}
R^TR =
\begin{bmatrix}
1 && 1\\
1 && 2\\
2 && 1\\
\end{bmatrix}
\begin{bmatrix}
1 && 1 && 2\\
1 && 2 && 1\\
\end{bmatrix}
=
\begin{bmatrix}
2 && 3 && 3\\
3 && 5 && 4\\
3 && 4 && 5\\
\end{bmatrix}
$$
$\\
$
$$
\begin{vmatrix}
2 - \lambda && 3 && 3\\
3 && 5 && 4\\
3 && 4 && 5\\
\end{vmatrix}
=
\lambda^3 - 12\lambda^2 + 11\lambda = 0 \rightarrow \lambda = 0, 1, 11
$$
$
Positive Semi-definite
\vspace{3mm}
%~~~~~~~~~~~~~~~~~~~~~~~~~~~~~~~~~~~~~~~~~~~~~~~~~~~~~~~~~~~~~~ 
\item[6.5.10] Which 3 by 3 symmetric matrices $A$ and $B$ produce these quadratics?\\
$x^TAx = 2(x_1^2 + x_2^2 + x_3^2 - x_1x_2 - x_2x_3)$. Why is $A$ positive definite?\\
$x^TBx = 2(x_1^2 + x_2^2 + x_3^2 - x_1x_2 - x_1x_3-x_2x_3)$. Why is $B$ semidefinite?\\
%~~~~~~~~~~~~~~~~~~~~~ ANSWER TO 6.5.10 ~~~~~~~~~~~~~~~~~~~~~~~~
$
$$
A =
\begin{bmatrix}
2 && -1 && 0\\
-1 && 2 && -1\\
0 && -1 && 2\\
\end{bmatrix}
$$
$
After elimination
$
$$
\begin{bmatrix}
2 && -1 && 0\\
0 && \frac{3}{2} && -1\\
0 && 0 && \frac{4}{3}\\
\end{bmatrix}
$$
$\\
A has all positive pivots so A  is positive definite\\
$
$$
B =
\begin{bmatrix}
2 && -1 && -1\\
-1 && 2 && -1\\
-1 && -1 && 2\\
\end{bmatrix}
\rightarrow
\begin{vmatrix}
2 - \lambda && -1 && -1\\
-1 && 2 - \lambda && -1\\
-1 && -1 && 2 -\lambda
\end{vmatrix}
=
\lambda^3 + 6\lambda^2 + 9\lambda = 0 \hspace{3mm} \lambda = 0, 3, 3
$$
$\\
B is semi definite because all of its $\lambda$'s are either positive or zero.\\
\vspace{3mm}
%~~~~~~~~~~~~~~~~~~~~~~~~~~~~~~~~~~~~~~~~~~~~~~~~~~~~~~~~~~~~~~ 
\item[6.5.17] A diagonal entry $a_{jj}$ of a symmetric matrix cannot be smaller than all the $\lambda$'s. If it were, then $A - a_{jj}I$ would have \underline{\hspace{7mm}} eigenvalues and would be positive definite. But $A - a_{jj}I$ has a \underline{\hspace{7mm}} on the main diagonal.\\
%~~~~~~~~~~~~~~~~~~~~~ ANSWER TO 6.5.17 ~~~~~~~~~~~~~~~~~~~~~~~~
positive - first blank\\
zero - second blank\\

\vspace{3mm}
%~~~~~~~~~~~~~~~~~~~~~~~~~~~~~~~~~~~~~~~~~~~~~~~~~~~~~~~~~~~~~~ 
\item[6.5.18] If $Ax = \lambda x$ then $x^TAx =$ \underline{\hspace{7mm}}. If $x^TAx > 0$, prove that $\lambda > 0$.\\
%~~~~~~~~~~~~~~~~~~~~~ ANSWER TO 6.5.18 ~~~~~~~~~~~~~~~~~~~~~~~~
$x^TAx = x^T\lambda x \rightarrow \lambda = \frac{x^TAx}{x^Tx}$ if $x^TAx > 0$\\

\vspace{3mm}
%~~~~~~~~~~~~~~~~~~~~~~~~~~~~~~~~~~~~~~~~~~~~~~~~~~~~~~~~~~~~~~ 
\item[6.5.19] Reverse Problem 18 to show that if all $\lambda > 0$ then $x^TAx > 0$. We must do this for every nonzero x, not just the eigenvectors. So write x as a combination of the eigenvectors and explain why all "cross terms" are $x_i^Tx_j = 0$. Then $x^TAx$ is\\
\vspace{3mm}
$(c_1x_1 + \cdots + c_nx_n)^T(c_1\lambda_1x_1 + \cdots + c_n\lambda_nx_n) = c_1^2\lambda_1x_1^Tx_1 + \cdots + c_n^2\lambda_nx_n^Tx_n > 0$.
%~~~~~~~~~~~~~~~~~~~~~ ANSWER TO 6.5.19 ~~~~~~~~~~~~~~~~~~~~~~~~
$x^TAx = x^T\lambda x \rightarrow x^TAx = \lambda x^Tx \rightarrow x^TAx = \lambda \lVert x \rVert^2 > 0$ if $\lambda > 0$\\
$x_i^T x_j = 0$ because evectors of symmetric matrices are orthogonal.\\

\vspace{3mm}
%~~~~~~~~~~~~~~~~~~~~~~~~~~~~~~~~~~~~~~~~~~~~~~~~~~~~~~~~~~~~~~ 
\item[6.5.20] Give a quick reason why each of these statements is true:\\
(a) Every positive definite matrix is invertible.\\
%~~~~~~~~~~~~~~~~~~~~~ ANSWER TO 6.5.20 ~~~~~~~~~~~~~~~~~~~~~~~~
Positive definite matrices have non-zero evalues\\

\vspace{3mm}
%~~~~~~~~~~~~~~~~~~~~~~~~~~~~~~~~~~~~~~~~~~~~~~~~~~~~~~~~~~~~~~ 
(b) The only positive definite porjection matrix is $P = I$.\\
%~~~~~~~~~~~~~~~~~~~~~ ANSWER TO 6.5.20 ~~~~~~~~~~~~~~~~~~~~~~~~
All projection matrices are singular except for I\\

\vspace{3mm}
%~~~~~~~~~~~~~~~~~~~~~~~~~~~~~~~~~~~~~~~~~~~~~~~~~~~~~~~~~~~~~~ 
(c) A diagonal matrix with positive diagonal entries is positive definite.\\
%~~~~~~~~~~~~~~~~~~~~~ ANSWER TO 6.5.20 ~~~~~~~~~~~~~~~~~~~~~~~~
It has positive evalues and pivots\\

\vspace{3mm}
%~~~~~~~~~~~~~~~~~~~~~~~~~~~~~~~~~~~~~~~~~~~~~~~~~~~~~~~~~~~~~~ 
(d) A symmetric matrix with a positive determinant might not be positive definite!\\
%~~~~~~~~~~~~~~~~~~~~~ ANSWER TO 6.5.20 ~~~~~~~~~~~~~~~~~~~~~~~~
It could have two negative evalues because the det $= \pi_i \lambda_i$\\

\vspace{3mm}
%~~~~~~~~~~~~~~~~~~~~~~~~~~~~~~~~~~~~~~~~~~~~~~~~~~~~~~~~~~~~~~ 
\item[6.5.28] Without multiplying
$
$$
A =
\begin{bmatrix}
cos\theta && -sin\theta\\
sin\theta && cos\theta \\
\end{bmatrix}
\begin{bmatrix}
2 && 0\\
0 && 5\\
\end{bmatrix}
\begin{bmatrix}
cos\theta && sin\theta\\
-sin\theta && cos\theta \\
\end{bmatrix}
$$
$\\
(a) The determinant of $A$ \hspace{5mm} (b) the eigenvalues of $A$\\
(c) the eigenvectors of $A$ \hspace{6.5mm} (d) a reason why $A$ is symmetric positive definite.\\
%~~~~~~~~~~~~~~~~~~~~~ ANSWER TO 6.5.28 ~~~~~~~~~~~~~~~~~~~~~~~~
(a) $\lvert A \rvert = 2 \cdot 5 = 10$\\
(b) Values along the diagonal of middle matrix, so $\lambda = 2, 5$\\
(c) Columns of the first matrix
$
$$
\begin{bmatrix}
cos\theta\\
sin\theta\\
\end{bmatrix}
\begin{bmatrix}
-sin\theta\\
cos\theta\\
\end{bmatrix}
$$
$\\
(d) A has all positive evalues\\

\vspace{3mm}
%~~~~~~~~~~~~~~~~~~~~~~~~~~~~~~~~~~~~~~~~~~~~~~~~~~~~~~~~~~~~~~ 
\end{enumerate}
\vspace{5mm}
\textbf{Chapter 6.6}
\begin{enumerate}
\item[6.6.17] True of False, with a good reason:\\
(a) A symmetri  matrix can't be similar to a nonsymmetric matrix.\\
%~~~~~~~~~~~~~~~~~~~~~ ANSWER TO 6.6.17 ~~~~~~~~~~~~~~~~~~~~~~~~
False 
$
$$
A =
\begin{bmatrix}
6 && -2\\
2 && 1\\
\end{bmatrix}
\hspace{2mm}
\lambda_1 = 2
\hspace{2mm}
x_1 =
\begin{bmatrix}
1\\
2\\
\end{bmatrix}
\hspace{2mm}
\lambda_2 = 5
\hspace{2mm}
x_2 =
\begin{bmatrix}
2\\
1\\
\end{bmatrix}
$$
$\\
$
$$
A = 
\begin{bmatrix}
1 && 2\\
2 && 1\\
\end{bmatrix}
\begin{bmatrix}
2 && 0\\
0 && 5\\
\end{bmatrix}
\begin{bmatrix}
1 && -2\\
-2 && 1\\
\end{bmatrix}
-y_3
$$
$
A is similar to
$
$$
\begin{bmatrix}
2 && 0\\
0 && 5\\
\end{bmatrix}
$$
$\\

\vspace{3mm}
%~~~~~~~~~~~~~~~~~~~~~~~~~~~~~~~~~~~~~~~~~~~~~~~~~~~~~~~~~~~~~~
(b) An invertible matrix can't be similar to a singular matrix.\\
%~~~~~~~~~~~~~~~~~~~~~ ANSWER TO 6.6.17 ~~~~~~~~~~~~~~~~~~~~~~~~
True, Ranks of similar matrices are the same.\\

\vspace{3mm}
%~~~~~~~~~~~~~~~~~~~~~~~~~~~~~~~~~~~~~~~~~~~~~~~~~~~~~~~~~~~~~~
(c) $A$ can't be similar to $-A$ unless $A = 0$.\\
%~~~~~~~~~~~~~~~~~~~~~ ANSWER TO 6.6.17 ~~~~~~~~~~~~~~~~~~~~~~~~
False
$
$$
\begin{bmatrix}
0 && 1\\
-1 && 0\\
\end{bmatrix}
$$
$
and
$
$$
\begin{bmatrix}
0 && -1\\
1 && 0\\
\end{bmatrix}
$$
$
are similar with $\lambda = -1, 1$\\
\vspace{3mm}
%~~~~~~~~~~~~~~~~~~~~~~~~~~~~~~~~~~~~~~~~~~~~~~~~~~~~~~~~~~~~~~
(d) $A$ can't be similar to $A + I$.\\
%~~~~~~~~~~~~~~~~~~~~~ ANSWER TO 6.6.17 ~~~~~~~~~~~~~~~~~~~~~~~~
True, Adding $I$ increases the evalues by 1\\

\vspace{15mm}
%~~~~~~~~~~~~~~~~~~~~~~~~~~~~~~~~~~~~~~~~~~~~~~~~~~~~~~~~~~~~~~ 
\item[6.6.18] If $B$ is invertible, prove that $AB$ is similar to $BA$. \textit{They have the ame eigenvalues}.\\
%~~~~~~~~~~~~~~~~~~~~~ ANSWER TO 6.6.18 ~~~~~~~~~~~~~~~~~~~~~~~~
$AB = \lambda x \rightarrow MAB = \lambda Mx \rightarrow$\\
\hspace{45mm} $MAB = MBA \rightarrow AB = MBAM^{-1}$\\
$BA = \lambda x \rightarrow MBA = \lambda Mx \rightarrow$\\

\vspace{3mm}
%~~~~~~~~~~~~~~~~~~~~~~~~~~~~~~~~~~~~~~~~~~~~~~~~~~~~~~~~~~~~~~ 
\item[6.6.20] Why are these statements all true?\\
(a) If $A$ is similar to $B$ then $A^2$ is similar to $B^2$.\\
%~~~~~~~~~~~~~~~~~~~~~ ANSWER TO 6.6.20 ~~~~~~~~~~~~~~~~~~~~~~~~
$Ax = \lambda x \rightarrow A^2x = \lambda^2x$ both evalues are squared\\
$Bx = \lambda x \rightarrow B^2x = \lambda^2x$\\

\vspace{3mm}
%~~~~~~~~~~~~~~~~~~~~~~~~~~~~~~~~~~~~~~~~~~~~~~~~~~~~~~~~~~~~~~ 
(b) $A^2$ and $B^2$ can be similar when $A$ and $B$ are not similar (try $\lambda = 0, 0$).\\
%~~~~~~~~~~~~~~~~~~~~~ ANSWER TO 6.6.20 ~~~~~~~~~~~~~~~~~~~~~~~~
$A^2 = (-A)^2$ but $B \neq -A$\\

\vspace{3mm}
%~~~~~~~~~~~~~~~~~~~~~~~~~~~~~~~~~~~~~~~~~~~~~~~~~~~~~~~~~~~~~~ 
(c) $\left[\begin{smallmatrix} 3 && 0\\ 0 && 4 \end{smallmatrix} \right]$ is not similar to $\left[\begin{smallmatrix} 3 && 1\\ 0 && 4 \end{smallmatrix} \right]$.\\
%~~~~~~~~~~~~~~~~~~~~~ ANSWER TO 6.6.20 ~~~~~~~~~~~~~~~~~~~~~~~~
Both have $\lambda = 3, 4$\\

\vspace{3mm}
%~~~~~~~~~~~~~~~~~~~~~~~~~~~~~~~~~~~~~~~~~~~~~~~~~~~~~~~~~~~~~~ 
(d) $\left[\begin{smallmatrix} 3 && 0\\ 0 && 3 \end{smallmatrix} \right]$ is not similar to $\left[\begin{smallmatrix} 3 && 1\\ 0 && 3 \end{smallmatrix} \right]$.\\
%~~~~~~~~~~~~~~~~~~~~~ ANSWER TO 6.6.20 ~~~~~~~~~~~~~~~~~~~~~~~~
Both have $\lambda = 3, 3$ so there are not two evectors to construct a invertible matrix with\\

\vspace{3mm}
%~~~~~~~~~~~~~~~~~~~~~~~~~~~~~~~~~~~~~~~~~~~~~~~~~~~~~~~~~~~~~~ 
(e) If we echange rows 1 and 2 of $A$, and then exchange columns 1 and 2, \textbf{the eigenvalues stay the same}. In this case $M =$ \underline{\hspace{7mm}}.\\
%~~~~~~~~~~~~~~~~~~~~~ ANSWER TO 6.6.20 ~~~~~~~~~~~~~~~~~~~~~~~~
$A = PAP^T$ \hspace{5mm} 
$
$$
M = P =
\begin{bmatrix}
0 && 1 && 0\\
1 && 0 && 0\\
0 && 0 && 1\\
\end{bmatrix}
$$
$\\

\vspace{3mm}
%~~~~~~~~~~~~~~~~~~~~~~~~~~~~~~~~~~~~~~~~~~~~~~~~~~~~~~~~~~~~~~ 
\end{enumerate}
\vspace{5mm}
\textbf{Chapter 6.7}
\begin{enumerate}
\item[6.7.4] Find the eigenvalues and unit egienvectors of $A^TA$ and $AA^T$. Keep each $Av = \sigma \pmb{u}$:\\
\begin{center}
\textbf{Fibonacci matrix} 
$
$$
\begin{bmatrix}
1 && 1\\
1 && 0 \\
\end{bmatrix}
$$
$\\
\end{center}
Construct the singular value decomposition and verify that $A$ equals $U \Sigma V^T$.\\
%~~~~~~~~~~~~~~~~~~~~~ ANSWER TO 6.7.4 ~~~~~~~~~~~~~~~~~~~~~~~~
$
$$
A = 
\begin{bmatrix}
1 && 1\\
1 && 0\\
\end{bmatrix}
AA^T =
\begin{bmatrix}
1 && 1\\
1 && 0\\
\end{bmatrix}
\begin{bmatrix}
1 && 1\\
1 && 0\\
\end{bmatrix}
=
\begin{bmatrix}
2 && 1\\
1 && 1\\
\end{bmatrix}
$$
$\\
$\lvert AA^T - \lambda I \rvert = (2 - \lambda)(1 - \lambda) - 1 = \lambda^2 - 3\lambda + 1\hspace{2mm} \lambda = \frac{3 += \sqrt{5}}{2}$\\
$
$$
AA^T - (\frac{3}{2} + \frac{\sqrt{5}}{2}) I =
\begin{bmatrix}
\frac{1}{2} - \frac{\sqrt{5}}{2} && 1\\
1 && \frac{-1}{2}  - \frac{\sqrt{5}}{2}\\
\end{bmatrix}
\hspace{3mm}
N(AA^T - (\frac{3}{2} + \frac{\sqrt{5}}{2} I) =
\begin{bmatrix}
1\\
\frac{5}{2} - \frac{1}{2}\\
\end{bmatrix}
$$
$\\
$
$$
AA^T - (\frac{3}{2} - \frac{\sqrt{5}}{2}) I =
\begin{bmatrix}
\frac{1}{2} + \frac{\sqrt{5}}{2} && 1\\
1 && \frac{-1}{2}  + \frac{\sqrt{5}}{2}\\
\end{bmatrix}
\hspace{3mm}
N(AA^T - (\frac{3}{2} - \frac{\sqrt{5}}{2} I) =
\begin{bmatrix}
1\\
\frac{1}{2} - \frac{\sqrt{5}}{2}\\
\end{bmatrix}
$$
$\\
$\sigma_1 = \frac{1}{2} + \frac{\sqrt{5}}{2} = \lambda_1(A)\sigma_2 = \frac{\sqrt{5}}{2} - \frac{1}{2} = \lambda_2(A)$\\
\vspace{3mm}
%~~~~~~~~~~~~~~~~~~~~~~~~~~~~~~~~~~~~~~~~~~~~~~~~~~~~~~~~~~~~~~ 

\item[6.7.6] Compute $A^TA$ and $AA^T$ and their eigenvalues and unit eigenvectors for $V$ and $U$.\\
\begin{center}
\textbf{Rectangular matrix}
$
$$
A =
\begin{bmatrix}
1 && 1 && 0\\
0 && 1 && 1\\
\end{bmatrix}
$$
$\\
\end{center}
Check $AV = U \Sigma$ (this will decide $+=$ signs in $U$). $\Sigma$ has the same shape as $A$.\\
%~~~~~~~~~~~~~~~~~~~~~ ANSWER TO 6.7.6 ~~~~~~~~~~~~~~~~~~~~~~~~
$
$$
A = 
\begin{bmatrix}
1 && 1 && 0\\
0 && 1 && 1\\
\end{bmatrix}
\hspace{5mm}
A^TA =
\begin{bmatrix}
1 && 0\\
1 && 1\\
0 && 1\\
\end{bmatrix}
\begin{bmatrix}
1 && 1 && 0\\
0 && 1 && 1\\
\end{bmatrix}
=
\begin{bmatrix}
1 && 1 && 0\\
1 && 2 && 1\\
0 && 1 && 1\\
\end{bmatrix}
$$
$\\
$
$$
\hspace{35mm}
AA^T =
\begin{bmatrix}
1 && 1 && 0\\
0 && 1 && 1\\
\end{bmatrix}
\begin{bmatrix}
1 && 0\\
1 && 1\\
0 && 1\\
\end{bmatrix}
=
\begin{bmatrix}
2 && 1\\
1 && 2\\
\end{bmatrix}
$$
$\\
$\lvert AA^T - \lambda I \rvert = (2 - \lambda)(2 - \lambda) - 1 = \lambda^2 - 4\lambda + 3 = (\lambda - 1)(\lambda - 3)$\\
$
$$
AA^T - I =
\begin{bmatrix}
1 && 1\\
1 && 1\\
\end{bmatrix}
\hspace{3mm}
N(AA^T - I) = 
\begin{bmatrix}
1\\
-1\\
\end{bmatrix}
\frac{1}{\sqrt{2}}
$$
$\\
$
$$
AA^T - 3I = 
\begin{bmatrix}
-1 && 1\\
1 && -1\\
\end{bmatrix}
\hspace{3mm}
N(AA^T-3I) =
\begin{bmatrix}
1\\
1\\
\end{bmatrix}
\frac{1}{\sqrt{2}}
$$
$\\
$
$$
A^TA - I = 
\begin{bmatrix}
0 && 1 && 0\\
1 && 1 && 1\\
0 && 1 && 0\\
\end{bmatrix}
\hspace{3mm}
N(A^TA - I) =
\begin{bmatrix}
1\\ 
0\\
-1\\
\end{bmatrix}
\frac{1}{\sqrt{2}}
$$
$\\
$
$$
A^TA - 3I = 
\begin{bmatrix}
-2 && 1 && 0\\
1 && -1 && 1\\
0 && 1 && -2\\
\end{bmatrix}
\hspace{3mm}
N(A^TA - 3I) =
\begin{bmatrix}
1\\ 
2\\
1\\
\end{bmatrix}
\frac{1}{\sqrt{6}}
$$
$\\
$
$$
N(A^TA) =
\begin{bmatrix}
1\\
-1\\
1\\
\end{bmatrix}
\frac{1}{\sqrt{3}}
$$
$\\
$
$$
A =
\begin{bmatrix}
\frac{1}{\sqrt{2}} && \frac{1}{\sqrt{2}}\\
\frac{-1}{\sqrt{2}} && \frac{1}{\sqrt{2}}
\end{bmatrix}
\begin{bmatrix}
\sqrt{3} && 0 && 0\\
0 && 1 && 0\\
\end{bmatrix}
\begin{bmatrix}
\frac{1}{\sqrt{6}} && \frac{2}{\sqrt{6}} && \frac{1}{\sqrt{6}}\\
\frac{1}{\sqrt{2}} && 0 && \frac{-1}{\sqrt{2}}\\
\frac{1}{\sqrt{3}} && \frac{-1}{\sqrt{3}} && \frac{1}{\sqrt{3}}\\
\end{bmatrix}
$$
$\\
\vspace{3mm}
%~~~~~~~~~~~~~~~~~~~~~~~~~~~~~~~~~~~~~~~~~~~~~~~~~~~~~~~~~~~~~~ 
\item[6.7.7] .\\
\vspace{5mm}
%~~~~~~~~~~~~~~~~~~~~~ ANSWER TO 6.7.7 ~~~~~~~~~~~~~~~~~~~~~~~~
The closest rank one matrix will be the combination of $u_i \sigma_i v_i^T$, where $i$ is determined by the $i$ value of the largest evalue.\\
$
$$
\sqrt{3}
\begin{bmatrix}
\frac{1}{\sqrt{2}}\\
\frac{-1}{\sqrt{2}}\\
\end{bmatrix}
\begin{bmatrix}
\frac{1}{\sqrt{6}} && \frac{2}{\sqrt{6}} && \frac{1}{\sqrt{6}}\\
\end{bmatrix}
=
\frac{1}{2}
\begin{bmatrix}
1 && 2 && 1\\
-1 && -2 && -1\\
\end{bmatrix}
$$
$\\
\vspace{3mm}
%~~~~~~~~~~~~~~~~~~~~~~~~~~~~~~~~~~~~~~~~~~~~~~~~~~~~~~~~~~~~~~ 
\item[6.7.10] Construct the matrix with rank one that has $Av = 12u$ for $v = \frac{1}{2}(1, 1, 1, 1)$ and $u = \frac{1}{3}(2, 2, 1)$. It only sigular value is $\sigma_1 = $ \underline{\hspace{7mm}}.\\
%~~~~~~~~~~~~~~~~~~~~~ ANSWER TO 6.7.10 ~~~~~~~~~~~~~~~~~~~~~~~~
$
$$
A =
12uv^T = 12
\begin{bmatrix}
\frac{2}{3}\\
\frac{2}{3}\\
\frac{1}{3}\\
\end{bmatrix}
\begin{bmatrix}
\frac{1}{2} && \frac{1}{2} && \frac{1}{2} && \frac{1}{2}
\end{bmatrix}
=
2
\begin{bmatrix}
2 && 2 && 2 && 2\\
2 && 2 && 2 && 2\\
1 && 1 && 1 && 1\\
\end{bmatrix}
$$
$

\vspace{3mm}
%~~~~~~~~~~~~~~~~~~~~~~~~~~~~~~~~~~~~~~~~~~~~~~~~~~~~~~~~~~~~~~ 
\item[6.7.11] Suppose $A$ has orthogonal columns $w_1, w_2, \cdots , w_n$ of lengths $\sigma_1, \sigma_2, \cdots, \sigma_n$. What are $U, \Sigma,$ and $V$ in the SVD?\\
%~~~~~~~~~~~~~~~~~~~~~ ANSWER TO 6.7.11 ~~~~~~~~~~~~~~~~~~~~~~~~
$A^TA = I = V$\\
$
$$
\Sigma =
\begin{bmatrix}
\sigma_1 && && && \\
 && \sigma_2 &&  &&\\
 && && \cdots && \\
 && && && \sigma_n\\
\end{bmatrix}
\hspace{3mm}
U =
\begin{bmatrix}
\frac{AV_i}{\sigma_i} && &&\\
&& \cdots &&\\
&& && \frac{AV_n}{\sigma_n}\\
\end{bmatrix}
$$
$\\

\vspace{3mm}
%~~~~~~~~~~~~~~~~~~~~~~~~~~~~~~~~~~~~~~~~~~~~~~~~~~~~~~~~~~~~~~ 
\end{enumerate}
\end{document}  