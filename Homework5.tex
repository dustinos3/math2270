\documentclass[10pt,twoside,reqno]{article}
\usepackage[marginparsep=1em]{geometry}
\geometry{lmargin=1.0in,rmargin=1.0in, bmargin=0.75in,  tmargin=0.75in}
\usepackage[usenames,dvipsnames,svgnames,table]{xcolor}
\usepackage{graphicx}
\usepackage{amssymb}
\usepackage{epstopdf}
\usepackage{tikz}
\usepackage{enumerate}
\usepackage{amsthm}
\usepackage{pgfplots}
\usepackage{tikz-3dplot}
\usetikzlibrary{shapes.geometric}
\usepackage{float}
\usepackage{amsmath}
\usepackage{fancyhdr}
\usepackage{lmodern}
\usepackage{chngcntr}
\usepackage{multicol, comment}

\pagestyle{fancy}
\fancyhf{}
\renewcommand{\sectionmark}[1]{\markright{\thesection.\ #1}}
\lhead{\fancyplain{}{}} 
\fancyhead[RE,RO]{MATH 2270}
\fancyfoot[RE,LO]{Dr. Heavilin}
\fancyfoot[LE,RO]{\thepage}


\begin{document}
\begin{flushright}
\begin{minipage}{.25\textwidth}
Dustin Ginos: \\
A01233669\\
Chandler Kinch: \\
A01662772\\
Jeff Wasden: \\
A01657029\\

\today
\end{minipage}
\end{flushright}

\center{\textbf{\underline{Homework 5}}}\\
\vspace{5mm}
\textbf{Chapter 5.1}
\begin{enumerate}
\item[5.1.3] True or false, with a reason if true or a counterexample if false: \\
{\addtolength{\leftskip}{10mm}
(a) The determinant of $I+A$ is $1$ + det$A$. \\ \vspace{2mm}
%~~~~~~~~~~~~~~~~~~~~~ ANSWER TO 5.1.3a ~~~~~~~~~~~~~~~~~~~~~~~~
\hspace{3mm}
False,
\hspace{5mm}
$
A=
\begin{bmatrix}
7&1&3\\
0&4&7\\
0&0&1\\
\end{bmatrix}
$
\hspace{5mm}
$
A+I=
\begin{bmatrix}
8&1&3\\
0&5&7\\
0&0&2\\
\end{bmatrix}
$
\hspace{5mm}
$1+|A|=29\neq|I+A|=80$ \\ \vspace{3mm}
%~~~~~~~~~~~~~~~~~~~~~~~~~~~~~~~~~~~~~~~~~~~~~~~~~~~~~~~~~~~~~~ 
(b) The determinant of $ABC$ is $|A||B||C|$. \\ \vspace{2mm}
%~~~~~~~~~~~~~~~~~~~~~ ANSWER TO 5.1.3b ~~~~~~~~~~~~~~~~~~~~~~~~
\hspace{3mm}
True, because of property \#9. \\
\vspace{3mm}
%~~~~~~~~~~~~~~~~~~~~~~~~~~~~~~~~~~~~~~~~~~~~~~~~~~~~~~~~~~~~~~ 
(c) The determinant of 4$A$ is $4|A|$. \\ \vspace{2mm}
%~~~~~~~~~~~~~~~~~~~~~ ANSWER TO 5.1.3c ~~~~~~~~~~~~~~~~~~~~~~~~
\hspace{3mm}
False, det
$
\left(4
\begin{bmatrix}
2&0\\
0&2\\
\end{bmatrix} \right)
= 8*8
$
 $\neq$ 
$
4
\begin{vmatrix}
2&0\\
0&2\\
\end{vmatrix}
= 4*4
$. \\
\vspace{3mm}
%~~~~~~~~~~~~~~~~~~~~~~~~~~~~~~~~~~~~~~~~~~~~~~~~~~~~~~~~~~~~~~ 
(d) The determinant of $AB-BA$ is zero. Try an example with 
$
A= 
\begin{bmatrix}
0&0\\
0&1\\
\end{bmatrix}
$. \\
%~~~~~~~~~~~~~~~~~~~~~ ANSWER TO 5.1.3d ~~~~~~~~~~~~~~~~~~~~~~~~
\hspace{3mm}
False, 
\hspace{5mm}
$
A=
\begin{bmatrix}
0&0\\
0&1\\
\end{bmatrix}
$
\hspace{5mm}
$
B=
\begin{bmatrix}
0&1\\
1&0\\
\end{bmatrix}
$ \\
\hspace{10mm}
$AB-BA=
\begin{bmatrix}
0&-1\\
1&0\\
\end{bmatrix}
$
 which is invertible, meaning the determinant is not zero. \\
\vspace{3mm}
%~~~~~~~~~~~~~~~~~~~~~~~~~~~~~~~~~~~~~~~~~~~~~~~~~~~~~~~~~~~~~~ 
}

\item[5.1.24] Elimination reduces $A$ to $U$. Then $A = LU$: \\
\begin{center}
$
A=
\begin{bmatrix}
3&3&4\\
6&8&7\\
-3&5&-9\\
\end{bmatrix}
=
\begin{bmatrix}
1&0&0\\
2&1&0\\
-1&4&1\\
\end{bmatrix}
\begin{bmatrix}
3&3&4\\
0&2&-1\\
0&0&-1\\
\end{bmatrix}
=LU
$. \\
\end{center}
Find the determinants of $L,U,A,U^{-1},L^{-1}, and U^{-1}L^{-1}A$. \\
%~~~~~~~~~~~~~~~~~~~~~ ANSWER TO 5.1.24 ~~~~~~~~~~~~~~~~~~~~~~~~
\vspace{2mm}
{\addtolength{\leftskip}{10mm}
If one reduces $L$ to its reduced row echelon form it becomes $I$. So $|L|=1$ \\ \vspace{2mm}
$|U|=3*2*1=-1$ \\ \vspace{2mm}
$|A|=|U|=-6$ \\ \vspace{2mm}
$|U^{-1}L^{-1}|=\frac{1}{|U|}*\frac{1}{|L|}=-\frac{1}{6}$ \\ \vspace{2mm}
$|U^{-1}L^{-1}A|=|A|=1$ \\ \vspace{3mm}
}
%~~~~~~~~~~~~~~~~~~~~~~~~~~~~~~~~~~~~~~~~~~~~~~~~~~~~~~~~~~~~~~~ 

\item[5.1.27] Compute the determinants of these matrices by row operations: \\
\vspace{1mm}
\begin{center}
$
A=
\begin{bmatrix}
0&a&0\\
0&0&b\\
c&0&0\\
\end{bmatrix}
$
\hspace{8mm}
and
\hspace{8mm}
$
B=
\begin{bmatrix}
0&a&0&0\\
0&0&b&0\\
0&0&0&c\\
d&0&0&0\\
\end{bmatrix}
$
\hspace{8mm}
and
\hspace{8mm}
$
C=
\begin{bmatrix}
a&a&a\\
a&b&b\\
a&b&c\\
\end{bmatrix}
$. \\
\end{center}
%~~~~~~~~~~~~~~~~~~~~~ ANSWER TO 5.1.27 ~~~~~~~~~~~~~~~~~~~~~~~~
$A$: Two row swaps are required to get $A$ in the form 
$
\begin{bmatrix}
a&0&0\\
0&b&0\\
0&0&c\\
\end{bmatrix}
$
 so $|A|=(-1)(-1)abc=abc$. \\ \vspace{1mm}
$B$: Three row swaps are required to get $B$ in the form 
$
\begin{bmatrix}
d&0&0&0\\
0&a&0&0\\
0&0&b&0\\
0&0&0&c\\
\end{bmatrix}
$
\\ \hspace{10mm} so $|B|=(-1)(-1)(-1)abcd=abcd$. \\ \vspace{1mm}
$C$: You get RREF($C$) as  
$
\begin{bmatrix}
a&0&0\\
0&b-a&0\\
0&0&c-b\\
\end{bmatrix}
$
 so $|C|=a(b-a)(c-b)$. \\
\vspace{3mm}
%~~~~~~~~~~~~~~~~~~~~~~~~~~~~~~~~~~~~~~~~~~~~~~~~~~~~~~~~~~~~~~~

\item[5.1.28] True or false (give a reason if true or a 2 by 2 example if false):\\
{\addtolength{\leftskip}{10mm}
(a) If $A$ is not invertible then $AB$ is not invertible. \\ \vspace{2mm}
%~~~~~~~~~~~~~~~~~~~~~ ANSWER TO 5.1.28a ~~~~~~~~~~~~~~~~~~~~~~~~
\hspace{3mm}
True, det($AB$)=det($A) * $det($B$)=$0*$det($B$)=0. \\ \vspace{3mm}
%~~~~~~~~~~~~~~~~~~~~~~~~~~~~~~~~~~~~~~~~~~~~~~~~~~~~~~~~~~~~~~~
(b) The determinant of $A$ is always the products of its pivots. \\ \vspace{2mm}
%~~~~~~~~~~~~~~~~~~~~~ ANSWER TO 5.1.28b ~~~~~~~~~~~~~~~~~~~~~~~~
\hspace{3mm}
False, 
$
\begin{bmatrix}
0&1\\
1&0\\
\end{bmatrix}
$
 requires a row swap therefore the product is the products of its pivots times -1. \\ \vspace{2mm}
%~~~~~~~~~~~~~~~~~~~~~~~~~~~~~~~~~~~~~~~~~~~~~~~~~~~~~~~~~~~~~~~
(c) The determinant of $A-B$ equals det($A$) - det($B$). \\ \vspace{2mm}
%~~~~~~~~~~~~~~~~~~~~~ ANSWER TO 5.1.28c ~~~~~~~~~~~~~~~~~~~~~~~~
\hspace{3mm}
False, 
$
\left|
\begin{bmatrix}
1&0\\
1&1\\
\end{bmatrix}
 - 
\begin{bmatrix}
0&-1\\
0&0\\
\end{bmatrix}
\right|
=
\begin{vmatrix}
1&1\\
1&1\\
\end{vmatrix}
=0 \neq
\begin{vmatrix}
1&0\\
1&1\\
\end{vmatrix}
 - 
\begin{vmatrix}
0&-1\\
0&0\\
\end{vmatrix}
=1-0=1
$ \\ \vspace{2mm}
%~~~~~~~~~~~~~~~~~~~~~~~~~~~~~~~~~~~~~~~~~~~~~~~~~~~~~~~~~~~~~~~
(d) $AB$ and $BA$ have the same determinant. \\ \vspace{2mm}
%~~~~~~~~~~~~~~~~~~~~~ ANSWER TO 5.1.28d ~~~~~~~~~~~~~~~~~~~~~~~~
\hspace{3mm}
True, since multiplication is commutative. $|AB|=|A||B|=|B||A|=|BA|$ \\ \vspace{2mm}
%~~~~~~~~~~~~~~~~~~~~~~~~~~~~~~~~~~~~~~~~~~~~~~~~~~~~~~~~~~~~~~~
}
\end{enumerate}
\vspace{5mm}
\textbf{Chapter 5.2}
\begin{enumerate}
\item[5.2.9] Show that 4 is the largest determinant for a 3 by 3 matrix of 1's and -1's. \\
%~~~~~~~~~~~~~~~~~~~~~ ANSWER TO 5.2.9 ~~~~~~~~~~~~~~~~~~~~~~~~
\vspace{3mm}



\vspace{3mm}
%~~~~~~~~~~~~~~~~~~~~~~~~~~~~~~~~~~~~~~~~~~~~~~~~~~~~~~~~~~~~~~

\item[5.2.23] With 2 by 2 blocks in 4 by 4 matrices, you cannot always use block determinants: \\
\begin{center}
$
\begin{vmatrix}
A&B\\
0&D\\
\end{vmatrix}
=
\begin{vmatrix}
A\\
\end{vmatrix}
\begin{vmatrix}
D\\
\end{vmatrix}
$
\hspace{10mm}
$
\begin{vmatrix}
A&B\\
C&D\\
\end{vmatrix}
\neq
\begin{vmatrix}
A\\
\end{vmatrix}
\begin{vmatrix}
D\\
\end{vmatrix}
-
\begin{vmatrix}
C\\
\end{vmatrix}
\begin{vmatrix}
B\\
\end{vmatrix}
$. \\
\end{center}
{\addtolength{\leftskip}{10mm}
(a) Why is the first statement true? Somehow $B$ doesn't enter. \\ \vspace{2mm}
(b) Show by example that equality fails (as shown) when $C$ enters. \\ \vspace{2mm}
(c) Show by example that the answer det($AD-CB$) is also wrong. \\ \vspace{2mm}
}
%~~~~~~~~~~~~~~~~~~~~~ ANSWER TO 5.2.23 ~~~~~~~~~~~~~~~~~~~~~~~~
\vspace{3mm}



\vspace{3mm}
%~~~~~~~~~~~~~~~~~~~~~~~~~~~~~~~~~~~~~~~~~~~~~~~~~~~~~~~~~~~~~~~ 

\item[5.2.33] The symmetric Pascal matrices have determinant 1. If I subtract 1 from the $n, n$ entry, why does the determinant become zero? (Use rule 3 or cofactors.) \\
\begin{center}
det
$
\begin{bmatrix}
1&1&1&1\\
1&2&3&4\\
1&3&6&10\\
1&4&10&20\\
\end{bmatrix}
=1
$
 (known)
\hspace{12mm}
det
$
\begin{bmatrix}
1&1&1&1\\
1&2&3&4\\
1&3&6&10\\
1&4&10&\textbf{19}\\
\end{bmatrix}
=\textbf{0}
$
 (to explain). \\
\end{center}
%~~~~~~~~~~~~~~~~~~~~~ ANSWER TO 5.2.33 ~~~~~~~~~~~~~~~~~~~~~~~~
\vspace{3mm}



\vspace{3mm}
%~~~~~~~~~~~~~~~~~~~~~~~~~~~~~~~~~~~~~~~~~~~~~~~~~~~~~~~~~~~~~~~ 
\end{enumerate}
\end{document}  