\documentclass[10pt,twoside,reqno]{article}
\usepackage[marginparsep=1em]{geometry}
\geometry{lmargin=1.0in,rmargin=1.0in, bmargin=0.75in,  tmargin=0.75in}
\usepackage[usenames,dvipsnames,svgnames,table]{xcolor}
\usepackage{graphicx}
\usepackage{amssymb}
\usepackage{epstopdf}
\usepackage{tikz}
\usepackage{enumerate}
\usepackage{amsthm}
 \usepackage{pgfplots}
 \usepackage{tikz-3dplot}
 \usetikzlibrary{shapes.geometric}
 \usepackage{float}
\usepackage{amsmath}
\usepackage{fancyhdr}
\usepackage{lmodern}
\usepackage{chngcntr}
\usepackage{multicol, comment}

\pagestyle{fancy}
\fancyhf{}
\renewcommand{\sectionmark}[1]{\markright{\thesection.\ #1}}
\lhead{\fancyplain{}{}} 
\fancyhead[RE,RO]{MATH 2270}
\fancyfoot[RE,LO]{Dr. Heavilin}
\fancyfoot[LE,RO]{\thepage}


\begin{document}
\begin{flushright}
\begin{minipage}{.25\textwidth}
Dustin Ginos: \\
A01233669\\
Chandler Kinch: \\
A01662772\\
Jeff Wasden: \\
A01657029\\

\today
\end{minipage}
\end{flushright}

\center{\textbf{\underline{Homework 3}}}\\
\vspace{5mm}
\textbf{Chapter 3.1}
\begin{enumerate}
\item[3.1.10] Which of the following subsets of ${\mathbb R}^3$ are actually subspaces? \\ \vspace{1mm}
{\addtolength{\leftskip}{5mm}
(a) The plane of vectors $(b_1, b_2 , b_3)$ with $b_1 = b_2$. \\ \vspace{1mm}
\vspace{3mm}
%~~~~~~~~~~~~~~~~~~~~~ ANSWER TO 3.1.10 ~~~~~~~~~~~~~~~~~~~~~~~~

yes\\
\vspace{3mm}

%~~~~~~~~~~~~~~~~~~~~~~~~~~~~~~~~~~~~~~~~~~~~~~~~~~~~~~~~~~~~~~~
(b) The plane of vectors with $b_1 = 1$. \\ \vspace{1mm}
\vspace{3mm}
%~~~~~~~~~~~~~~~~~~~~~ ANSWER TO 3.1.10 ~~~~~~~~~~~~~~~~~~~~~~~~

no\\
\vspace{3mm}

%~~~~~~~~~~~~~~~~~~~~~~~~~~~~~~~~~~~~~~~~~~~~~~~~~~~~~~~~~~~~~~~
(c) The vectors with $b_1b_2b_3 = 0$. \\ \vspace{1mm}
\vspace{3mm}
%~~~~~~~~~~~~~~~~~~~~~ ANSWER TO 3.1.10 ~~~~~~~~~~~~~~~~~~~~~~~~

no\\
\vspace{3mm}

%~~~~~~~~~~~~~~~~~~~~~~~~~~~~~~~~~~~~~~~~~~~~~~~~~~~~~~~~~~~~~~~
(d) All linear combinations of $v = (1,4,0)$ and $w = (2,2,2)$. \\ \vspace{1mm}
\vspace{3mm}
%~~~~~~~~~~~~~~~~~~~~~ ANSWER TO 3.1.10 ~~~~~~~~~~~~~~~~~~~~~~~~

yes\\
\vspace{3mm}

%~~~~~~~~~~~~~~~~~~~~~~~~~~~~~~~~~~~~~~~~~~~~~~~~~~~~~~~~~~~~~~~
(e) All vectors that satisfy $b_1 + b_2 + b_3 = O$. \\ \vspace{1mm}
\vspace{3mm}
%~~~~~~~~~~~~~~~~~~~~~ ANSWER TO 3.1.10 ~~~~~~~~~~~~~~~~~~~~~~~~

yes\\
\vspace{3mm}
%~~~~~~~~~~~~~~~~~~~~~~~~~~~~~~~~~~~~~~~~~~~~~~~~~~~~~~~~~~~~~~~
(f) All vectors with $b_1 < b_2 < b_3$.  \\
\vspace{3mm}

%~~~~~~~~~~~~~~~~~~~~~ ANSWER TO 3.1.10 ~~~~~~~~~~~~~~~~~~~~~~~~

no\\
\vspace{3mm}

%~~~~~~~~~~~~~~~~~~~~~~~~~~~~~~~~~~~~~~~~~~~~~~~~~~~~~~~~~~~~~~~
}

\item[3.1.17] \hspace{5mm}(a) Show that the set of \textit{invertible} matrices in $M$ is not a subspace. \\ \vspace{1mm}
%~~~~~~~~~~~~~~~~~~~~~ ANSWER TO 3.1.17 ~~~~~~~~~~~~~~~~~~~~~~~~
\vspace{3mm}
No because there is no zero matrix\\
\vspace{3mm}
%~~~~~~~~~~~~~~~~~~~~~~~~~~~~~~~~~~~~~~~~~~~~~~~~~~~~~~~~~~~~~~~
\hspace{5mm}(b) Show that the set of \textit{singular} matrices in $M$ is not a subspace.\\
\vspace{3mm}
%~~~~~~~~~~~~~~~~~~~~~ ANSWER TO 3.1.17 ~~~~~~~~~~~~~~~~~~~~~~~~
No\\
ex:
$
$$
\begin{bmatrix}
1&2\\
2&4\\
\end{bmatrix}
+
\begin{bmatrix}
3&6\\
12&24\\
\end{bmatrix}
=
\begin{bmatrix}
4&8\\
14&28\\
\end{bmatrix}
$$
$


\vspace{3mm}
%~~~~~~~~~~~~~~~~~~~~~~~~~~~~~~~~~~~~~~~~~~~~~~~~~~~~~~~~~~~~~~~
\item[3.1.18] True or false (check addition in each case by an example): \\ \vspace{1mm}
{\addtolength{\leftskip}{5mm}
(a) The symmetric matrices in $M$ (with $A^T = A$) form a subspace. \\ \vspace{1mm}
\vspace{3mm}
%~~~~~~~~~~~~~~~~~~~~~ ANSWER TO 3.1.18 ~~~~~~~~~~~~~~~~~~~~~~~~
True:
$
$$
\begin{bmatrix}
1&2&7\\
2&1&2\\
7&2&1\\
\end{bmatrix}
+
\begin{bmatrix}
3&1&4\\
1&3&2\\
4&2&3\\
\end{bmatrix}
=
\begin{bmatrix}
4&3&11\\
3&4&4\\
11&4&4\\
\end{bmatrix}
$$
$

\vspace{3mm}
%~~~~~~~~~~~~~~~~~~~~~~~~~~~~~~~~~~~~~~~~~~~~~~~~~~~~~~~~~~~~~~~
(b) The skew-symmetric matrices in $M$ (with $A^T = -A$) form a subspace. \\ \vspace{1mm}
\vspace{3mm}
%~~~~~~~~~~~~~~~~~~~~~ ANSWER TO 3.1.18 ~~~~~~~~~~~~~~~~~~~~~~~~
True:
$
$$
\begin{bmatrix}
0&1&2\\
-1&0&1\\
-2&-1&0\\
\end{bmatrix}
+
\begin{bmatrix}
0&5&1\\
-5&0&5\\
-1&-5&0\\
\end{bmatrix}
=
\begin{bmatrix}
0&6&3\\
-6&0&6\\
-3&-6&0\\
\end{bmatrix}
$$
$

\vspace{3mm}
%~~~~~~~~~~~~~~~~~~~~~~~~~~~~~~~~~~~~~~~~~~~~~~~~~~~~~~~~~~~~~~~
(c) The unsymmetric matrices in $M$ (with $A^T \neq -A$) form a subspace.\\ \vspace{1mm}
\vspace{3mm}
%~~~~~~~~~~~~~~~~~~~~~ ANSWER TO 3.1.18 ~~~~~~~~~~~~~~~~~~~~~~~~
False:
$
$$
\begin{bmatrix}
1&0&0\\
0&1&0\\
-1&0&1\\
\end{bmatrix}
+
\begin{bmatrix}
1&0&0\\
0&1&0\\
-1&0&1\\
\end{bmatrix}
=
\begin{bmatrix}
1&0&0\\
0&1&0\\
0&0&1\\
\end{bmatrix}
$$
$

\vspace{3mm}
%~~~~~~~~~~~~~~~~~~~~~~~~~~~~~~~~~~~~~~~~~~~~~~~~~~~~~~~~~~~~~~~
}
\item[3.1.27] True or false (with a counterexample if false): \\ \vspace{1mm}
{\addtolength{\leftskip}{5mm}
(a) The vectors $b$ that are not in the column space $C (A)$ form a subspace. \\ \vspace{1mm}
\vspace{3mm}
%~~~~~~~~~~~~~~~~~~~~~ ANSWER TO 3.1.27 ~~~~~~~~~~~~~~~~~~~~~~~~
False: Vectors that aren't in a column space can't form a subspace.\\
\vspace{3mm}
%~~~~~~~~~~~~~~~~~~~~~~~~~~~~~~~~~~~~~~~~~~~~~~~~~~~~~~~~~~~~~~~
(b) If $C (A)$ contains only the zero vector, then $A$ is the zero matrix.\\ \vspace{1mm} 
\vspace{3mm}
%~~~~~~~~~~~~~~~~~~~~~ ANSWER TO 3.1.27 ~~~~~~~~~~~~~~~~~~~~~~~~
True\\
\vspace{3mm}
%~~~~~~~~~~~~~~~~~~~~~~~~~~~~~~~~~~~~~~~~~~~~~~~~~~~~~~~~~~~~~~~
(c) The column space of $2A$ equals the column space of $A$. \\ \vspace{1mm}
\vspace{3mm}
%~~~~~~~~~~~~~~~~~~~~~ ANSWER TO 3.1.27 ~~~~~~~~~~~~~~~~~~~~~~~~
True\\
\vspace{3mm}
%~~~~~~~~~~~~~~~~~~~~~~~~~~~~~~~~~~~~~~~~~~~~~~~~~~~~~~~~~~~~~~~
(d) The column space of $A - I$ equals the column space of $A$ (test this).\\ \vspace{1mm}
\vspace{3mm}
%~~~~~~~~~~~~~~~~~~~~~ ANSWER TO 3.1.27 ~~~~~~~~~~~~~~~~~~~~~~~~
False\\
\vspace{3mm}
%~~~~~~~~~~~~~~~~~~~~~~~~~~~~~~~~~~~~~~~~~~~~~~~~~~~~~~~~~~~~~~~
}

\item[3.1.28] Construct a 3 by 3 matrix whose column space contains (1, 1,0) and (1,0,1) but not (1,1, 1). Construct a 3 by 3 matrix whose column space is only a line. \\
\vspace{3mm}
%~~~~~~~~~~~~~~~~~~~~~ ANSWER TO 3.1.28 ~~~~~~~~~~~~~~~~~~~~~~~~
\begin{center}
$
$$
\begin{bmatrix}
1&1&0\\
0&1&0\\
1&0&0\\
\end{bmatrix}
$$
$
\hspace{10mm}
$
$$
\begin{bmatrix}
2&4&0\\
3&6&0\\
4&8&0\\
\end{bmatrix}
$$
$
\end{center}

%~~~~~~~~~~~~~~~~~~~~~~~~~~~~~~~~~~~~~~~~~~~~~~~~~~~~~~~~~~~~~~~
\end{enumerate}
\vspace{5mm}
\textbf{Chapter 3.2}
\begin{enumerate}
\item[3.2.9] True or false (with reason if true or example to show it is false): \\ \vspace{1mm}
{\addtolength{\leftskip}{5mm}
(a) A square matrix has no free variables. \\ \vspace{1mm}
\vspace{3mm}
%~~~~~~~~~~~~~~~~~~~~~ ANSWER TO 3.2.9 ~~~~~~~~~~~~~~~~~~~~~~~~
False:
$
$$
\begin{bmatrix}
1&1&1\\
2&3&4\\
3&4&5\\
\end{bmatrix}
\rightarrow
\begin{bmatrix}
1&1&1\\
0&1&2\\
0&1&2\\
\end{bmatrix}
\rightarrow
\begin{bmatrix}
1&1&1\\
0&1&2\\
0&0&0\\
\end{bmatrix}
$$
$\\
\vspace{3mm}
%~~~~~~~~~~~~~~~~~~~~~~~~~~~~~~~~~~~~~~~~~~~~~~~~~~~~~~~~~~~~~~~
(b) An invertible matrix has no free variables. \\ \vspace{1mm}
\vspace{3mm}
%~~~~~~~~~~~~~~~~~~~~~ ANSWER TO 3.2.9 ~~~~~~~~~~~~~~~~~~~~~~~~
True because an invertible matrix has three pivots\\
\vspace{3mm}
%~~~~~~~~~~~~~~~~~~~~~~~~~~~~~~~~~~~~~~~~~~~~~~~~~~~~~~~~~~~~~~~
(c) An m by n matrix has no more than n pivot variables. \\ \vspace{1mm}
\vspace{3mm}
%~~~~~~~~~~~~~~~~~~~~~ ANSWER TO 3.2.9 ~~~~~~~~~~~~~~~~~~~~~~~~
True because a matrix can't have more pivot variables than it has pivot columns\\
\vspace{3mm}
%~~~~~~~~~~~~~~~~~~~~~~~~~~~~~~~~~~~~~~~~~~~~~~~~~~~~~~~~~~~~~~~
(d) An m by n matrix has no more than m pivot variables. \\
\vspace{3mm}
%~~~~~~~~~~~~~~~~~~~~~ ANSWER TO 3.2.9 ~~~~~~~~~~~~~~~~~~~~~~~~
True because a matrix can't have more pivot variables than it has variables\\
\vspace{3mm}
%~~~~~~~~~~~~~~~~~~~~~~~~~~~~~~~~~~~~~~~~~~~~~~~~~~~~~~~~~~~~~~~
}

\item[3.2.19] Prove that $U$ and $A = LU$ have the same nullspace when $L$ is invertible: \\
\begin{center}
If $Ux = 0$ then $LUx = 0$. If $LUx = 0$, how do you know $Ux = 0$? \\
\end{center}
\vspace{3mm}
%~~~~~~~~~~~~~~~~~~~~~ ANSWER TO 3.2.19 ~~~~~~~~~~~~~~~~~~~~~~~~


\vspace{3mm}
%~~~~~~~~~~~~~~~~~~~~~~~~~~~~~~~~~~~~~~~~~~~~~~~~~~~~~~~~~~~~~~~
\item[3.2.21] Construct a matrix whose nullspace consists of all combinations of (2,2,1,0) and (3,1,0,1).\\
\vspace{3mm}
%~~~~~~~~~~~~~~~~~~~~~ ANSWER TO 3.2.21 ~~~~~~~~~~~~~~~~~~~~~~~~
\begin{center}
$
$$
\begin{bmatrix}
2&2&1&0\\
3&1&0&1\\
5&3&1&1\\
\end{bmatrix}
$$
$\\
\end{center}
\vspace{3mm}
%~~~~~~~~~~~~~~~~~~~~~~~~~~~~~~~~~~~~~~~~~~~~~~~~~~~~~~~~~~~~~~~
\item[3.2.22] Construct a matrix whose nullspace consists of all multiples of (4, 3, 2,1).\\
\vspace{3mm}
%~~~~~~~~~~~~~~~~~~~~~ ANSWER TO 3.2.22 ~~~~~~~~~~~~~~~~~~~~~~~~



%~~~~~~~~~~~~~~~~~~~~~~~~~~~~~~~~~~~~~~~~~~~~~~~~~~~~~~~~~~~~~~~
\item[3.2.23] Construct a matrix whose column space contains (1, 1, 5) and (0, 3, 1) and whose nullspace contains (1, 1,2). \\
\vspace{3mm}
%~~~~~~~~~~~~~~~~~~~~~ ANSWER TO 3.2.23 ~~~~~~~~~~~~~~~~~~~~~~~~



%~~~~~~~~~~~~~~~~~~~~~~~~~~~~~~~~~~~~~~~~~~~~~~~~~~~~~~~~~~~~~~~
\item[3.2.24] Construct a matrix whose column space contains (1, 1,0) and (0,1,1) and whose nullspace contains (1,0,1) and (0,0,1). \\
\vspace{3mm}
%~~~~~~~~~~~~~~~~~~~~~ ANSWER TO 3.2.24 ~~~~~~~~~~~~~~~~~~~~~~~~



%~~~~~~~~~~~~~~~~~~~~~~~~~~~~~~~~~~~~~~~~~~~~~~~~~~~~~~~~~~~~~~~
\item[3.2.25] Construct a matrix whose column space contains (1, 1, 1) and whose nullspace is the line of multiples of (1, 1, 1, 1).\\
\vspace{3mm}
%~~~~~~~~~~~~~~~~~~~~~ ANSWER TO 3.2.25 ~~~~~~~~~~~~~~~~~~~~~~~~



%~~~~~~~~~~~~~~~~~~~~~~~~~~~~~~~~~~~~~~~~~~~~~~~~~~~~~~~~~~~~~~~
\item[3.2.26] Construct a 2 by 2 matrix whose nullspace equals its column space. This is possible. \\
\vspace{3mm}
%~~~~~~~~~~~~~~~~~~~~~ ANSWER TO 3.2.26 ~~~~~~~~~~~~~~~~~~~~~~~~



%~~~~~~~~~~~~~~~~~~~~~~~~~~~~~~~~~~~~~~~~~~~~~~~~~~~~~~~~~~~~~~~
\item[3.2.27] Why does no 3 by 3 matrix have a nullspace that equals its column space? \\
\vspace{3mm}
%~~~~~~~~~~~~~~~~~~~~~ ANSWER TO 3.2.27 ~~~~~~~~~~~~~~~~~~~~~~~~



%~~~~~~~~~~~~~~~~~~~~~~~~~~~~~~~~~~~~~~~~~~~~~~~~~~~~~~~~~~~~~~~
\item[3.2.28] If $AB = 0$ then the column space of $B$ is contained in the \underline{\hspace{8mm}} of $A$. Give an example of $A$ and $B$. \\
\vspace{3mm}
%~~~~~~~~~~~~~~~~~~~~~ ANSWER TO 3.2.28 ~~~~~~~~~~~~~~~~~~~~~~~~



%~~~~~~~~~~~~~~~~~~~~~~~~~~~~~~~~~~~~~~~~~~~~~~~~~~~~~~~~~~~~~~~
\item[3.2.29] The reduced form $R$ of a 3 by 3 matrix with randomly chosen entries is almost sure to be \underline{\hspace{8mm}}. What $R$ is virtually certain if the random $A$ is 4 by 3? \\
\vspace{3mm}
%~~~~~~~~~~~~~~~~~~~~~ ANSWER TO 3.2.29 ~~~~~~~~~~~~~~~~~~~~~~~~



%~~~~~~~~~~~~~~~~~~~~~~~~~~~~~~~~~~~~~~~~~~~~~~~~~~~~~~~~~~~~~~~
\end{enumerate}
\vspace{5mm}
\textbf{Chapter 3.3}
\begin{enumerate}
\item[3.3.10] Choose vectors $u$ and $v$ so that $A = uv^T$ = column times row: \\
\begin{center}
$
$$
A =
\begin{bmatrix}
3&6&6\\
1&2&2\\
4&8&8\\
\end{bmatrix}
\hspace{5mm}
$$
$
and
\hspace{5mm}
$
$$
A = 
\begin{bmatrix}
2&2&6&4\\
-1&-1&-3&-2\\
\end{bmatrix}
\hspace{5mm}
$$
$
. \\
\end{center}
\textit{$A = uv^T$ is the natural form for every matrix that has rank $r = 1$.}
\vspace{3mm}
%~~~~~~~~~~~~~~~~~~~~~ ANSWER TO 3.3.10 ~~~~~~~~~~~~~~~~~~~~~~~~
\begin{center}
$
$$
A =
\begin{bmatrix}
3&6&6\\
1&2&2\\
4&8&8\\
\end{bmatrix}
=
\begin{bmatrix}
3\\
1\\
4\\
\end{bmatrix}
\begin{bmatrix}
1&2&2\\
\end{bmatrix}
$$
$
and
$
$$
A =
\begin{bmatrix}
2&2&6&4\\
-1&-1&-3&-2\\
\end{bmatrix}
=
\begin{bmatrix}
-2\\
1\\
\end{bmatrix}
\begin{bmatrix}
-1&-1&-3&-2\\
\end{bmatrix}
$$
$
\end{center}

\vspace{3mm}
%~~~~~~~~~~~~~~~~~~~~~~~~~~~~~~~~~~~~~~~~~~~~~~~~~~~~~~~~~~~~~~~
\item[3.3.23] Answer the same questions as in Worked Example \textbf{3.3 C} for \\
\begin{center}
$
$$
A =
\begin{bmatrix}
1&1&2&2\\
2&2&4&4\\
1&c&2&2\\
\end{bmatrix}
\hspace{5mm}
$$
$
and
\hspace{5mm}
$
$$
B = 
\begin{bmatrix}
1-c&2\\
0&2-c\\
\end{bmatrix}
\hspace{5mm}
$$
$
. \\
\end{center}
\vspace{3mm}
%~~~~~~~~~~~~~~~~~~~~~ ANSWER TO 3.3.23 ~~~~~~~~~~~~~~~~~~~~~~~~
\begin{center}
If c = 1,
$
$$
R_A =
\begin{bmatrix}
1&1&2&2\\
0&1&0&0\\
0&0&0&0\\
\end{bmatrix}
$$
$
then
$
$$
N =
\begin{bmatrix}
-1&-2&-2\\
1&0&0\\
0&1&0\\
0&0&1\\
\end{bmatrix}
$$
$\\
Rank 1\\
If c $\neq$ 1,
$
$$
R_A =
\begin{bmatrix}
1&1&2&2\\
0&1&0&0\\
0&0&0&0\\
\end{bmatrix}
$$
$
then
$
$$
N =
\begin{bmatrix}
-1&-2\\
0&0\\
1&0\\
0&1\\
\end{bmatrix}
$$
$\\
\vspace{2mm}
Rank 2\\
\vspace{3mm}
If c = 1,
$
$$
R_B =
\begin{bmatrix}
1&-2\\
0&0\\
\end{bmatrix}
$$
$
then
$
$$
N =
\begin{bmatrix}
1\\
0\\
\end{bmatrix}
$$
$\\
\vspace{2mm}
Rank 1\\
\vspace{3mm}
If c = 2,
$
$$
R_B =
\begin{bmatrix}
1&-2\\
0&0\\
\end{bmatrix}
$$
$
then
$
$$
N =
\begin{bmatrix}
2\\
1\\
\end{bmatrix}
$$
$\\
\vspace{2mm}
If c $\neq 1, 2$ R = I\\
\vspace{3mm}
\end{center}

%~~~~~~~~~~~~~~~~~~~~~~~~~~~~~~~~~~~~~~~~~~~~~~~~~~~~~~~~~~~~~~~
\end{enumerate}
\vspace{5mm}
\textbf{Chapter 3.4}
\begin{enumerate}
\item[3.4.4] Find the complete solution (also called the \textit{general solution}) to\\
\vspace{3mm}
\begin{center}
$
$$
\begin{bmatrix}
1&3&1&2\\
2&6&4&8\\
0&0&2&4\\
\end{bmatrix}
\begin{bmatrix}
x\\
y\\
z\\
t\\
\end{bmatrix}
=
\begin{bmatrix}
1\\
3\\
1\\
\end{bmatrix}
$$
$
\end{center}
\vspace{3mm}
%~~~~~~~~~~~~~~~~~~~~~ ANSWER TO 3.4.4 ~~~~~~~~~~~~~~~~~~~~~~~~
\begin{center}
$
$$
\begin{bmatrix}
1&3&1&2\\
2&6&4&8\\
0&0&2&4\\
\end{bmatrix}
\begin{bmatrix}
x\\
y\\
z\\
t\\
\end{bmatrix}
=
\begin{bmatrix}
1\\
3\\
1\\
\end{bmatrix}
\rightarrow
\begin{bmatrix}
1&3&0&0\\
2&6&2&21\\
0&0&0&0\\
\end{bmatrix}
\begin{bmatrix}
x\\
y\\
z\\
t\\
\end{bmatrix}
=
\begin{bmatrix}
\frac{1}{2}\\
\frac{1}{2}\\
0\\
\end{bmatrix}
$$
$\\
\vspace{2mm}
$y$ and $t$ free variables $y = t = 0$ then $x = \frac{1}{2} = z$\\
\vspace{3mm}
$
$$
X_p =
\begin{bmatrix}
\frac{1}{2}\\
0\\
\frac{1}{2}\\
0\\
\end{bmatrix}
\hspace{4mm}
X_n = c_1
\begin{bmatrix}
3\\
-1\\
0\\
0\\
\end{bmatrix}
+ c_2
\begin{bmatrix}
0\\
0\\
-2\\
1\\
\end{bmatrix}
$$
$\\
\vspace{3mm}
$
$$
X_{complete} =
\begin{bmatrix}
\frac{1}{2}\\
0\\
\frac{1}{2}\\
0\\
\end{bmatrix}
+
c_1
\begin{bmatrix}
3\\
-1\\
0\\
0\\
\end{bmatrix}
+ c_2
\begin{bmatrix}
0\\
0\\
-2\\
1\\
\end{bmatrix}
$$
$
\end{center}
\vspace{3mm}

%~~~~~~~~~~~~~~~~~~~~~~~~~~~~~~~~~~~~~~~~~~~~~~~~~~~~~~~~~~~~~~~
\item[3.4.19] Find the rank of $A$ and also of $A^TA$ and also of $AA^T$:
\begin{center}
$
$$
A =
\begin{bmatrix}
1&1&5\\
1&0&1\\
\end{bmatrix}
$$
$
and
$
$$
A =
\begin{bmatrix}
2&0\\
1&1\\
1&2\\
\end{bmatrix}
$$
$
\end{center}
\vspace{3mm} 
%~~~~~~~~~~~~~~~~~~~~~ ANSWER TO 3.4.19 ~~~~~~~~~~~~~~~~~~~~~~~~
$
$$
A^T =
\begin{bmatrix}
1&1\\
1&0\\
5&1\\
\end{bmatrix}
$$
$\\
\vspace{2mm}
$
$$
R_A =
\begin{bmatrix}
1&0&1\\
0&1&4\\
\end{bmatrix}
$$
$
rank = 2\\
\vspace{2mm}
$
$$
A^TA =
\begin{bmatrix}
1&1\\
1&0\\
5&1\\
\end{bmatrix}
\begin{bmatrix}
1&1&5\\
1&0&1\\
\end{bmatrix}
=
\begin{bmatrix}
2&1&6\\
1&1&5\\
6&5&26\\
\end{bmatrix}
\rightarrow
\begin{bmatrix}
1&1&5\\
2&1&6\\
6&5&26\\
\end{bmatrix}
\rightarrow
\begin{bmatrix}
1&1&5\\
0&-1&-4\\
0&-1&-4\\
\end{bmatrix}
\rightarrow
\begin{bmatrix}
1&1&5\\
0&1&4\\
0&0&0\\
\end{bmatrix}
= R_{A^TA}
$$
$ rank = 2\\
\vspace{2mm}
$
$$
AA^T =
\begin{bmatrix}
1&1&5\\
1&0&1\\
\end{bmatrix}
\begin{bmatrix}
1&1\\
1&0\\
5&1\\
\end{bmatrix}
=
\begin{bmatrix}
27&6\\
6&2\\
\end{bmatrix}
\rightarrow
\begin{bmatrix}
1&\frac{2}{9}\\
6&2\\
\end{bmatrix}
\rightarrow
\begin{bmatrix}
1&\frac{2}{9}\\
0&\frac{2}{3}\\
\end{bmatrix}
= R_{AA^T}
$$
$ rank = 2\\
%~~~~~~~~~~~~~~~~~~~~~~~~~~~~~~~~~~~~~~~~~~~~~~~~~~~~~~~~~~~~~~~
\item[3.4.21] Find the complete solution in the form $x_p + x_n$ to these full rank systems:\\
\vspace{3mm}
(a) $x + y + z = 4$ \hspace{5mm} (b) $x + y + z = 4$\\
\hspace{38mm} $x - y + z = 4$\\
\vspace{3mm}
%~~~~~~~~~~~~~~~~~~~~~ ANSWER TO 3.4.21 ~~~~~~~~~~~~~~~~~~~~~~~~



%~~~~~~~~~~~~~~~~~~~~~~~~~~~~~~~~~~~~~~~~~~~~~~~~~~~~~~~~~~~~~~~
\item[3.4.24] Give examples of matrices $A$ for which the number of solutions to $Ax = b$ is\\
\vspace{3mm}
(a) $0$ or $1$, depending on $\pmb{b}$\\
%~~~~~~~~~~~~~~~~~~~~~ ANSWER TO 3.4.24 ~~~~~~~~~~~~~~~~~~~~~~~~



%~~~~~~~~~~~~~~~~~~~~~~~~~~~~~~~~~~~~~~~~~~~~~~~~~~~~~~~~~~~~~~~
(b) $\infty$, regardless of $\pmb{b}$\\
%~~~~~~~~~~~~~~~~~~~~~ ANSWER TO 3.4.24 ~~~~~~~~~~~~~~~~~~~~~~~~



%~~~~~~~~~~~~~~~~~~~~~~~~~~~~~~~~~~~~~~~~~~~~~~~~~~~~~~~~~~~~~~~
(c) $0$ or $\infty$, depending on $\pmb{b}$\\
%~~~~~~~~~~~~~~~~~~~~~ ANSWER TO 3.4.24 ~~~~~~~~~~~~~~~~~~~~~~~~



%~~~~~~~~~~~~~~~~~~~~~~~~~~~~~~~~~~~~~~~~~~~~~~~~~~~~~~~~~~~~~~~
(d) $1$, regardless of $\pmb{b}$.\\
%~~~~~~~~~~~~~~~~~~~~~ ANSWER TO 3.4.24 ~~~~~~~~~~~~~~~~~~~~~~~~



%~~~~~~~~~~~~~~~~~~~~~~~~~~~~~~~~~~~~~~~~~~~~~~~~~~~~~~~~~~~~~~~
\vspace{3mm}


\item[3.4.33] The complete solution to $Ax = \left[\begin{smallmatrix} 1\\ 3 \end{smallmatrix} \right]$ is $x = \left[\begin{smallmatrix} 1\\ 0 \end{smallmatrix} \right] + c \left[\begin{smallmatrix} 0\\ 1 \end{smallmatrix} \right]$. Find $A$.\\
 \vspace{3mm}
%~~~~~~~~~~~~~~~~~~~~~ ANSWER TO 3.4.33 ~~~~~~~~~~~~~~~~~~~~~~~~



%~~~~~~~~~~~~~~~~~~~~~~~~~~~~~~~~~~~~~~~~~~~~~~~~~~~~~~~~~~~~~~~
\end{enumerate}
\vspace{5mm}
\textbf{Chapter 3.5}
\begin{enumerate}
\item[3.5.16]  Find a basis for each of these subspaces of \textbf{R}$^4$\\
(a) All vectors whose components are equal.\\
\vspace{3mm}
%~~~~~~~~~~~~~~~~~~~~~ ANSWER TO 3.5.16 ~~~~~~~~~~~~~~~~~~~~~~~~



%~~~~~~~~~~~~~~~~~~~~~~~~~~~~~~~~~~~~~~~~~~~~~~~~~~~~~~~~~~~~~~~
(b) All vectors whose components add to zero.\\
\vspace{3mm}
%~~~~~~~~~~~~~~~~~~~~~ ANSWER TO 3.5.16 ~~~~~~~~~~~~~~~~~~~~~~~~



%~~~~~~~~~~~~~~~~~~~~~~~~~~~~~~~~~~~~~~~~~~~~~~~~~~~~~~~~~~~~~~~
(c) All vectors that are perpendicular to $(1, 1, 0, 0)$ and $(1, 0, 1, 1)$.\\
\vspace{3mm}
%~~~~~~~~~~~~~~~~~~~~~ ANSWER TO 3.5.16 ~~~~~~~~~~~~~~~~~~~~~~~~



%~~~~~~~~~~~~~~~~~~~~~~~~~~~~~~~~~~~~~~~~~~~~~~~~~~~~~~~~~~~~~~~
(d) The column space and the nullspace of \textit{\textbf{I}} $(4$ by $4)$.\\
\vspace{3mm}
%~~~~~~~~~~~~~~~~~~~~~ ANSWER TO 3.5.16 ~~~~~~~~~~~~~~~~~~~~~~~~



%~~~~~~~~~~~~~~~~~~~~~~~~~~~~~~~~~~~~~~~~~~~~~~~~~~~~~~~~~~~~~~~
\item[3.5.24] True of false (give a good reason):\\
(a) If the columns of a matrix are dependent, so are the rows.\\
\vspace{3mm} 
%~~~~~~~~~~~~~~~~~~~~~ ANSWER TO 3.5.24 ~~~~~~~~~~~~~~~~~~~~~~~~



%~~~~~~~~~~~~~~~~~~~~~~~~~~~~~~~~~~~~~~~~~~~~~~~~~~~~~~~~~~~~~~~
(b) The column space of a $2$ by $2$ matrix is the same as its row space.\\
\vspace{3mm} 
%~~~~~~~~~~~~~~~~~~~~~ ANSWER TO 3.5.24 ~~~~~~~~~~~~~~~~~~~~~~~~



%~~~~~~~~~~~~~~~~~~~~~~~~~~~~~~~~~~~~~~~~~~~~~~~~~~~~~~~~~~~~~~~
(c) The column space of a $2$ by $2$ matrix has the same dimension as its row space.\\
\vspace{3mm} 
%~~~~~~~~~~~~~~~~~~~~~ ANSWER TO 3.5.24 ~~~~~~~~~~~~~~~~~~~~~~~~



%~~~~~~~~~~~~~~~~~~~~~~~~~~~~~~~~~~~~~~~~~~~~~~~~~~~~~~~~~~~~~~~
(d) The columns of a matrix are a basis for the column space.\\
\vspace{3mm} 
%~~~~~~~~~~~~~~~~~~~~~ ANSWER TO 3.5.24 ~~~~~~~~~~~~~~~~~~~~~~~~



%~~~~~~~~~~~~~~~~~~~~~~~~~~~~~~~~~~~~~~~~~~~~~~~~~~~~~~~~~~~~~~~
\item[3.5.26]  Find a basis (and the dimension) for each of these subspaces of $3$ by $3$ matrices:\\
(a) All diagonal matrices.\\
\vspace{3mm}
%~~~~~~~~~~~~~~~~~~~~~ ANSWER TO 3.5.26 ~~~~~~~~~~~~~~~~~~~~~~~~



%~~~~~~~~~~~~~~~~~~~~~~~~~~~~~~~~~~~~~~~~~~~~~~~~~~~~~~~~~~~~~~~
(b) All symmetric matrices $(A^T = A)$.\\
\vspace{3mm}
%~~~~~~~~~~~~~~~~~~~~~ ANSWER TO 3.5.26 ~~~~~~~~~~~~~~~~~~~~~~~~



%~~~~~~~~~~~~~~~~~~~~~~~~~~~~~~~~~~~~~~~~~~~~~~~~~~~~~~~~~~~~~~~
(c) All skew-symmetric matrices $(A^T = -A)$.\\
\vspace{3mm}
%~~~~~~~~~~~~~~~~~~~~~ ANSWER TO 3.5.26 ~~~~~~~~~~~~~~~~~~~~~~~~



%~~~~~~~~~~~~~~~~~~~~~~~~~~~~~~~~~~~~~~~~~~~~~~~~~~~~~~~~~~~~~~~
\end{enumerate}
\vspace{5mm}
\textbf{Chapter 3.6}
\begin{enumerate}
\item[3.6.2] Find bases and dimensions for the four subspaces associated with $A$ and $B$:\\
\begin{center}
$
$$
A =
\begin{bmatrix}
1&2&4\\
2&4&8\\
\end{bmatrix}
$$
$
and
$
$$
B =
\begin{bmatrix}
1&2&4\\
2&5&8\\
\end{bmatrix}
$$
$
\end{center}
\vspace{3mm}
%~~~~~~~~~~~~~~~~~~~~~ ANSWER TO 3.6.2 ~~~~~~~~~~~~~~~~~~~~~~~~



%~~~~~~~~~~~~~~~~~~~~~~~~~~~~~~~~~~~~~~~~~~~~~~~~~~~~~~~~~~~~~~~
\item[3.6.3] Find a basis for each of the four subspaces associated with A:\\
\begin{center}
$
$$
A =
\begin{bmatrix}
0&1&2&3&4\\
0&1&2&4&6\\
0&0&0&1&2\\
\end{bmatrix}
=
\begin{bmatrix}
1&0&0\\
1&1&0\\
0&1&1\\
\end{bmatrix}
\begin{bmatrix}
0&1&2&3&4\\
0&0&0&1&2\\
0&0&0&0&0\\
\end{bmatrix}
$$
$
\end{center}
\vspace{3mm}
%~~~~~~~~~~~~~~~~~~~~~ ANSWER TO 3.6.3 ~~~~~~~~~~~~~~~~~~~~~~~~



%~~~~~~~~~~~~~~~~~~~~~~~~~~~~~~~~~~~~~~~~~~~~~~~~~~~~~~~~~~~~~~~
\item[3.6.13] True of false (with a reason or a counterexample):\\
(a) If $m = n$ then the row space of $A$ equals the column space.\\
%~~~~~~~~~~~~~~~~~~~~~ ANSWER TO 3.6.13 ~~~~~~~~~~~~~~~~~~~~~~~~



%~~~~~~~~~~~~~~~~~~~~~~~~~~~~~~~~~~~~~~~~~~~~~~~~~~~~~~~~~~~~~~~
(b) The matrices $A$ and $-A$ share the same four subspaces.\\
%~~~~~~~~~~~~~~~~~~~~~ ANSWER TO 3.6.13 ~~~~~~~~~~~~~~~~~~~~~~~~



%~~~~~~~~~~~~~~~~~~~~~~~~~~~~~~~~~~~~~~~~~~~~~~~~~~~~~~~~~~~~~~~
(c) If $A$ and $B$ share the same four subspaces then $A$ is a multiple of $B$.\\
\vspace{3mm}
%~~~~~~~~~~~~~~~~~~~~~ ANSWER TO 3.6.13 ~~~~~~~~~~~~~~~~~~~~~~~~



%~~~~~~~~~~~~~~~~~~~~~~~~~~~~~~~~~~~~~~~~~~~~~~~~~~~~~~~~~~~~~~~
\item[3.6.21] Suppose $A$ is the sum of two matrices of rank one: $A = \pmb{u}\pmb{v}^T + \pmb{w}\pmb{z}^T$.\\
(a) Which vectors span the column space of A?\\
%~~~~~~~~~~~~~~~~~~~~~ ANSWER TO 3.6.21 ~~~~~~~~~~~~~~~~~~~~~~~~



%~~~~~~~~~~~~~~~~~~~~~~~~~~~~~~~~~~~~~~~~~~~~~~~~~~~~~~~~~~~~~~~
(b) Which vectors span the row space of A?\\
%~~~~~~~~~~~~~~~~~~~~~ ANSWER TO 3.6.21 ~~~~~~~~~~~~~~~~~~~~~~~~



%~~~~~~~~~~~~~~~~~~~~~~~~~~~~~~~~~~~~~~~~~~~~~~~~~~~~~~~~~~~~~~~
(c) The rank is less than 2 if \underline{\hspace{7mm}} or if \underline{\hspace{7mm}}.\\
%~~~~~~~~~~~~~~~~~~~~~ ANSWER TO 3.6.21 ~~~~~~~~~~~~~~~~~~~~~~~~



%~~~~~~~~~~~~~~~~~~~~~~~~~~~~~~~~~~~~~~~~~~~~~~~~~~~~~~~~~~~~~~~
(d) Compute A and its rank if $\pmb{u} = \pmb{z} = (1, 0, 0)$ and $\pmb{v} = \pmb{w} = (0, 0, 1)$.\\
\vspace{3mm} 
%~~~~~~~~~~~~~~~~~~~~~ ANSWER TO 3.6.21 ~~~~~~~~~~~~~~~~~~~~~~~~



%~~~~~~~~~~~~~~~~~~~~~~~~~~~~~~~~~~~~~~~~~~~~~~~~~~~~~~~~~~~~~~~
\item[3.6.22] Construct $A = uv^T + wz^T$ whose column space has basis $(1, 2, 4), (2, 2, 1)$ and whose row space has basis $(1, 0), (1, 1)$. Write A as $(3$ by $2)$ times $(2$ by $2)$.\\
\vspace{3mm} 
%~~~~~~~~~~~~~~~~~~~~~ ANSWER TO 3.6.22 ~~~~~~~~~~~~~~~~~~~~~~~~



%~~~~~~~~~~~~~~~~~~~~~~~~~~~~~~~~~~~~~~~~~~~~~~~~~~~~~~~~~~~~~~~
\item[3.6.23] Without multiplying matrices, find bases for the row and column spaces of A:\\
\begin{center}
$
$$
A =
\begin{bmatrix}
1&2\\
4&5\\
2&7\\
\end{bmatrix}
\begin{bmatrix}
3&0&3\\
1&1&2\\
\end{bmatrix}
$$
$\\
\end{center}
How do you know from these shapes that A cannot be invertible?\\
\vspace{3mm}
%~~~~~~~~~~~~~~~~~~~~~ ANSWER TO 3.6.23 ~~~~~~~~~~~~~~~~~~~~~~~~



%~~~~~~~~~~~~~~~~~~~~~~~~~~~~~~~~~~~~~~~~~~~~~~~~~~~~~~~~~~~~~~~
\item[3.6.25] True or false (with a reason or a counterexample):\\
(a) $A$ and $A^T$ have the same number of pivots.\\
\vspace{3mm}
%~~~~~~~~~~~~~~~~~~~~~ ANSWER TO 3.6.25 ~~~~~~~~~~~~~~~~~~~~~~~~



%~~~~~~~~~~~~~~~~~~~~~~~~~~~~~~~~~~~~~~~~~~~~~~~~~~~~~~~~~~~~~~~
(b) $A$ and $A^T$ have the same lft nullspace.\\
\vspace{3mm}
%~~~~~~~~~~~~~~~~~~~~~ ANSWER TO 3.6.25 ~~~~~~~~~~~~~~~~~~~~~~~~



%~~~~~~~~~~~~~~~~~~~~~~~~~~~~~~~~~~~~~~~~~~~~~~~~~~~~~~~~~~~~~~~
(c) If the row space equals the column space then $A^T = A$.\\
\vspace{3mm}
%~~~~~~~~~~~~~~~~~~~~~ ANSWER TO 3.6.25 ~~~~~~~~~~~~~~~~~~~~~~~~



%~~~~~~~~~~~~~~~~~~~~~~~~~~~~~~~~~~~~~~~~~~~~~~~~~~~~~~~~~~~~~~~
(d) If $A^T = -A$ then the row space of $A$ equals the column space.\\
\vspace{3mm}
%~~~~~~~~~~~~~~~~~~~~~ ANSWER TO 3.6.25 ~~~~~~~~~~~~~~~~~~~~~~~~



%~~~~~~~~~~~~~~~~~~~~~~~~~~~~~~~~~~~~~~~~~~~~~~~~~~~~~~~~~~~~~~~
\item[3.6.26] (\textbf{\textit{Rank of AB}}) if $AB = C$, the rows of $C$ are combinations of the rows of \underline{\hspace{8mm}}. So the rank of $C$ is not greater than the rank of \underline{\hspace{8mm}}. Since $B^TA^T = C^T$, the rank of $C$ is also not greater than the rank of \underline{\hspace{8mm}}.\\  
\vspace{3mm}
%~~~~~~~~~~~~~~~~~~~~~ ANSWER TO 3.6.26 ~~~~~~~~~~~~~~~~~~~~~~~~



%~~~~~~~~~~~~~~~~~~~~~~~~~~~~~~~~~~~~~~~~~~~~~~~~~~~~~~~~~~~~~~~
\item[3.6.28] Find the ranks of the 8 by 8 checkerboard matrix $B$ and the chess matrix $C$:\\
\vspace{3mm}
$
$$
B =
\begin{bmatrix}
1&0&1&0&1&0&1&0\\
0&1&0&1&0&1&0&1\\
1&0&1&0&1&0&1&0\\
\cdot &\cdot &\cdot &\cdot &\cdot &\cdot &\cdot &\cdot\\
0&1&0&1&0&1&0&1\\
\end{bmatrix}
$$
$
and 
$
$$
C =
\begin{bmatrix}
r&n&b&q&k&b&n&r\\
p&p&p&p&p&p&p&p\\
&&&four& zero& rows\\
p&p&p&p&p&p&p&p\\
r&n&b&q&k&b&n&r\\
\end{bmatrix}
$$
$\\
\vspace{3mm}
The numbers $r, n, b, q, k, p$ are all different. Find bases for the row space and left nullspace of $B$ and $C$. Challenge problem: Find a basis for the nullspace of $C$.\\
\vspace{3mm}
%~~~~~~~~~~~~~~~~~~~~~ ANSWER TO 3.6.28 ~~~~~~~~~~~~~~~~~~~~~~~~



%~~~~~~~~~~~~~~~~~~~~~~~~~~~~~~~~~~~~~~~~~~~~~~~~~~~~~~~~~~~~~~~

\end{enumerate}
\end{document}